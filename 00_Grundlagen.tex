\section{Motivation}

\subsection{Euler's Polyedersatz}
\label{subsec:0.1}

\begin{prop}[Vermutung]
\label{VermPolyedersatz}
%TODO: Bild, Polyedertypen
Sei $\PP$ ein Polyeder, dann gilt $v-e+f =2$, wobei $v$ (vertex) die Anzahl der Ecken, $e$
(edge) die Anzahl der Kanten und $f$ (face) die Anzahl der Flächen ist.\fishhere
\end{prop}

\ref{VermPolyedersatz} ist sicher falsch, wenn $\PP$  aus mehreren
Stücken (nicht zusammenhängend) zusammengesetzt ist. Von jedem Stück erhält man
einen Beitrag $2$ zur Formel $v-e+f$.

\begin{bspn}
Ein Würfel, aus dessen Innerem ein kleiner Würfel entfernt wurde (keine
zusammenhängende Fläche)\bsphere
%TODO: Bild, Würfel ohne kleinen Würfel.
\end{bspn}

Nehmen, wir also an, dass $\PP$ zusammenhängend ist. Reicht das aus?

\begin{bspn}
%TODO: Polyeder \PP6
$\PP_6$ ist zusammenhängend aber es gibt eine geschlossene Kurve auf der
Oberfläche von $\PP_6$, die diese nicht in zwei separate Flächen teilt. $\PP_6$
ist vergleichbar mit einem Torus (``Autoreifen'').\bsphere
\end{bspn}

\begin{prop}[Satz (Euler 1750)]
\label{prop:Polyedersatz}
Sei $\PP$ ein Polyeder mit folgenden Eigenschaften
\begin{enumerate}
  \item Je zwei Ecken von $\PP$ können durch eine Folge von Kanten verbunden
  werden.
  \item Jeder geschlossene Streckenzug auf der Oberfläche von $\PP$ teilt
  dieselbe in mindestens zwei getrennte Flächen.
\end{enumerate}
Dann ist $v-e+f=2$ für $\PP$.\fishhere
\end{prop}   
Euler bewies diesen Satz 1750 für konvexe Polyeder, Lhuilies 1813 zusätzlich
für Polyeder vom Typ $\PP_5,\PP_6$ und schließlich Staudt 1847 die oben
stehende Proposition.

Hier soll zunächst eine Beweisskizze genügen, dafür benötigen wir noch zwei
Definitionen

\begin{defnn}
\begin{enumerate}[label=(\roman{*})]
  \item Ein \emph{Graph} sind Ecken und Kanten, die die Ecken verbinden.
  \item \emph{Wege} sind Ketten von Kanten so, dass aufeinanderfolgende Glieder der
Kette eine gemeinsame Ecke besitzten.
\item  Ein Graph heißt \emph{zusammenhängend}, wenn je zwei Ecken durch einen Weg
verbunden werden können.
\item Ein \emph{Baum} ist ein zusammenhängender Graph, der keine geschlossenen
Wege enthält.\fishhere
\end{enumerate}
\end{defnn}

\begin{proof}
Man sieht leicht, dass für einen Baum $T$ gilt $v(T) - e(T)=1$.

Für einen Polyeder $\PP$, sei $G = \{\text{Ecken}\}\cup \{\text{Kanten}\}$.
Entfernen wir in $G$ aus geschlossenen Wegen Kanten, erhalten wir einen Baum
$T\subseteq G$.

Sei $\Gamma(T)$ der duale Graph von $T$, d.h. die Ecken von $\Gamma$ sind die
Seitenflächen von $\PP$, dann gelten die folgenden Fakten:
\begin{enumerate}
  \item $\Gamma$ ist zusammenhängend (zwei Ecken von $G$ könnten nur durch
  einen geschlossenen Weg in $T$ getrennt werden, aber $T$ ist Baum).
  \item $\Gamma$ ist Baum, denn ein geschlossener Weg in $\Gamma$ würde wegen
  Bedingung b) in \ref{prop:Polyedersatz} $\PP$ in mindestens zwei Teile
  teilen. Diese könnten dann nicht durch einen Weg in $T$ verbunden werden,
  weil sie durch den geschlossenen Weg in $\Gamma$ gehen müssen. Aber $T$ ist
  zusammenhängend.
\end{enumerate}
Es gilt also,
\begin{align*}
&v(T) = v,\quad v(\Gamma) = f\\
&v(\Gamma)-e(\Gamma) = 1,\quad v(T)-e(T) = 1,\quad e(T)+e(\Gamma) = e\\
\Rightarrow & v-e+f = v(T)-e(T)-e(\Gamma)+v(\Gamma) = 1 + 1 = 2.\qedhere
\end{align*}
\end{proof}

\subsection{Topologische Äquivalenz}
\label{subsec:0.2}

\begin{defn}
Ein \emph{Homöomorphismus} von einer Oberfläche $O_1$ auf eine Oberfläche $O_2$
(in $\R^3$) ist eine bijektive, stetige Abbildung $h: O_1\to O_2$, deren Inverse
ebenfalls stetig ist.\fishhere
\end{defn}

\begin{propn}[Intuitive Idee]
Wir ''deformieren'' Oberflächen stetig. Wir dehnen und stauchen sie aber wir
reißen sie niemals und wir identifizieren nie verschiedene Punkte.\fishhere
\end{propn}

\begin{defn}
Zwei Oberflächen $O_1$ und $O_2$ in $\R^3$ heißen \emph{topologisch äquivalent}
oder homöomorph, wenn es einen Homöomorphismus von $O_1$ auf $O_2$ gibt. In
diesem Fall schreiben wir $O_1\cong O_2$.\fishhere
\end{defn}

Offensichtlich ist $\cong$ eine Äquivalenzrelation.

Nun können wir Euler's Satz allgemeiner formulieren
\begin{lem}
Ein Polyeder $\PP$, der die Bedingungen a) und b) in \ref{prop:Polyedersatz}
erfüllt, ist topologisch äquivalent zur $\S^3$ (Kugeloberfläche).\fishhere
\end{lem}

Später zeigen wir die Umkehrung des obigen Satzes und erhalten damit
den folgenden:
\begin{prop}
Ein Polyeder $\PP$ ist genau dann topologisch äquiavlent zu $\S^3$
wenn $\PP$ zusammenhängend ist und jeder geschlossene Streckenzug auf der
Obefläche von $\PP$ diese in mindestens zwei Teile teilt.\fishhere
\end{prop}

\begin{defn}
Die Zahl $v-e+f$ für einen Polyeder $\PP$ mit $v$ Ecken und $e$ Kanten und $f$
Seiten ist die \emph{Eulerzahl} von $\PP$.
\end{defn}

Folgender Satz ist der Beginn der modernen Topologie.
\begin{prop}
Topologisch äquivalente Polyeder haben dieselbe Eulerzahl.\fishhere
\end{prop}

Die Eulerzahl ist ein Beispiel für eine \emph{topologische Invariante}, d.h.
eine Größe, die für homöomorphe Räume gleich ist. Mit Hilfe von topologischen
Invarianten kann man Räume unterscheiden.

Später werden wir die Eulerzahl für weitaus mehr topologische Räume definieren.
Dazu sollten wir aber erst einmal definieren, was ein topologischer Raum
überhaupt ist.

\subsection{Topologische Räume}
\label{subsec:0.3}

%TODO: Bild, Standardflächen

Wir wollen Oberflächen ''abgesehen von stetigen Deformationen'' klassifizieren.
Wir denken uns Oberflächen aus dehnbarem Material, das aufgeblasen, eingebeult,
gedreht und verzwirbelt werden kann.

Betrachten wir das Möbiusband eingebettet in $\R^3$ stellen wir fest, dass  
kein Homöomorphismus von $\R^3$ in sich existiert, der das einfach getwistete
Möbiusband auf das mehrfach getwistete Möbiusband homöomorph abbildet, obwohl
diese sehrwohl - als topologische Räume - homöomorph sind.

%TODO: Bild, Möbiusband

Daher sind wir daran interessiert, Oberflächen unabhängig von der Einbettung
in den euklidischen $\R^3$ als topologische Räume durch interne Eigenschaften zu
definieren. So sind einfach und mehrfach getwistete Möbuisbänder derselbe
topologische Raum, nur verschieden eingebettet in den $\R^3$.

Dafür benötigen wir eine abstrakte Definition eines topoligschen Raumes. Mit
``abstrakt'' ist eine Definition gemeint, die alle Fälle umfasst, jedoch nicht
zu abstrakt ist, sondern noch genügend Informationen enthält, so dass
Stetigkeit definiert werden kann.

\begin{defn}[Vorläufige Definition]
\label{defn:0.3.1}
Ein \emph{topologischer Raum} ist eine Menge $X$, so dass für jedes $x\in X$ ein
nichtleeres System $\UU_x \subseteq \PP(X)$ von Teilmengen von $X$ existiert,
dessen Elemente \emph{Umgebungen von $x$ in $X$} heißen und folgenden Axiomen
genügen:
\begin{enumerate}
  \item $A\in \UU_x \Rightarrow x\in A$
  \item $A,B\in \UU_x \Rightarrow A\cap B \in \UU_x$
  
Endlich viele Durchschnitte von Umgebungen von $x$ sollen wieder Umgebungen von
$x$ sein.
  \item $A\in \UU_x, B\subseteq X : A\subseteq B \Rightarrow B\in \UU_x$
  
  Obermengen von Umgebungen sind Umgebungen.
  \item $A\in \UU_x \Rightarrow A^\circ = \setdef{z\in A}{A\in \UU_z}\in \UU_x$.
\end{enumerate}

Die Menge $A^\circ = \setdef{z\in A}{A\in \UU_z}$ heißt \emph{Inneres} oder
\emph{offener Kern} von $A$. Die Abbildung $\TT: X\to \PP\PP(X),\;x\mapsto
\UU_x$ heißt \emph{Topologie} auf $X$.\fishhere
\end{defn}

\begin{bemn}
$\UU_x\subseteq \PP(X)$, d.h. $\UU_x\in \PP\PP(X)$.\maphere
\end{bemn}

Nun wollen wir Abbildungen zwischen topologischen Räumen betrachten. Damit wir
Umgebungen in Bild- und Urbildraum unterscheiden können, bezeichnen wir das
Umgebungssystem $\UU_x$ in $X$ im Folgenden mit $\UU_x(X)$.

\begin{defn}
\label{defn:0.3.2}
Seien $X,Y$ topologische Räume, $f: X\to Y$ eine Abbildung, dann heißt $f$
\emph{stetig} in $x\in X$, falls
\begin{align*}
f^{-1}(N)\in \UU_x(X),\quad \forall N\in \UU_{f(x)}(Y).\fishhere
\end{align*}
\end{defn}

\begin{bsp}
\label{bsp:0.3.3}
\begin{enumerate}
  \item Sei $X$ Menge, dann ist $\UU_x = \{X\}\subseteq \PP(X)$ die
  \emph{gröbste} Topologie, die $X$ hat.
  \item Sei $X$ Menge, dann ist $\UU_x = \setdef{A\subseteq X}{x\in A}$ die
  \emph{feinste} Topologie, die $X$ hat.\bsphere
\end{enumerate}
\end{bsp}

\begin{defn}
\label{defn:0.3.4}
Sei $X$ ein topologischer Raum, $Y\subseteq X$. Wir machen $Y$ zum
topologischen Raum durch folgenden Regel:

Sei $y\in Y$, dann ist $\UU_y(Y) = \setdef{A\cap Y}{A\in \UU_y(X)}$.\fishhere
\end{defn}

Umgebungen in $Y$ entstehen als Schnitt von Umgebungen in $X$ mit $Y$. Dass wir
hier überhaupt von einem topologischen Raum sprechen können, zeigt das folgende
Lemma

\begin{lem}
\label{prop:0.3.5}
Sei $X$ topologischer Raum, $Y\subseteq X$. Dann definiert $\UU_y(Y)$ für $y\in
Y$ eine Topologie auf $Y$, die \emph{Unterraum-} oder \emph{Spurtopologie von
$y$ in $X$} gennant wird.

Trägt $Y\subseteq X$ die Spurtopologie bzgl. $X$, so heißt $Y$
\emph{(topologischer) Unterraum} von $X$.\fishhere
\end{lem}

\begin{bspn}
\begin{enumerate}
  \item Sei $Q$ ein Quader, der aus der $\S^3$ Spähre herausragt,
  dann ist $\UU_y(\R^3)$ 3-dimensional aber $\UU_y(\S^2)$ nur noch
  2-dimensional.
  \item Es gilt $A\in \UU_x$ genau dann, wenn $x\in A^\circ$. Für den $\R^n$
  ist
  \begin{align*}
  A^\circ = \setdef{z\in\R^3}{\exists \delta > 0 : U_\delta(z)\subseteq A}
  \end{align*}
  und $U_\delta = \setdef{y\in\R^n}{\norm{x-y} < \delta}$ der bekannte
  $\delta$-Ball.\bsphere
\end{enumerate}
\end{bspn}

\begin{bemn}[Vorsicht:]
Ist $Y\subseteq X$ ein Unterraum, dann ist $U\in \UU_y(Y) \Rightarrow U\in
\UU_y(X)$ im Allgemeinen falsch. Die Umkehrung $U\in \UU_y(X), U\subseteq Y
\Rightarrow U\in \UU_y(Y)$ gilt aber.\maphere
\end{bemn}

\begin{lem}
\label{prop:0.3.6}
Sei $X$ ein topologischer Raum, $Y\subseteq X$ Unterraum, dann ist die
natürliche Inklusion $\iota: Y\to X$ von $Y$ in $X$ stetig.

In der Tat ist die Spurtopologie die gröbste Topologie (d.h. die Topologie mit
den wenigsten Umgebungen) für die die Einbettung von $\iota$ stetig
ist.\fishhere
\end{lem}

\begin{propn}
Jede Oberfläche im $\R^3$ ist ein topologischer Raum, wenn wir sie mit der
Spurtopologie versehen.\fishhere
\end{propn}

\begin{defn}
\label{prop:0.3.7}
Zwei topologische Räume $X$ und $Y$ heißen \emph{topologisch äquivalent}, falls
ein Homöomorphismus $f: X\to Y$ existiert.\fishhere
\end{defn}

\begin{defn}
\label{prop:0.3.8}
Eine \emph{Fläche} (``Oberfläche'') ist ein topologischer Raum in welchem jeder
Punkt (= Element) eine zum Einheitsvollkreis $D^2 = \setdef{x\in\R^n}{\norm{x} \le
1}$ homöomorphe Umgebung besitzt und für die je zwei Punkte auch verschiedene
Umgebungen haben.\fishhere
\end{defn}

\subsection{Klassifikation von Flächen}
\label{subsec:0.4}

Wir wollen eine Liste von Prototypen von Flächen angeben, die paarweise nicht
homöomorph sind, so dass jede Fläche homöomorph zu genau einem dieser
Prototypen ist.

Dabei beschränken wir uns auf Flächen mit den Eigenschaften
\begin{enumerate}
  \item zusammenhängend,
  \item kompakt,
  \item randlos.
\end{enumerate}

Es gibt genau zwei Prototypen

\begin{enumerate}[label=\arabic{*}. Typ]
  \item Eine Spähre mit $n$ Henkeln $(n\in\N)$
  (orientierbare Flächen).
  \item Schneide aus einer Spähre $n$-Scheiben $(n\in\N)$ und
  ersetzte die entstehenden Löcher durch Möbiusbänder, indem der Rand eines Möbiusbandes mit
  dem Rand eines Loches identifiziert wird (nicht orientierbare Flächen).
  
  $n=1$ projektive Ebene,
  
  $n=2$ kleinsche Flasche
\end{enumerate}

\begin{prop}
\label{prop:0.4.1}
Jede zusammenhängende kompakte Fläche ohne Rand ist homöomorph zu einer Fläche
vom Typ 1 (orientierbar) oder Typ 2 (nicht orientierbar).\fishhere
\end{prop}

Möbius hat diese Aussage für Typ 1 bereits 1861 bewiesen.

\subsection{Philosophischer Exkurs}
\label{subsec:0.5}

Das höchste Ziel wäre eine Klassifikation aller topologischer Räume, doch dies
ist hoffnungslos. Zwar ist es möglich, ein Konstruktionsverfahren anzugeben mit
dem man alle nicht homöomophen topologischen Räume erhält. Es stellt sich
jedoch oft als äußert kompliziert heraus, zwei topologische Räume auf
Homöomorphie zu vergleichen. Später werden wir sehen, dass wir dieses Problem
auf Invarianten zurückführen können aber auch hier ist es oft schwierig, diese
für einen konkret gegebenen Raum zu bestimmen.

Wir sind also mit zwei Problemen konfrontiert
\begin{enumerate}[label=\arabic{*}.)]
  \item Finde alle nichthomöomorphen topologischen Räume. (Prototypen)
  \item Gegeben ist ein topologischer Raum $X$, zu welchem Prototyp ist dieser
  homöomoph? (Wiedererkennungsproblem)
\end{enumerate}

Um diese Probleme zu lösen geht man wie folgt vor
\begin{enumerate}[label=zu \arabic{*}.)]
  \item Einschränkungen (kompakte Räume, kompakte bzw. zusammenhängende
  Flächen) können die Schwierigkeit alle zu finden reduzieren.
  \item Topologische Invarianten, d.h. Größen, die unter Homöomorphismen
  erhalten bleiben z.B. Eulerzahl, Zusammenhang oder Fundamentalgruppen.\\
  Wesentliche Eigenschaften der Räume lassen sich durch die Struktur der
  Gruppen beschreiben. Homoemorphe, zusammenhängende Räume haben beispielsweise
  isomorphie Fundamentalgruppen.
\end{enumerate}

Folgenden Satz werden wir später beweisen:
\begin{prop}
\label{prop:0.5.1}
Die Flächen von Typ 1 und Typ II in \ref{subsec:0.4} haben alle verschiedene
Fundamentalgruppen und sind daher paarweise nicht homöomorph.\fishhere
\end{prop}
