\section{Kompakte Flächen}

\begin{bemn}[Erinnerung:]
Eine $n$-dimensionale (topologische) Mannigfaltigkeit ist ein Hausdorffraum,
der lokal zum $\R^n$ homöomorph ist, d.h. jeder Punkt besitzt eine Umgebung,
die zum $\R^n$ homöomorph ist.\maphere
\end{bemn}

Wir wissen bereits, dass jede jede $n$-Mannigfaltigkeit lokal kompakt ist.
Unser Ziel ist es, alle zusammenhängenden, kompakten randlosen Flächen
(=$2$-Mannigfaltigkeiten) zu finden.
\begin{bemn}
Punkte von ``Flächen''$\subseteq\R^2$ können keine zum $\R^2$ homöomorphe
Umgebung besitzen:
%TODO: Bild, Umgebung
d.h. die Zylinderfläche ist keine Fläche!
\end{bemn}
\begin{bemn}[1. Annäherung:]
Finde leicht zu behandelnde Beispielflächen: Simplizialkomplexe sind 
topologische Invarianten, wie z.B. die Fundamentalgruppe, die leicht zu
berechnen sind.
\end{bemn}

Wir wollen ein konkretes Resultat erzielen, z.B. die Klassifikation der Flächen.

Unser Problem ist allerdings, dass die topologische Struktur alleine nicht viel
hergibt, um ein solches Ergebnis zu beweisen. Auch algebraische  Invarianten
nützen nicht viel, es sei denn, wir haben eine Methode zur Berechnung dieser
Invarianten für eine genügend große Kollektion solcher Räume.\\
Wir arbeiten also mit Räumen, die in Stücke zerlegt werden können, die wir
schon hinreichend gut kennen, und die man schön zusammensetzen kann,
\emph{triangulierbare Räume}.

\begin{bspn}
\begin{enumerate}[label=\arabic{*}.)]
  \item %TODO: Bild
Ein Homöomorphismus von der Oberfläche eines Tetraeders auf die 2-Spähre ergibt
eine Zerlegung derselben in 4 Dreiecke, die an ihren Kanten verschweißt sind.
\item %TODO: Triangulierung des Möbisubandes
Wir können Flächen durch Dreiecke triangulieren. Für Mannigfaltigkeiten höherer
Dimension brauchen wir entsprechende Bausteine höherer Dimension.\bsphere
\end{enumerate}
\end{bspn}

\subsection{Simpliziale Komplexe und Triangulierungen}

\begin{defn}
\label{defn:4.1.1}
Eine \emph{(affine) Hyperebene} $H$ des $\R^n$ der Dimension $k$ ist von der
Form,
\begin{align*}
v_0 + U = H,\; U\leqslant \R^n,\; \dim_\R (U) = k.
\end{align*}
Ist $(x_1,\ldots,x_n)$ Basis von $U$ und $v_i = v_0+x_i\in H$ (für
$i=1,\ldots,k$), so ist
\begin{align*}
H &= \setdef{v_0+\lambda_1(v_1-v_0)+\ldots+\lambda_k(v_k-v_0)}{\lambda_i\in\R}
\\ &= \setdef{\mu_0v_0+\mu_1v_1+\ldots+\mu_k v_k}{\mu_i\in\R,
\sum\limits_{i=0}^k \mu_i = 1}
\end{align*}
Seien $v_0,v_1,\ldots,v_k\in\R^n$, dann ist
$H=\setdef{\sum\limits_{i=0}^{k}\mu_iv_i}{\sum\limits_{i=0}^k
\mu_i=1,\;\mu_i\in\R}$ die von den Vektoren $v_0,\ldots,v_k$ aufgespannte
\emph{Hyperebene}. Die Punkte $v_0,\ldots,v_k$ heißen \emph{in allgemeiner
Position}, falls jede echte Teilmenge dieser Punkte eine echt kleinere
Hyperebene aufspannt. Dies ist geanu dann der Fall, wenn die Vektoren
$v_1-v_0,v_2-v_0,\ldots,v_k-v_0$ linear unabhängig sind.

Seien $v_0,v_1,\ldots,v_k\in\R^n$. Dann ist die \emph{kleinste konvexe
Teilmenge} vom $\R^n$, die $v_0,\ldots,v_k$ enthält, gegeben als,
\begin{align*}
\setdef{\mu_0v_0+\mu_1v_1+\ldots+\mu_kv_k}{\mu_i\in\R,\;\sum\limits_{i=0}^k
\mu_i = 1,\;\mu_i\ge 0}.
\end{align*}
Sie heißt die \emph{konvexe Hülle} der Punkte $v_0,\ldots,v_k$.
\begin{bspn}
%TODO: Bild, Konvexe Hülle,
\begin{align*}
&T = \setdef{\mu_0x_0+\mu_1x_1+\mu_2x_2}{\mu_i\in\R, \mu_0+\mu_1+\mu_2 = 1},\\
&S = \setdef{\mu_0x_0+\mu_1x_1+\mu_2x_2 \in T}{\mu_i \ge 0, \mu_0+\mu_1+\mu_2 =
1}.
\end{align*} 
\end{bspn}
Sind zudem $v_0,v_1,\ldots,v_k$ in allgemeiner Position, so heißt ihre konvexe
Hülle das von $v_0,\ldots,v_k$ aufgespannte \emph{Simplex} $\sigma$, oder auch
$k$-Simplex, wobei $k$ die Dimension des Simplex' ist.
\begin{bspn}
$0$-Simplex (Punkt), $1$-Simplex (Strecke), $2$-Simplex (Dreieck), $3$-Simplex
(Tetraeder)
%TODO: Bild der Simplexe.
\end{bspn}
Die Punkte $v_0,\ldots,v_k$ heißen \emph{Ecken} von $\simp$. (Punkte sind
$0$-Simplizes)

Simplizes haben \emph{Seiten}, nämlich die von den Teilmengen der Menge der
Ecken aufgespannten Simplizes.

Sei $\tau$ eine Seite von $\simp$, d.h. $\tau$ wird von
$A\leqslant\{v_0,\ldots,v_k\}$ aufgespannt (und ist Simplex), dann schreiben
wir $\tau\leqslant\simp$.\fishhere
\end{defn}

\begin{defn}
\label{defn:4.1.2}
Eine endliche Menge $K$ von Simplizes im $\R^n$ heißt \emph{simplizialer
Komplex (Simplizalkomplex)}, wenn gilt:
\begin{enumerate}[label=(\roman{*})]
  \item Für jedes $\simp\in K$ und $\tau\leqslant\simp$ ist $\tau\in K$.
  \item Sind $\simp$ und $\tau$ in $K$, so ist $\simp\cap\tau$ entweder leer
  oder besteth aus einer gemeinsamen Seite von $\simp$ und $\tau$.\fishhere
\end{enumerate}
\end{defn}
\begin{bspn}
\begin{enumerate}[label=\arabic{*}.)]
  \item $K=\{\simp, \overline{x_1x_2}, \overline{x_0x_1}, \overline{x_0x_2},
 x_0,x_1,x_2\}$.
%TODO: Bild, Simplex
 \item $K=\{\simp, \overline{x_1x_2}, \overline{x_0x_1}, \overline{x_0x_2},
 x_0,x_1,x_2, x_3, \overline{x_1x_3},\overline{x_1x_2},\overline{x_2x_3}\}$
 %TODO: Bild, Simplex
 \item Keine Simplizialkompexe.\bsphere
 %TODO: Bild, keine Simpl.
\end{enumerate}
\end{bspn}

\begin{bem}
\label{bem:4.1.3}
\begin{enumerate}[label=\arabic{*}.)]
  \item Man kann (abstrakt) einen $k$-Simplex als Menge $\sigma
  =\{v_0,v_1,\ldots,v_k\}$ auffassen. Der zu $\sigma$ gehörende
  Simplizialkomplex ist dann gerade $\PP(\sigma)\setminus\{\varnothing\}$. Dies
  ist der Vorteil beim Aufbau durch Dreiecke, denn für z.B. Rechtecke gilt dies
  nicht
  %TODO: Bild, Rechteck
  die Diagonale $\{x_0,x_1\}$ spannt keine Seite auf.
  \item Ebenso kann man abstrakte (kombinatorische) Simplizialkomplexe $K$
  durch Angabe einiger Teilmengen von einer endlichen Punktmenge angeben:
  \begin{align*}
  &K_0 = \{v_0,\ldots,v_m\},\; \sigma_1,\ldots,\sigma_l\subseteq K_0
  \text{ maximale Simplizes},\\
  \sim & K=\setdef{v_0,\ldots,v_m,\;\sigma_1,\ldots,\sigma_l}{\tau\in K_0 \text{
  mit }\tau \subseteq \sigma_i \exists i=1,\ldots,l}.\maphere
  \end{align*}
\end{enumerate}
\end{bem}

Eine Einbettung von $K$ in den $\R^n$ erhält man, indem man $m+1$ Punkte
$v_0,\ldots,v_m$ in allgemeiner Position im $\R^n$ nimmt und die von den
$\sigma_i, i=1,\ldots,l$ aufgespannten Simplizes und ihre Seiten dazu nimmt.
Der so entstehende topologische Raum (Spurtologie im $\R^n$) heißt
\emph{Realisierung} von $K$ und wird mit $\abs{K}$ bezeichnet.

Alle Realisierungen eines Simplizialkomplexes sind homöomorph.

\begin{defn}
\label{defn:4.1.4}
Eine \emph{Triangulierung} eines topologischen Raumes $X$ besteht aus einem
Simplizialkomplex $K$ und einem Homöomorphismus $h: \abs{K}\to X$. $X$ heißt
\emph{triangulierbar}, falls $X$ eine Triangulierung besitzt.\fishhere
\end{defn}
\begin{bemn}[Beachte:]
\begin{enumerate}[label=\arabic{*}.)]
  \item $\abs{K}$ in \ref{defn:4.1.4} ist kompakt und metrisierbar, also ist
  auch $X$ kompakt und metrisierbar.
  \item Triangulierungen von $X$, falls sie existieren, sind nicht eindeutig!
\end{enumerate}
\end{bemn}

\begin{bspn}
\begin{enumerate}[label=\arabic{*}.)]
  \item Zwei verschiedene Triangulierungen der $2$-Sphäre.
%TODO: Bild, Triangulierung 2-Sphäre
\item Triangulierungen des Kreiszylinders und des Trous.\bsphere
%TODO: Bild, Triangulierungen des Kreiszylinders und des Trous
\end{enumerate}
\end{bspn}

\begin{propn}[Sätze]
\begin{enumerate}[label=\arabic{*}.)]
  \item Rado 1923:

Jede kompakte Fläche ist triangulierbar.
\item Moise 1952:

Jede kompakte $3$-dimensionale Mannigfaltigkeit ist triangulierbar.
\item Cairns, 1935:

Jede kompakte, differenzierbare Mannigfaltigkeit ist triangulierbar.\fishhere
\end{enumerate}
\end{propn}
\begin{propn}[Offenes Problem]
Ist jede kompakte topologische Mannigfaltigkeit der Dimension $>3$
triangulierbar?\fishhere
\end{propn}

\begin{defn}
\label{defn:4.1.5}
Für einen Simplex $\sigma=\{x_0,\ldots,x_k\}$ mit Ecken $x_0,\ldots,x_k$ sei
die Menge
\begin{align*}
\setdef{\sum\limits_{i=0}^k \lambda_i x_i}{\sum\limits_{i=0}^k \lambda_i = 1,
\lambda_i\in\R, \lambda_i > 0, \forall i=0,\ldots,k},
\end{align*}
das \emph{Innere} von $\abs{\sigma}\subseteq\R^n$.\fishhere
\end{defn}
Für $k\neq n$ unterscheidet sich das Innere vom topologischen Innern der
Teilmenge $\abs{\sigma}\subseteq \R^n$.
\begin{defnn}
\emph{Zusammenhang} bei Simplizialkomplexen ist das, was man sich
geometrisch darunter vorstellt.\fishhere
\end{defnn}
\begin{bspn}
\begin{enumerate}[label=\arabic{*}.)]
  \item
  $K=\{a,b,c,\ldots,k, \lin{a,b},\lin{a,c},\ldots, 
 \lin{a,b,c},\lin{b,c,d},\ldots,\lin{i,j,k,l}$
%TODO: Bild, zusammenhängeder Simplizialkomplex
\item $K\{a,\ldots\}$
%TODO: Bild, nicht zusammenhängder Simplizialkomplex.
$K$ ist nicht zusammenhängend.\bsphere 
\end{enumerate}
\end{bspn}

\begin{lem}
\label{prop:4.1.6}
Sei $K$ Simplizialkomplex mit Polyeder $\abs{K}\subseteq\R^n$ dann gilt:
\begin{enumerate}[label=\arabic{*}.)]
  \item $\abs{K}$ ist abgeschlossen und beschränkt, also kompakt im $\R^n$.
  \item Jeder Punkt $x$ von $\abs{K}$ liegt im Innern genau eines Simplexes von
  $K$. Dieses Simplex wird Träger von $x$ genannt.
  \item $\abs{K}=\bigcup_{\sigma\in K} \abs{\sigma}$ trägt die
  Quotiententopologie bezüglich der Einbettungen $\abs{\sigma}\opento\abs{K}$, $\sigma\leqslant
  \abs{K}$.
  \item Ist $K$ zusammenhängend, so ist $\abs{K}$ wegzusammenhängend.\fishhere
\end{enumerate}
\end{lem}
\begin{proof}
Übung.\qedhere
\end{proof}

\begin{defn}
\label{defn:4.1.7}
Seien $K,L$ Simplizialkomplexe mit Eckenmenge $K_0$ bzw. $L_0$. Eine Abbildung,
\begin{align*}
\ph: K_0\to L_0,
\end{align*}
heißt \emph{simplizial}, falls gilt:

Spannt $\{v_0,\ldots,v_k\}\subseteq K_0$ einen Simplex in $K$ auf, so spannt
$\{\ph(v_0),\ldots,\ph(v_k)\}$ einen Simplex in $L$ auf. ($\ph$ muss nicht
injektiv sein!).

Ist $\ph$ eine Bijektion und sind $\ph$ und $\ph^{-1}$ simplizial, so heißt
$\ph$ \emph{Isomorphismus} von $K$ auf $L$.\fishhere
\end{defn}
 
\begin{bemn}
Seien $K,L$ Simplizialkomplexe und $\ph: K\to L$ simpliziale Abbildung.
\begin{enumerate}[label=\alph{*})]
  \item $\ph$ ist eindeutig durch die Bilder der Ecken von $K$ bestimmt.
  \item Sind $\sigma,\tau\in K$ und $\tau\leqslant\sigma$, dann ist
  $\ph(\tau)\leqslant\ph(\sigma)$ in $L$.
  \item Für $\sigma\in K$ ist $\dim \sigma \ge \dim\ph(\sigma)$.
\end{enumerate}
Wir haben eine Kategorie der Simplizialkomplexe mit den simplizialen
Abbildungen also Morphismen. ($\id_K: K\to K$ ist simplizial und die
Komposition von simplizialen Abbildungen ist wieder eine simpliziale
Abbildung.)\maphere
\end{bemn}
\begin{bemn}[Bemerkungen.]
\begin{enumerate}[label=\arabic{*}.)]
  \item Sei $\ph: K_0\to L_0$ simplizial.

Dann lässt sich $\ph$ in natürlicher Weise zu einer Abbildung $\ph: K\to L$
erweitern, durch:
Ist $\Delta = \{v_0,\ldots,v_k\}\in K$, d.h. $\Delta$ ist Simplex in $K$, so
sei $\ph(\Delta) = \{\ph(v_0),\ldots,\ph(v_k)\}\in L$.
\item Seien $\abs{K}$ und $\abs{L}$ die zugehörigen Polyeder im $\R^n$. Sei
$\ph: K\to L$ simplizial und $x\in\abs{K}$, dann gibt es einen Simplex
$\{v_0,\ldots,v_k\}$ in $K$ mit $x=\sum\limits_{i=0}^k \lambda_i v_i$ wobei
$\lambda_i\in\R,\;\sum \lambda_i = 1,\;\lambda_i > 0$.

Definiere $s(x) = \sum\limits_{i=0}^k \lambda_i \ph(v_i)\in\abs{L}$. Dann wird
$s: \abs{K}\to\abs{L}$ Abbildung.

Diese ist wohldefiniert, da jedes $x$ nur im Innern von genau einem Simplex
liegt. Klar ist auch, dass $s$ stetig und linear auf den $\abs{\Delta}$ mit
$\Delta \in K$ ist.

In Zukunft wollen wir daher nicht mehr zwischen
\begin{align*}
s: \abs{K}\to\abs{L} \text{ und } \ph: K\to L,
\end{align*}
unterscheiden.\maphere
\end{enumerate}
\end{bemn}
\begin{lem}
\label{prop:4.1.8}
Seien $K,L$ Simplizialkomplexe und sei $\ph: K\to L$ Isomorphismus. Dann ist
die zugehörige simpliziale Abbildung $s: \abs{K}\to\abs{L}$ ein
Homöomorphismus.\fishhere
\end{lem}
\begin{proof}
Übung.\qedhere
\end{proof}

\begin{bemn}
Es kann durchaus passieren, dass $\abs{K}$ und $\abs{L}$ homöomorph aber nicht
isomorph sind.\maphere
%TODO: Bild, K,L
\end{bemn}
\begin{bemn}[Bemerkungen.]
\begin{enumerate}[label=\arabic{*}.)]
  \item Es gibt viel mehr stetige Abbildung als simpliziale Abbildungen von
  $\abs{K}$ nach $\abs{L}$.

Ist $m$ bzw. $n$ die Anzahl der Ecken von $K$ bzw. $L$, so gibt es höchstens
$m^n$ simpliziale Abbildungen von $K$ nach $L$.
\item Sei $\ph: K\to L$ simplizial, $\sigma\in K$, $\tau=\ph(\sigma)\in L$,
dann bildet $s$, $\abs{\sigma}$ auf $\abs{\tau}$ ab.

Ist $\dim\sigma = \dim \tau$, so ist $s\big|_{\abs{\sigma}}:
\abs{\sigma}\to\abs{\tau}$ eine bijektive affine Abbildung.

Ist $\dim\sigma > \dim \tau$, so ``drückt'' $s$, $\abs{\sigma}$ auf
$\abs{\tau}$ zusammen.

Den Fall $\dim\sigma <\dim\tau$ gibt es nicht (siehe \ref{defn:4.1.7}).\maphere
\end{enumerate}
\end{bemn}

%\begin{bspn}
%TODO: Bilder, simplizale Abbildungen.
%\end{bspn}

\subsection{Die Kantengruppe eines Simplizialkomplexes}

Sei $X$ wegzusammenhängender, triangulierbarer Raum, etwa mit Triangulierung
$h: \abs{K}\to X$, $K$ Simplizialkomplex. Da nach Voraussetzung $h$ ein Homöomorphismus
ist, induziert $h$ einen Isomorphismus $h_* : \Pi(\abs{K})\to \Pi(X)$.

Die Fundamentalgruppe von $\abs{K}$ ist aber besonder schön zu berechnen:
Wir können annhehmen, dass alle Schleifen an eine Ecke von $\abs{K}$ über
Kanten (=1-Simplizes von $K$) laufen.

\begin{defn}
\label{defn:4.2.1}
Sei $K$ ein $S$-Komplex. Ein \emph{Kantenzug} in $K$ ist eine Folge
$v_0,v_1,\ldots,v_k$ von Ecken in $K$ ($k\in\N_0$), so dass aufeinanderfolgende
Ecken $v_i,v_{i+1}$ entweder gleich sind oder durch eine Kante in $K$
verbunden sind.

Ein \emph{geschlossener Kantenzug} erfüllt zusätzlich $v_0 = v_k$ und wird mit
$vv_1\ldots v_{k-1}v$ bezeichnet.

Auf der Menge der Kantenzüge an der Ecke $v_0$ definieren wir eine
Äquivalenzrelation $\sim$ wie folgt:
%TODO: Bild, Äquivalenzrelation Kantenzüge

Die von diesen Relationen erzeugte Äquivalenzrelation wird ebenfalls mit $\sim$
bezeichnet. Die Äquivalenzklasse bzgl. ``$\sim$'' von dem geschlossenen
Kantenzug $vv_1\ldots v_{k-1}v$ bezeichnen wir mit $\nrm{vv_1\ldots v_{k-1}v}$.

Wir definieren eine Multiplikation auf der Menge der Äquivalenzklassen durch,
\begin{align*}
\nrm{v_0v_1\ldots v_kv_0}\nrm{v_0w_1\ldots w_mv_0} = \nrm{v_0v_1\ldots
v_kw_1\ldots w_m v_0}.
\end{align*}
Diese binäre Operation ist offensichtlich wohldefiniert und assoziativ.

Der Kantenzug $\nrm{v_0}=\nrm{v_0v_0}=\nrm{v_0v_1v_0}$ ist die Identität und
$\nrm{v_0 v_k v_{k-1} \ldots v_1 v_0}$ das Inverse $\nrm{v_0 v_1 \ldots v_{k-1}
v_0}^{-1}$ von $\nrm{v_0 v_1 \ldots v_{k-1} v_0}$.

Damit wird die Menge der Äquivalenzklassen von geschlossenen Kantenzügen an
$v_0$ eine Gruppe $E(K,v_0)$, die \emph{Kantengruppe} an $v_0\in K$.\fishhere
\end{defn}

Wir wollen zeigen, dass die Abbildung,
\begin{align*}
&E(K,v_0)\to\Pi(\abs{K},v_0),\; \nrm{v_0v_1\ldots v_{k}v_0}\to \lin{\text{Weg}
(v_0v_1\ldots v_kv_0)},
\end{align*}
ein Isomorphismus ist. 

\begin{defn}
\label{defn:4.2.2}
Die erste \emph{baryzentrische Unterteilung} $K^1$ von einem Simplizialkomplex $K$ ist
definiert wie folgt:

Jedes einzelne $q$-Simplex $\sigma^q = \lin{v_0,v_1,\ldots,v_q}, q\in\N_0$ wird
enzeln unterteilt, wobei die Ecken von $(\sigma^q)^1\in K^1$ alle Schwerpunkte
$\widehat{v_{j_0}\ldots v_{j_i}}$ der $i$-dimensionalen Seiten von $\sigma^q$
sind.

Hierbei ist der \emph{Schwerpunkt} des $q$-Simplex $\sigma$ bezeichnet als
$\hat{\sigma}$, der Punkt mit den baryzentrischen Koordianten $\lambda_i=
\frac{1}{1+q},\; i=0,\ldots,q$.

D.h. ist $\sigma =\lin{x_0,x_1,\ldots,x_q}$, dann ist $\hat{\sigma} =
\frac{1}{1+q}(x_0+x_1+\ldots+x_q)$.\fishhere
\end{defn}
\begin{bspn}
%TODO: Bild, Dreieck baryzentrische Unterteilung
\begin{align*}
\frac{1}{3}(x_0+x_1+x_2) = x_0+ \underbrace{\frac{1}{3}(x_1-x_0)}_{:=a} +
\underbrace{\frac{1}{3}(x_2-x0)}_{:=b}.\bsphere
\end{align*}
\end{bspn}

\begin{propn}
Die Schwerpunkte einer Kette
\begin{align*}
\sigma^0 < \sigma^1 < \ldots <\sigma^q,
\end{align*}
von $i$-Simplizes $\sigma^i$ ($i=0,\ldots,q$) gegeben als
\begin{align*}
\hat{\sigma}^0, \hat{\sigma}^1, \ldots, \hat{\sigma}^q,
\end{align*}
sind in allgemeiner Position und spannen einen $q$-Simplex auf.\fishhere
\end{propn}
\begin{proof}
Übung.\qedhere
\end{proof}

Daher ist ein $k$-dimensionaler Simplex in der Unterteilung $(\sigma^q)^1$ von
$\sigma^q$ bestimmt durch die Ecken,
\begin{align*}
\{\widehat{v_{j_0}}, \widehat{v_{j_0}v_{j_1}},
\widehat{v_{j_0}v_{j_1}v_{j_2}},\ldots,\widehat{v_{j_0}v_{j_1}\ldots v_{j_k}}\},
\end{align*}
für jedes geordnete Tupel $(v_{j_0}, \ldots, v_{j_k})$ in $\{v_0,\ldots,v_q\}$.

Der $q$-Simplex $\sigma^q$ wird dadurch in $(q+1)!$ viele $q$-Simplizes zerlegt.
%TODO: Bild, Baryzentrische Zerlegung.

\begin{bemn}[Klar]
$\abs{\sigma^q}$ ist die Vereinigung der $q$-Simplizes in $\abs{(\sigma^q)^1}$,
\begin{align*}
\Rightarrow\; & \abs{\sigma^q} = \abs{(\sigma^q)^1}\\
\Rightarrow\; &\abs{K} = \abs{K^1} \text{ als Teilmenge des }\R^n.
\end{align*}
Induktiv erhalten  wir die $m$-te baryzentrische Unterteilung $K^m$ von $K$.
\end{bemn}

\begin{defn}
\label{defn:4.2.3}
Seien $K$ und $L$ Simplizialkomplexe, $f: \abs{K}\to\abs{L}$ eine stetige Abbildung.
Eine \emph{simplizale Approximation} von $f$ ist eine simpliziale Abbildung
\begin{align*}
s: K\to L,
\end{align*}
für die gilt, dass $s(x)$ im Träger von $f(x)\in L$ liegt für alle
$x\in\abs{K}$.\fishhere
\end{defn}

\begin{bspn}
Seien $K=L=[0,1]$,
\begin{align*}
&K=\left\{0,\frac{1}{3},1,\lin{0,\frac{1}{3}},\lin{\frac{1}{3},1}\right\},\\
&L=\left\{0,\frac{2}{3},1,\lin{0,\frac{2}{3}},\lin{\frac{2}{3},1}\right\}.
\end{align*}
%TODO: Bild, S-Approx
$f: \abs{K}\to\abs{L},\; x\mapsto x^2$ ist stetig.

Sei $s$ simpliziale Approximation von $f$, dann ist
\begin{align*}
&f(0) = 0 = s(0),\\
&f(1) = 1 = s(1),
\end{align*}
und daher $s\left(\frac{1}{3}\right) = \frac{2}{3}$ (und
$s\left(\frac{\sqrt{2}}{\sqrt{3}}\right) = \frac{2}{3}$)
(Wäre z.B. $s\left(\frac{1}{3}\right) = 0$, so wäre
$s\left(\lin{\frac{1}{3},1}\right) = \lin{0,1} \notin L$)

Nun ist aber $f\left(\frac{1}{2}\right) = \frac{1}{4}\in
\lin{0,\frac{2}{3}}\ni s\left(\frac{1}{2}\right)$,
da $s\left(\frac{1}{2}\right) \in s\left(\lin{\frac{1}{3},1}\right) \in
\lin{\frac{2}{3},1}$ sein muss.

Also kann $s$ nicht simplizial sein und daher finden wir keine simpliziale
Approximation von $f$.

\begin{bemn}[Ähnlich:]
Keine simpliziale Abbildung $s: K^1 \to L$ approximiert $f$ simplizial.
\end{bemn}
Aber wir finden eine simpliziale Abbildung $s: K^2\to L$:
%TODO: Bild, S-Approx Bar-U
\begin{align*}
0,\frac{1}{12},\frac{1}{6}, \frac{1}{4},\frac{1}{3},\frac{1}{2}&\mapsto
0,\\
\frac{2}{3},\frac{5}{6}&\mapsto \frac{2}{3},\\
1&\mapsto 1,
\end{align*}
also ist $\sigma: \abs{K^2}\to \abs{L}$ gegeben durch,
\begin{align*}
&\nrm{0,\frac{1}{2}} \to 0,\\
&\nrm{\frac{1}{2},\frac{2}{3}} \to \nrm{0,\frac{2}{3}},\\
&\nrm{\frac{2}{3},\frac{5}{6}} \to \frac{2}{3},\\
&\nrm{\frac{5}{6},1} \to \nrm{\frac{2}{3},1}.
\end{align*}
(stetig durch verkleben)

$s$ ist simpliziale Approximation von $f$, da $s$ so gewählt wird,
dass $s\left(\frac{\sqrt{2}}{\sqrt{3}}\right) = \frac{2}{3}$ ist.\bsphere
\end{bspn}
\begin{propn}[Problem]
Finden Sie eine zweite von $\sigma$ verschiedene simpliziale Approximation $t:
K^2\to L$ von $f$.\fishhere
\end{propn}
Wir können $f$ also nicht auf dem ursprünglichen Simplex simplizial
approximieren aber auf $K^2$.

Ist insbesondere $x\in K$ Ecke $f(x)$ im Innern von $\sigma^q\in L$, dann ist
$s(x)$ auch Ecke von $\sigma^q$. Ist $f(x) = w$ Ecke von $L$, dann folgt
$s(x)=f(x)=w$.

Die Funktionswerte von $f$ und $s$ durchlaufen also immer dieselben Simplizes.

\begin{prop}[Simplizialer Approximationssatz]
\label{prop:4.2.4}
Seien $K,L$ Simplizialkomplexe, $f: \abs{K}\to\abs{L}$ stetig, dann gibt es ein
$m\in\N$, so dass $f: \abs{K^m}\to\abs{L}$ eine simpliziale Approximation $s:
K^m\to L$ besitzt.\fishhere
\end{prop}
\begin{proof}
Siehe Armstrong.\qedhere
\end{proof}

\begin{prop}
\label{prop:4.2.5}
Seien $K,L$ Simplizialkomplexe, $f: \abs{K}\to\abs{L}$ stetig und $s: K\to L$
simpliziale Approximation von $f$, dann sind $s$ und $f$ homotop.\fishhere
\end{prop}
\begin{proof}
Es gilt $\abs{L}\subseteq\R^n$, sowie $f,s: \abs{K}\to\R^n$ wenn man den
Wertebereich ausdehnt.

Sei $F: \abs{K}\times I\to \R^n,\; (x,t)\mapsto (1-t)s(x)+tf(x)$ die
Streckenhomotopie von $s$ nach $f$ als Abbildungen von $\abs{K}$ in den
konvexen Raum $\R^n$. Für $x\in K$ sei $\Delta\in L$ das Trägersimplex von
$f(x)$. Dann ist $s(x)\in\Delta$, da $s$ simpliziale Approximation von $f$ ist
und daher ist $(1-t)s(x)+tf(x)\in\Delta$ für alle $t\in I$, da $\Delta$ konvex
ist.

Also ist $F: \abs{K}\times I\to \abs{L}$ in der Tat Homotopie von $s$ nach $f$
als Abbildungen von $\abs{K}$ nach $\abs{L}$.\qedhere
\end{proof}

\begin{prop}
\label{prop:4.2.6}
Sei $K$ ein Simplizialkomplex, $v\in K_0$ (=Eckenmenge). Dann ist die Kantengruppe
$E(K,v)\cong \Pi(K,v)$.\fishhere
\end{prop}
\begin{proof}
Die Abbildung $\ph: E(K,v)\to \Pi(K,v)$, die jeder Äquivalenzklasse eines
geschlossenen Kantenzuges an $v$ eine Homotopieklasse des zugrundeliegenden
(stückweise linearen) geschlossenen Weges zuordnet, ist offensichtlich
wohldefiniert, da äquivalente Kantenzüge auch homotop $\rel{0,1}$ sind.

Klar ist auch, dass $\ph$ Gruppenhomomorphismus ist, denn die Multiplikation
zweier Kantenzüge ist die Hintereinanderausführung der zugrundeliegenden Wege.
\begin{enumerate}[label=\arabic{*}.)]
  \item $\ph$ ist surjektiv:

Sei $\alpha: I\to\abs{K}$ Schleife an $v$. z.Z. $\alpha$ ist homotop
$\rel{0,1}$ zu einem geschlossen Kantenzug an $v$.

Sei $L$ das 1-Simplex $\lin{x_0,x_1}$, so ist $\abs{L}=I$. Nach dem
simplizialen Approximationssatz \ref{prop:4.2.4} gibt es ein $m\in\N$ und eine
simpliziale Abbildung $s: \abs{L^m}\to K$, die $\alpha$ simplizial
approximiert. Nach dem Beweis von \ref{prop:4.2.5} sind $s$ und $f$ homotop
mit der Streckenhomotopie,
\begin{align*}
F(x,t) = (1-t)s(x) + tf(x),\quad \text{für } t\in I.
\end{align*}
Also gilt für alle $t\in I$,
\begin{align*}
F(1,t) = s(1) = \alpha(1) = v,\\
F(0,t) = s(0) = \alpha(0) = v,
\end{align*}
d.h. die Homotopie ist $\rel{0,1}$ und $\lin{s} = \lin{\alpha}$. Also ist $\ph$
surjektiv.
\item $\ph$ ist injektiv:

Dazu reicht es zu zeigen, dass $\ker \ph = \nrm{v}$ ist.
D.h. ein geschlossener Kantenzug $vv_1\ldots v_{k-1}v$, der als Weg in
$\abs{K}$ nullhomotop ist, muss äquivalent zu $\nrm{v}$ sein.

Sei $\alpha: I\to\abs{K}$ der von $vv_1\ldots v_{k-1}v$ induzierte Weg. Da
$\alpha$ nullhomotop ist, gibt es eine stetige Abbildung
\begin{align*}
F: I\times I\to \abs{K},
\end{align*}
mit
\begin{align*}
&F(s,0) = \alpha(s),\\
&F(s,1) = v,
\end{align*}
für alle $s\in I$. Außerdem ist
\begin{align*}
F(0,t) = F(1,t) = v,
\end{align*}
für alle $t\in I$, da alle Kurven $\alpha_t: I\to K,\; s\mapsto F(s,t)$
Schleifen an $v$ sind. ($F$ ist Homotopie $\rel{0,1}$)

Wir machen $I\times I$ zum Polyeder eines Simplizialkomplexes $L$ mit Ecken,
\begin{align*}
&a = (0,0),\; b=(0,1),\;\; c=(1,1), d=(1,0),\\
&a_i = \left(\frac{i}{k},0\right),\; k\in\N,
\end{align*}
und $2$-Simplizes,
\begin{align*}
\lin{b,a_{i-1},a_i}, \lin{b,c,d}.
\end{align*}
%TODO: Bild, Zerlegung von IxI in Dreiecke
Dann ist $L$ Simplizialkomplex bestehend aus Dreiecken.
%TODO: Bild, Strecke ad
wird durch $\alpha=\alpha_0$ Weg $\alpha$ in $\abs{K}$ an $v\in\abs{K}$,
%TODO: Bild Rechteck,
wird durch Homotopie $F$ konstant auf $v\in\abs{K}$ abgebildet.
%TODO: Punkt
zusätzliche Ecken bei baryzentrischer Unterteilung der Kantenzüge $abcd$ bzw.
$\alpha a_1 a_2 \ldots a_{k-1} d$ auf den involvierten Kanten.

In $L$ sind die Kantenzüge $E_{1,0} := a a_1 \ldots a_{k-1}d$ und
$E_{2,0}:=abcd$ offensichtlich äquivalent.

In der baryzentrischen Unterteilung $L^m$, $m\in\N$ von $L$ erhalten wir zwei
Kantenzüge $E_{1,m}$ und $E_{2,m}$.

Induktion über $m$ ergibt, dass $E_{1,m}$ und $E_{2,m}$ äquivalent in
$\abs{L^m}$ sind. Mit \ref{prop:4.2.4} folgt, es existiert ein $m\in\N$ und eine simpliziale
Approximation $S: L^m\to K$ zu $F: \abs{L^m}\to\abs{K}$.

\begin{bemn}[Leichte Übung:]
Simpliziale Abbildungen erhalten die Relationen \ref{defn:4.2.1:1},
\ref{prop:4.2.1:2} und \ref{prop:4.2.1:3} in \ref{defn:4.2.1} und daher
erhalten sie auch die Äquivalenzrelation $\sim$ auf geschlossenen Kantenzügen.
\end{bemn}
\begin{bemn}[Also:]
Die Bilder $S(E_1),S(E_2)$ von $E_1$ und $E_2$ unter $S$ sind äquivalent in
$K$,
\begin{align*}
&S(E_2) = \text{ die Ecke } v (3\cdot 2^m + 1)\text{-mal},\\
&\nrm{S(E_2)} = \nrm{v\ldots v} = \nrm{v}.
\end{align*}
\end{bemn}
Andererseits mit $F(a_i) = v_i\in K$ $1\le i\le k-1$ ist $S$ (neue Ecke
zwischen $a_i$ und $a_{i+1}$) $=v_i\in\abs{K}$ oder $=v_{i+1}$, da $S$ eine
Abbildung $F$ simplizial approximiert.

Also ist $S(E_1)$ ein Kantenzug äquivalent zu $vv_1\ldots v_{k-1}v$ und daher
ist
\begin{align*}
\nrm{v} = \nrm{vv_1\ldots v_{k-1}v}.
\end{align*}
Also ist $\ph$ injektiv.
\end{enumerate}
$\Rightarrow \ph$ ist Isomorphismus.\qedhere
\end{proof}

\begin{bemn}[Wollen:]
Erzeugende und Relationen für $E(K,v)$ direkt am Simplizialkomplex $K$ ablesen.
\end{bemn}

\begin{defn}
\label{defn:4.2.7}
Sei $K$ ein Simplizialkomplex.
\begin{enumerate}[label=\arabic{*}.)]
  \item Ein \emph{Unterkomplex} $L$ von $K$ ($L\leqslant K$) ist eine Teilmenge
  von $K$, so dass $L$ ein Simplizialkomplex ist und jeder Simplex von $L$ auch Simplex
  von $K$ ist.

$L$ heißt dann \emph{voll}, wenn jeder Simplex von $K$ dessen sämtliche Ecken
in $L$ liegen, schon in $L$ liegt.
\item Ein eindimensionaler Simplizialkomplex heißt \emph{Graph} (genauer Graph
ohne Schleifen und Doppelkanten)
%TOD: Graph
\item Ein zusammenhängender, einfach zusammenhängender Graph heißt \emph{Baum}.
(Graph ohne geschlossene Kantenzüge)\fishhere
%TODO: Baum
\end{enumerate}
\end{defn}

\begin{lem}
\label{prop:4.2.8}
Jeder zusammenhängende Simplizialkomplex $K$ enthält einen maximalen Baum als
Unterkomplex. Solch ein Baum enthält alle Ecken von $K$.\fishhere
\end{lem}
\begin{proof}
Klar ist, dass jeder zusammenhängende Simplizialkomplex einen maximalen Baum
als Unterkomplex enthält. ($K$ hat nur endlich viele Elemente)

Sei $T$ maximaler
Baum $\leqslant K$, $v$ Ecke in $K$ mit $v\notin T$, $u$ Ecke in $T$. Sei $uv_1\ldots v_k v$ Kantenzug von $u$ nach $v$. Ein solcher
existiert, da $K$ zusammenhängend ist. Sei $v_i$ die letzte Ecke in diesem
Kantenzug, die noch in $T$ liegt. Sei $T' = T\cup
\{v_{i+1},\lin{v_i,v_{i+1}}\}$ (für $i=k$ sei $v_{i+1} = v$)
%TODO: Bild, Baum mit Ecke
$T$ ist starker Deformationsretrakt von $T'$ also ist $T'$ einfach
zusammenhängend und daher ist $T'\leqslant K$ Baum, aber $T$ war bereits maximal, ein
Widerspruch\dipper

Also liegen alle Ecken von $K$ in $T$.
\end{proof}

\begin{defn}
\label{defn:4.2.9}
Sei $K$ zusammenhängender S-Komplex, $T$ ein maximaler Baum in $K$ und
$K_0=\{v=v_0,v_1,\ldots,v_k\}$ die Menge Menge der Ecken in von $K$.

Sei $G(K,T)$ die Gruppe, die von Elementen $g_{ij}$ mit $i,j\in\{0,\ldots,s\}$
erzeugt wird, wobei $\lin{v_i,v_j}\in K$ ($K\setminus T$) und die folgenden
Relationen erfüllt sind:
\begin{enumerate}[label=(\roman{*})]
  \item\label{defn:4.2.9:1} $g_{ij} = 1$, falls $\lin{v_i,v_j}\in T$. (entfällt)
  \item\label{defn:4.2.9:2} $g_{ij}g_{jk} = g_{ik}$, falls $\lin{v_i,v_j,v_k}\in
  K$.
\end{enumerate}
Aus \ref{defn:4.2.9:2} folgt, $g_{ij}^{-1} = g_{ji}$, da $g_{ii} = 1$. Somit
gilt,
\begin{align*}
G(K,T) = \lin{g_{ij}\ : \ 0\le i < j \le s, \lin{v_i,v_j}\in K\setminus T,\;
g_{ij}g_{jk}=g_{ik},\text{ für } \lin{v_i,v_j,v_k}\in K}.
\end{align*}
Im Übrigen berühren die Relationen keine höherdimensionalen Simplizes als
Dreiecke.\fishhere
\end{defn}
\begin{prop}
\label{prop:4.2.10}
Sei $K$ zusammenhängender S-Komplex und $T$ maximaler Baum in $K$. Dann ist
\begin{align*}
G(K,T)\cong E(K,v),
\end{align*}
wobei $v$ eine beliebige Ecke von $K$ ist.\fishhere
\end{prop}
\begin{proof}
Wir definieren,
\begin{align*}
\phi: G(K,T)\to E(K,v),
\end{align*}
auf den Erzeugenden $g_{ij}$ von $G(K,T)$ durch,
\begin{align*}
\phi(g_{ij}) := \nrm{E_i v_iv_j E_j^{-1}},
\end{align*}
mit $E_i$ ist ein fest gewählter Kantenzug in $T$ von $v=v_0$ nach $v_i$.

Ist $\lin{v_i,v_j}\in T$, so ist $E_iv_iv_jE_j^{-1}$ Schleife in $T$ und daher
äquivalent zu $v=E_0$, d.h. in $E(K,v)$ ist $\nrm{E_iv_iv_jE_j^{-1}}$ das
Einselement.

Spannen $v_i,v_j,v_k$ ein $2$-Simplex in $K$ auf, so ist
\begin{align*}
\phi(g_{ij})\phi(g_{jk}) &= \nrm{E_iv_iv_jE_j^{-1}}\nrm{E_jv_jv_kE_k^{-1}} =
\nrm{E_jv_iv_jE_j^{-1}E_jv_jv_kE_k^{-1}} \\ &=
\nrm{E_jv_iv_jv_jv_kE_k^{-1}} =
\nrm{E_jv_iv_kE_k^{-1}} = \phi(g_{ij}\cdots g_{jk}) = \phi(g_{ik}).  
\end{align*}
Also ist $\phi$ ein Homomorphismus.

Analog definieren wir,
\begin{align*}
\psi: E(K,v)\to G(K,T),
\end{align*}
durch
\begin{align*}
\psi(\nrm{vv_kv_lv_m\cdots v_n v}) = g_{0k}g_{kl}g_{lm}\cdots g_{n0}.
\end{align*}
Eine leichte Übung zeigt, dass auch $\psi$ ein Gruppenhomomorphismus ist.

Wir haben daher,
\begin{align*}
\psi\circ\phi(g_{ij}) = \psi(\nrm{E_iv_iv_jE_j^{-1}}) = g_{ij},
\end{align*}
da die Paare von Ecken in $E_i$ und $E_j^{-1}$ Kanten in $T$ aufspannen und
daher die entsprechenden $g_{uv}=1$ sind.

Ist $vv_kv_lv_m\cdots v_n v$ ein geschlossener Kantenzug in $K$ an $v$, so ist
\begin{align*}
\nrm{vv_kv_l\cdots v_nv} =
\nrm{E_0vv_kE_k^{-1}}\nrm{E_kv_kv_lE_l^{-1}}\cdots\nrm{E_nv_nvE_0^{-1}}.
\tag{*}
\end{align*}
Nun ist,
\begin{align*}
\phi\circ\psi(\nrm{E_iv_iv_jE_j^{-1}}) = \phi(g_{ij}) = \nrm{E_iv_iv_jE_j^{-1}},
\end{align*}
und daraus folgt mit (*), dass
\begin{align*}
\phi\circ\psi(\nrm{vv_kv_l\cdots v_nv}) = \nrm{vv_kv_l\cdots v_nv}.
\end{align*}
Also sind $\phi$ und $\psi$ zueinander inverse Isomorphismen von $E(K,v)$ und
$G(K,T)$.\qedhere
\end{proof}
\begin{bemn}[Beobachtung.]
Sei $K$ ein S-Komplex. Die Teilmenge der Ecken ($0$-Simplizes), Kanten
($1$-Simplizes) und Dreiecke ($2$-Simplizes) bilden einen Unterkomplex $K_2$,
\begin{align*}
K_2 = \setdef{q\text{-Simplizes in }K}{q\le 2},
\end{align*}
das sogenannte \emph{$2$-Skelett} von $K$. Alle Konstruktionen in diesem
Abschnitt (also $G(K,T)$ und $E(K,v)$) involvieren nur $q$-Simplizes mit $q\le
2$.

Um diese Objekte für $K$ zu bestimmen, genügt es also, sie für $K_2$ zu
bestimmen.
\begin{align*}
G(K_2,T) = G(K,T) = E(K,v) = E(K_2,v) \cong \Pi(K,v) = \Pi(K_2,v).\maphere 
\end{align*}
\end{bemn}

\begin{cor}
\label{prop:4.2.12}
Die Fundamentalgruppe eines wegzusammenhängenden, triangulierbaren Raums ist
\emph{endlich präsentierbar}, d.h. kann durch eine endliche Menge von Erzeugern
und eine endliche Menge von Relationen beschrieben werden.\fishhere
\end{cor}

Unser Ziel, die kompakten zusammenhängenden randlosen Flächen zu klassifizieren
haben wir aus Zeitgründen nicht erreicht. :(