\section{Weitere Grundbegriffe}

\subsection{Trennungs- und Abzählbarkeitsaxiome}

Sei nun $X$ topologischer Raum mit $\OO_X,\AA_X, \TT: X\to\PP\PP(X)$. Für
$y\in X$ sei $U(y) \in\UU_y$.

\begin{prop}[Fragen]
\label{prop:2.1.2}
\begin{enumerate}
  \item Wann ist der Grenzwert eines konvergenten Filters auf $X$ eindeutig?
  \item Wann kann man zwei Punkte $y,z\in X$ ($z\neq x$) durch Umgebungen
  trennen?
  \item Wann kann man Punkte $y,z\in Y$ durch eine stetige reelle Funktion
  trennen, d.h. $f: X\to \R$ stetig und $f(z)\neq f(y)$?\fishhere
\end{enumerate}
\end{prop}
Um diese Fragen zu beantworten brauchen wir einen Begriff der ``Nähe''. Dazu
führen wir die Trennungsaxiome ein.

\begin{defnn}
Sei $(X,\OO_X)$ ein topologischer Raum mit $\AA_X$, dann ist $X$ ein
$T_i$-Raum, falls $(X,\OO_X)$ das Axiom $T_i$ erfüllt:
\begin{enumerate}
  \item[$T_0$] Zu je zwei Punkten $z\neq y$ hat einer der Punkte eine
  Umgebung, die den anderen Punkt nicht enthält.
\begin{center}
\psset{unit=1cm}
\psset{linecolor=gdarkgray}
\psset{fillcolor=glightgray}
\begin{pspicture}(0,0)(5,4)

 %\psgrid

 % Kartoffel
 \psccurve[fillstyle=none]%
 (1,0.5)(1,3)(3,2.5)(4.5,3)(4.5,1)(3,1)
 
 \psdot(1.8,1.8)
 
 \psdot(3.4,1.8)
 
 \pscircle[linestyle=dotted]%
 (1.8,1.8){0.9}
 
 \rput[lt](1.9,1.8){\color{gdarkgray}$z$}
 \rput[lt](3.5,1.8){\color{gdarkgray}$y$}
 \rput[lt](1.4,2.4){\color{gdarkgray}$U(z)$}
 \rput[lt](1.4,2.4){\color{gdarkgray}$U(z)$}
 \rput[lt](4.4,0.6){\color{gdarkgray}$X$}
 
\end{pspicture}
\end{center}
  \item[$T_1$] Zu je zwei Punkten $z\neq y\in X$ gibt es Umgebungen $U(z)\in
  \UU_z$ und $U(y)\in \UU_y$ mit $y\notin U(z)$ und $z\notin U(y)$.
  \begin{center}
\psset{unit=1cm}
\psset{linecolor=gdarkgray}
\psset{fillcolor=glightgray}
\begin{pspicture}(0,0)(5,4)

 %\psgrid

 % Kartoffel
 \psccurve[fillstyle=none]%
 (1,0.5)(1,3)(3,2.5)(4.5,3)(4.5,1)(3,1)
 
 \psdot(1.8,1.8)
 
 \psdot(3.2,1.8)
 
 \pscircle[linestyle=dotted]%
 (1.8,1.8){0.9}
 
 \pscircle[linestyle=dotted]%
 (3.2,1.8){0.6}
 
 \rput[lt](1.9,1.8){\color{gdarkgray}$z$}
 \rput[lt](3.5,1.8){\color{gdarkgray}$y$}
 \rput[lt](1.4,2.4){\color{gdarkgray}$U(z)$}
 \rput[lt](3.8,2.4){\color{gdarkgray}$U(y)$}
 \rput[lt](4.4,0.6){\color{gdarkgray}$X$}
 
\end{pspicture}
\end{center}
  \item[$T_2$] Zu zwei Punkten $z\neq y$ in $X$ gibt es disjunkte Umgebungen,
  d.h. es gibt $U(z)\in\UU_z$ und $U(y)\in\UU_y$ mit $U(z)\cap U(y)=\varnothing$.
    \begin{center}
\psset{unit=1cm}
\psset{linecolor=gdarkgray}
\psset{fillcolor=glightgray}
\begin{pspicture}(0,0)(5,4)

 %\psgrid

 % Kartoffel
 \psccurve[fillstyle=none]%
 (1,0.5)(1,3)(3,2.5)(4.5,3)(4.5,1)(3,1)
 
 \psdot(1.8,1.8)
 
 \psdot(3.6,1.8)
 
 \pscircle[linestyle=dotted]%
 (1.8,1.8){0.9}
 
 \pscircle[linestyle=dotted]%
 (3.6,1.8){0.6}
 
 \rput[lt](1.9,1.8){\color{gdarkgray}$z$}
 \rput[lt](3.7,1.8){\color{gdarkgray}$y$}
 \rput[lt](1.4,2.4){\color{gdarkgray}$U(z)$}
 \rput[lt](4,2.6){\color{gdarkgray}$U(y)$}
 \rput[lt](4.4,0.6){\color{gdarkgray}$X$}
 
\end{pspicture}
\end{center} 
  Dieses Axiom ist äquivalent dazu, dass $X$ ein \emph{Hausdorffraum} ist.
  \item[$T_3$] Zu jeder abgeschlossenen Menge $A\in\AA_X$ und jedem Punkt $z\in
  X$ mit $z\notin A$ gibt es $U(A)\in\OO_X$, $U(x)\in\OO_X$ mit $x\in
  U(x),\;A\subseteq U(A)$ so, dass $U(x)\cap U(A)=\varnothing$ ist.
  \begin{center}
\psset{unit=1cm}
\psset{linecolor=gdarkgray}
\psset{fillcolor=glightgray}
\begin{pspicture}(0,0)(5,4)

 %\psgrid

 % Kartoffel
 \psccurve[fillstyle=none]%
 (1,0.5)(1,3)(3,2.5)(4.5,3)(4.5,1)(3,1)
 
 \psdot(1.8,1.8)
 
 \psdot(3.6,1.8)
 
 \pscircle[linestyle=dotted]%
 (1.8,1.8){0.9}
 \pscircle[fillstyle=solid,
 		   fillcolor=darkblue,
 		   linestyle=none]%
 (1.8,1.8){0.7}
 
 \pscircle[linestyle=dotted]%
 (3.6,1.8){0.6}
 
 \rput[lt](1.1,2.9){\color{darkblue}$A$}
 \rput[lt](3.7,1.8){\color{gdarkgray}$x$}
 \rput[lt](1.1,0.8){\color{gdarkgray}$U(A)$}
 \rput[lt](4,2.6){\color{gdarkgray}$U(x)$}
 \rput[lt](4.4,0.6){\color{gdarkgray}$X$}
 
\end{pspicture}
\end{center}
  \item[$T_4$] Zu je zwei disjunkten, abgeschlossenen Teilmengen $A,B\in\AA_X$
   gibt es offene Mengen $U(A),U(B)\in\OO_X$ mit $A\subseteq
  U(A),\;B\subseteq U(B)$ und $U(A)\cap U(B)=\varnothing$.
    \begin{center}
\psset{unit=1cm}
\psset{linecolor=gdarkgray}
\psset{fillcolor=glightgray}
\begin{pspicture}(0,0)(5,4)

 %\psgrid

 % Kartoffel
 \psccurve[fillstyle=none]%
 (1,0.5)(1,3)(3,2.5)(4.5,3)(4.5,1)(3,1)
 
 \psdot(1.8,1.8)
 
 \psdot(3.6,1.8)
 
 \pscircle[linestyle=dotted]%
 (1.8,1.8){0.9}
 \pscircle[fillstyle=solid,
 		   fillcolor=darkblue,
 		   linestyle=none]%
 (1.8,1.8){0.7}
 
 \pscircle[linestyle=dotted]%
 (3.6,1.8){0.6}
 
 \pscircle[fillstyle=solid,
 		   fillcolor=darkblue,
 		   linestyle=none]%
 (3.6,1.8){0.5}
 
 \rput[lt](1.1,2.9){\color{darkblue}$A$}
 \rput[lt](4.2,1.4){\color{darkblue}$B$}
 \rput[lt](1.1,0.8){\color{gdarkgray}$U(A)$}
 \rput[lt](4,2.6){\color{gdarkgray}$U(B)$}
 \rput[lt](4.4,0.6){\color{gdarkgray}$X$}
 
\end{pspicture}
\end{center} 
  \item[$T_{3a}$] Zu jedem $z\in X$ und jeder offenen Umgebung $U(z)\in\UU_z$
  von $z$ gibt es eine stetige Funktion $f:X\to[0,1]\subseteq \R$ mit $f(z)=0$
  und $f(y)=1$ für alle $y\in X$ mit $y\notin U(z)$.
    \begin{center}
\psset{unit=1cm}
\psset{linecolor=gdarkgray}
\psset{fillcolor=glightgray}
\begin{pspicture}(-1,-1)(4,2.5)

 %\psgrid
 
 \psaxes[labels=none,ticks=none]{->}%
 (0,0)(-0.2,-0.5)(3.5,2)[\color{gdarkgray}$X$,0][\color{gdarkgray}$\R$,0]

\psyTick(1){\color{gdarkgray}$1$}
\psxTick(1.5){\color{gdarkgray}$z$}

\psline[arrows=(-)](0.5,0)(2.5,0)

\psplot[linewidth=1.2pt,%
	     linecolor=darkblue,%
	     algebraic=true]%
	     {0.50001}{2.49999}{-(2.71828)^(-(x-1.5)^2/(1-(x-1.5)^2))+1}

\psline[linecolor=darkblue,linewidth=1.2pt](0,1)(0.50002,1)
\psline[linecolor=darkblue,linewidth=1.2pt](2.49998,1)(3.5,1)

\rput(-0.4,0){\color{gdarkgray}$0$}
\rput[lt](2.5,-0.3){\color{gdarkgray}$U(z)$}
\end{pspicture}
\end{center} 
  \item[$T_5$] Zu je zwei Mengen $A,B\subseteq X$ mit $\overline{A}\cap B =
  A\cap \overline{B} = \varnothing$ gibt es disjunkte offene
  Obermengen.\fishhere
\end{enumerate}
\end{defnn}

\begin{defn}
\label{defn:2.1.2}
$X$ ist \emph{hausdorffsch} bzw. \emph{Hausdorffraum}, wenn gilt
\begin{align*}
\forall y,z\in X : y\neq z \exists U(z)\in\UU_z, U(y)\in\UU_y : U(z)\cap U(y) =
\varnothing.\fishhere
\end{align*}
\end{defn}

\begin{defn}
\label{defn:2.1.3}
\begin{enumerate}
  \item Ein topologischer Raum $(X,\OO_X)$ heißt $T_i$-Raum, falls $X$ eines
  der Axiome $T_i, (i=0,1,2,3,3a,4,5)$ erfüllt.
  \item $(X,\OO_X)$ heißt \emph{regulär}, wenn $X$ ein $T_1$ und $T_3$ Raum ist.
  \item $(X,\OO_X)$ heißt \emph{vollständig regulär}, wenn $X$ ein $T_1$ und
  $T_{3a}$ Raum ist.
  \item $(X,\OO_X)$ heißt \emph{normal}, falls $X$ ein $T_1$- und ein
  $T_4$-Raum ist.
  \item $(X,\OO_X)$ heißt \emph{vollständig normal} oder \emph{normal erblich},
  falls $X$ ein $T_1$- und ein $T_5$-Raum ist.\fishhere
\end{enumerate}
\end{defn}

\begin{prop}
\label{prop:2.1.4}
$T_2\Rightarrow T_1\Rightarrow T_0$.\fishhere
\end{prop}

\begin{prop}
\label{prop:2.1.5}
$(X,\OO_X)$ ist ein $T_1$-Raum genau dann, wenn alle einelementigen Teilmengen
von $X$ abgeschlossen sind (und damit auch alle endlichen Teilmengen von
$X$).\fishhere
\end{prop}
\begin{proof}
``$\Rightarrow$'': Sei $X$ ein $T_1$-Raum und sei $z\in X$. Dann gibt es zu
jedem $y\in X\setminus\{z\}$ eine offene Umgebung $U(y)$ mit $z\neq U(y)$.
Folglich ist $y\in X\setminus\{z\} = \bigcup\limits_{y\in X\setminus\{z\}}
U(y)$. Also ist $\{x\} = X\setminus(X\setminus \{z\})$ abgeschlossen.

``$\Leftarrow$'': Seien $z,y\in X$, $z\neq y$, $\{z\}, \{y\} \in \AA_X$. Dann
sind $U(z) = X\setminus \{z\}$ und $U(y) = X\setminus\{y\}$ Umgebungen von $z$
bzw. von $y$ und mit $y\notin U(z)$, $z\notin U(y)$. Also gilt $T_1$.\qedhere
\end{proof}

\begin{cor}
\label{prop:2.1.6}
normal $\Rightarrow$ regulär $\Rightarrow T_2 \Rightarrow T_1 \Rightarrow
T_0$.\fishhere
\end{cor}

\begin{bem}
\label{bem:2.1.7}
Es gilt sogar vollständig normal $\Rightarrow$ normal $\Rightarrow$ vollständig
regulär $\Rightarrow$ regulär.\maphere
\end{bem}

\begin{prop}
\label{prop:2.1.8}
Sei $(X,\OO_X)$ topologischer Raum.
\begin{enumerate}
  \item Ist $X$ normal und $A\in\AA_X$, so ist $(A, \text{Spurtopoloige})$
  normal.
  \item Ist $X$ ein $T_1$- und $T_5$-Raum, so ist jeder Unterraum von $X$ normal
  (daher auch normal erblich).\fishhere
\end{enumerate}
\end{prop}
\begin{proof}
Hausaufgabe.\qedhere
\end{proof}

\begin{prop}
\label{prop:2.1.9}
$(X,\OO_X)$ ist hausdorffsch (d.h. $T_2$-Raum) genau dann, wenn jeder
konvergente Filter in $X$ gegen genau einen Punkt von $X$ konvergiert.
Insbesondere sind die Grenzwerte konvergenter Folgen in Hausdorffräumen
eindeutig bestimmt.\fishhere
\end{prop}
\begin{proof}
``$\Rightarrow$'': Sei $\FF$ Filter auf $X$ und konvergiere $\FF$ gegen die
Punkte $z,y\in X$. D.h. $U_z\subseteq \FF$ und $U_y\subseteq \FF$.

Angenommen $z\neq y$ seien $U(z)\in U_z, U(y)\in U_y$ mit $U(z)\cap
U(y)=\varnothing$. (Existenz wegen $T_2$) Dann ist $U(z),U(y)\in\FF$ wegen
$U_z,U_y\subseteq \FF$ aber dann ist auch $\varnothing\in \FF$, da $\FF$
abgeschlossen gegenüber Durchschnitten.\dipper

Also ist $z=y$.

``$\Leftarrow$'': Sei das $T_2$ Axiom nicht erfüllt, dann gibt es Punkte
$z,y\in X$, $z\neq y$ so, dass für alle $U(z)\in U_z, U(y)\in U_y$ gilt
$U(z)\cap U(y)=\varnothing$.
Das System aller dieser Teilmengen $U(z)\cap U(y)$ erfüllt FB1 und FB2. Sei
$\FF$ der von diesen Mengen erzeugte Filter. $U(z)\in U_z, U(y)\in U_y:
U(z)\supseteq U(z)\cap U(y)\subseteq U(y)$, also ist $U(z), U(y)\in \FF$, also
ist $U_z,U_y\in\FF$.

Damit konvergiert $\FF$ gegen $z$ und $y$, der Grenzwert ist also nicht
eindeutig.\dipper\qedhere
\end{proof}

\begin{prop}
\label{prop:2.1.10}
Sei $f:(X,\OO_X)\to(Y,\OO_Y)$ stetig und sei $Y$ hausdorffsch. Dann ist der
Graph $G(f) = \setdef{(x,y)\in X\times Y}{x\in X,y\in Y}$ abgeschlossen in der
Produkttopologie $\OO_{X\times Y}$ auf $X\times Y$.\fishhere
\end{prop}
\begin{proof}
Sei $(x,y)\in X\times Y$ mit $(x,y)\notin G$. Dann ist $y\neq f(x) = z\in Y$.
$Y$ ist $T_2$ also gilt
\begin{align*}
\exists U(y), U(z) \text{ mit } U(y)\cap U(z) = \varnothing.
\end{align*}
Ohne Einschränkung können wir annehmen, dass $U(y),U(z)\in\OO_X$, da jede
Umgebung eines Punktes eines topologischen Raumes eine offene Umgebung des
Punktes enthält.

Da $f$ stetig ist, existiert eine Umgebung $V(x)\in \UU_x$ mit 
\begin{align*}
f(V(x))\subseteq U(f(x)) = U(z),
\end{align*}
also ist auch $U(y)\cap f(V(x))=\varnothing$ und $V(x)\times U(y)$ ist eine
offene Umgebung von $(x,y)\in X\times Y$ in der Produkttopologie, die $G$ nicht
trifft. Daher ist das Komplement $(X\times Y)\setminus G$ offen in $X\times Y$
und folglich ist $G$ abgeschlossen.\qedhere
\end{proof}

\begin{prop}
\label{prop:2.1.11}
Ein topologischer Raum $(X,\OO_X)$ ist hausdorffsch genau dann, wenn die
Diagonale
\begin{align*}
\Delta = \setdef{(x,x)\in X\times X}{x\in X}\subseteq X\times X,
\end{align*}
eine abgeschlossene Teilmenge von $X\times X$ in der Produkttopologie
ist.\fishhere
\end{prop}
\begin{proof}
Die Diagonale ist der Graph der Identität und $\id$ ist Homöomorphismus, wenn
die Topologien gleich sind.\qedhere
\end{proof}

Wie wir bereits gesehen haben, enthält jede Umgebung $U$ eines
Punktes $x\in X$ eine nichtleere offene Umgebung $\OO_X\ni O\subseteq
U$. Für abgeschlossene Umgebungen ist diese Aussage im Allgemeinen falsch.

\begin{prop}
\label{prop:2.1.12}
Sei $(X,\OO_X)$ ein topologischer Raum. Dann sind folgende Aussagen äquivalent
\begin{enumerate}[label=\roman{*})]
  \item\label{prop:2.1.12:1} $X$ ist ein $T_3$-Raum.
  \item\label{prop:2.1.12:2} Sei $x\in X$. Dann enthält jede offene Umgebung von
  $x$ eine abgeschlossene Umgebung von $x$.
\end{enumerate}
\ref{prop:2.1.12:2} gilt insbesondere für normale Räume, da diese das $T_1$
und das $T_3$ Axiom erfüllen.\fishhere
\end{prop}
\begin{proof}
\ref{prop:2.1.12:1}$\Rightarrow$\ref{prop:2.1.12:2}: Sei $z\in X$ und sei
$U(z)\in\UU_z\cap \OO_X$. Dann ist $A=X\setminus U(z) \in \AA_X$.

Wegen $T_3$ gilt: Es gibt $V_A\in\OO_X,\;A\subseteq V_A$ und
$W(z)\in\OO_X\cap\UU_z$ mit $W(z)\cap V_A = \varnothing$.

$W(z)\subseteq X\setminus V_A\in \AA_X\Rightarrow \overline{W(z)} \subseteq
X\setminus V_A$. Also ist $\overline{W(z)}$ eine abgeschlossene Umgebung von
$z$, die $A$ nicht trifft (da sie $V(A)\supseteq A$ nicht trifft).

($z\in W(z)\subseteq\overline{W(z)}\subseteq X\setminus V_A\subseteq X\setminus
A = U(z)$)

\ref{prop:2.1.12:2}$\Leftarrow$\ref{prop:2.1.12:1}: Sei $z\in X$,
$A\in\AA_X$ mit $z\notin A$. Dann gilt $z\in X\setminus A = B\in\OO_X$. Nach
Vorraussetzung \ref{prop:2.1.12:2} gibt es eine abgeschlossene Umgebung
$V(z)\in \UU_z\cap \AA_z$ mit $z\in V(z)\subseteq B = X\setminus A$. In
$V(z)$ gibt es aber eine offene Umgebung $U(z)$ von $z$ und wir haben:
\begin{align*}
z\in U(z)\subseteq \overline{U(z)}\subseteq V(z)\subseteq B = X\setminus A.
\end{align*}
Also ist $A\subseteq X\setminus V(z)\in\OO_X$ $z\in U(z)$ also gilt
$U(z)\cap X\setminus V(z) = \varnothing$.\qedhere
\end{proof}

%Die folgenden Sätze spielen eine wichtige Rolle in der Analysis\ldots
%TODO: Satz von Uryson, Satz von Tieze

Motivation: Bisher haben wir den Konvergenzbegriff nur in Zusammenhang mit
Folgen verwendet, dies sind jedoch abzählbare Mengen, denn die Menge der
Endstückfilter besitzt eine abzählbare Basis. Aus der Analysis kennen wir einen
weiteren Konvergenzbegriff, die Folgenstetigkeit. Diese ist nicht ohne Weiteres
auf beliebige topologische Räume übertragbar.

\begin{defn}
\label{defn:2.1.13}
\begin{enumerate}
  \item Eine Filterbasis $U$ des Umgebungsfilters $\UU_x$ von $x\in X$ heißt
  \emph{Umgebungsbasis von $X$}.
  
  So: $\varnothing\neq U\subseteq \PP(X)$, jedes Element von $U$ ist Umgebung
  von $x$ und jede Umgebung von $x$ enthält ein Element von $U$ als Teilmenge.
  \begin{bspn}
  $\UU_x\cap \OO_X = \setdef{A\in \UU_x}{A \text{ ist offen}.} = \{\text{offene
  Umgebungen von }X\}$ ist Umgebungsbasis von $\UU_x$.\bsphere
  \end{bspn}
  \item $X$ erfüllt das \emph{1. Abzählbarkeitsaxiom}, wenn jeder Punkt von $X$
  eine abzählbare Umgebungsbasis besitzt.
  
  \begin{bspn}
  Für den $\R^n$ mit der natürlichen Topologie und $z\in\R^n$ ist,
  \begin{align*}
  \setdef{U_\ep(z)}{0<\ep\in\Q},
  \end{align*}
  Umgebungsbasis.\bsphere
  \end{bspn}
  \item $X$ erfüllt das \emph{2. Abzählbarkeitsaxiom}, wenn $X$ eine abzählbare
  Basis der Topologie $\OO_X$ besitzt.\fishhere
\end{enumerate}
\end{defn}

\begin{bemn}
Das 2. Abzählbarkeitsaxiom impliziert das 1., denn ist $\BB$ Basis von $\OO_X$,
dann ist $\BB\cap\UU_x$ Umgebungsbasis von $z$.
\end{bemn}

\begin{bsp}
\label{bsp:2.1.14}
\begin{enumerate}
  \item Jeder metrische Raum $X$ (mit induzierter Topologie) erfüllt das 1.
  Abzählbarkeitsaxiom, da
  \begin{align*}
  U = \setdef{U_{\frac{1}{n}}(x)}{n\in\N},
  \end{align*}
  eine abzählbare Umgebungsbasis von $x\in X$ bildet.
  \item Der $\R^n$ mit der natürlichen Topologie erfüllt das 2.
  Abzählbarkeitsaxiom, denn
  \begin{align*}
  \setdef{U_{y_n}(x)}{n\in\N, x\in\Q^n\subseteq \R^n},
  \end{align*}
  ist Basis der Topologie auf dem $\R^n$.
  \item
  Ein überabzählbares topologisches Produkt $\prod\limits_{i\in\II} X_i$ ($\II$
  überabzählbare Indexmenge, $(X_i,\OO_i)$ topologischer Raum für jedes
  $i\in\II$) erfüllt das 1. Abzählbarkeitsaxiom nicht, es sei denn die 
  $X_i$ tragen die indiskrete Topologie.\bsphere
\end{enumerate}
\end{bsp}
\begin{proof}
Hausaufgabe.\qedhere
\end{proof}

\begin{bemn}
In einem $T_2$-Raum, der das 1. Abzählbarkeitsaxiom erfüllt, kann man
Filterkonvergenz vollständig auf Konvergenz von Folgen zurückführen.\maphere
\end{bemn}

\begin{defn}
\label{defn:2.1.15}
Seien $X,Y$ topologische Räume und $f: X\to Y$ Abbildung. Dann heißt $f$
\emph{folgenstetig}, falls gilt:

Konvergiert die Folge $(x_n)$ in $X$ gegen $x\in X$, so konvergiert die Folge
der Bilder $(f(x_n))$ in $Y$ gegen $f(x)\in Y$.

Kurzform: $f\left(\lim\limits_{n\to\infty} x_n\right) =
\lim\limits_{n\to\infty} f(x_n)$.\fishhere
\end{defn}

\begin{prop}
\label{prop:2.1.16}
Seien $X,Y$ topologische Räume, $f: X\to Y$ Abbildung, dann gilt
\begin{enumerate}[label=\arabic{*}.)]
  \item Ist $f$ stetig, so ist $f$ folgenstetig.
  \item Erfüllt $X$ das 1. Abzählbarkeitsaxiom und ist $f$ folgenstetig, so ist
  $f$ stetig.\fishhere
\end{enumerate}
\end{prop}

%TODO: Folie 60

\begin{prop}
Seien $X,Y$ topologische Räume, $f: X\to Y$ Abbildung, $z\in X$, dann ist $f$
stetig in $z$ genau dann, wenn $f(U_z)$ gegen $f(z)$ konvergiert.\fishhere
\end{prop}
\begin{proof}
Übung.\qedhere
\end{proof}

%TODO: Folie 61

\subsection{Zusammenhang und Wegzusammenhang}
\begin{prop}
\label{prop:2.2.1}
Sei $(X,\OO_X)$ ein topologischer Raum und $A\subseteq X$ offen. Sei
$\OO_A$ die Spurtopologie auf $A$ und sei $Y\subseteq A$, dann ist $Y\in\OO_A$
genau dann, wenn $Y\in\OO_X$.\fishhere
\end{prop}
\begin{proof}
``$\Rightarrow$'': Gilt für beliebige $A\subseteq X$ wegen $Y=A\cap Y$ nach
\ref{prop:1.2.3}.

``$\Leftarrow$'': $Y\in\OO_A\Rightarrow\exists Z\in\OO_X : Y = Z\cap A
\Rightarrow Y\in\OO_X$.\qedhere
\end{proof}

\begin{prop}
\label{prop:2.2.2}
$A\in\OO_X \Rightarrow \OO_A = \OO_X\cap \PP(A)$.

$A\in\AA_X \Rightarrow \AA_A = \AA_X \cap \PP(A)$.\fishhere
\end{prop}

Gibt es auch außer der leeren Menge und dem ganzen Raum $X$ Mengen $A$ mit
$A\in \OO_X\cap \AA_X$?

%TODO: Folie 63

\begin{prop}
\label{prop:2.2.3}
Sei $(X,\OO_X)$ topologischer Raum, $A\subseteq X$ offen und abgeschlossen in
$X$. Dann ist $A^c\in\OO_X\cap\AA_X$ und $A^c=X\setminus A$. Seien $\OO_A$ bzw.
$\OO_B$ Spurtopologie von $A$ bzw. $A^c$ als Teilmengen von $X$. Dann ist $\OO_X$
die Summentopologie auf der disjunkten Vereinigung $A\dcup A^c$.\fishhere
\end{prop}

\begin{proof}
Sei $A\in\OO_X$, dann ist $A^c\in\AA_X$ und sei $A\in\AA_X$, dann ist
$A^c\in\OO_X$, also ist $A^c\in\OO_X\cap\AA_X$. Nach \ref{prop:2.2.2} gilt
\begin{align*}
&\OO_A = \OO_X\cap \PP(A),\;\OO_{A^c} = \OO_X\cap \PP(A^c),\\
&\AA_A = \AA_X \cap \PP(A),\;\AA_{A^c} = \AA_X \cap \PP(A^c).
\end{align*}
Sei $U\in\OO_X,\;U_1=U\cap A,\;U_2=U\cap A^c$ und
\begin{align*}
U_1\cap U_2 = (U\cap A)\cup(U\cap A^c) = U\cap(A\cup A^c) = U,
\end{align*}
also ist
\begin{align*}
\OO_X = \setdef{U_1\cup U_2}{U_1\in\OO_A,U_2\in\OO_B},
\end{align*}
und dies ist genau die Summentopologie der disjunkten Vereinigung $A\cup B$,
wenn $A,\ A^c$ die Spurtopologie tragen.

Also ist $X=A\dcup A^c$.\qedhere
\end{proof}

\begin{defn}
\label{defn:2.2.4}
Ein topologischer Raum $(X,\OO_X)$ heißt \emph{zusammenhängend}, wenn $X$ nicht
als disjunkte Vereinigung zweier nichtleerer offener Teilmengen geschrieben werden
kann.\fishhere
\end{defn}

\begin{bem}
\label{bem:2.2.5}
Der topologische Raum $(X,\OO_X)$ ist zusammenhängend genau dann, wenn
$\OO_X\cap\AA_X = \{\varnothing,X\}$ und genau dann, wenn $(X,\OO_X)$ nicht als
topologische Summe zweier echter Teilräume geschrieben werden kann.\maphere
\end{bem}

\begin{lem}
\label{prop:2.2.6}
Seien $A,B\le (X,\OO_X)$ und $A\subseteq B\Rightarrow A\le B$ bzw. $B\le A,
A\le X \Rightarrow B\le X$.

Sei $A\subseteq B\le X$ und $(X,\OO_X)$ topologischer Raum. Dann stimmt die
Spurtopologie von $A$ als Teilmenge von $B$ mit $A$ als Teilmenge von $X$
überein.\fishhere
\end{lem}
\begin{proof}
Sei $\OO_A$ die Spurtopologie von $A$ als Teilmenge von $X$ und $\tilde{\OO}_A$
die Spurtopologie $A$ also Teilmenge von $B$.

Zu zeigen ist nun, dass $\OO_A = \tilde{\OO}_A$:
\begin{align*}
&U\in\tilde{\OO}_A\Leftrightarrow \exists W\in\OO_B : U = A\cap W,\\
&W\in\OO_B \Leftrightarrow \exists Z\in\OO_X : W = Z\cap B,\\
\Rightarrow\; &U = A\cap W = A\cap Z\cap B = A\cap Z\in \OO_A.
\end{align*}
Also ist $\tilde{\OO}_A \subseteq \OO_A$. Ist $U\in\OO_A$, so gibt es ein
$V\in\OO_X$ mit
\begin{align*}
U=A\cap V = A\cap V\cap B\in\tilde{\OO}_A,
\end{align*}
da $V\cap B\in\OO_B$.\qedhere
\end{proof}

Das Bilden von Unterräumen ist also transitiv.

\begin{defn}
\label{defn:2.2.7}
Eine Teilmenge $A$ von $(X,\OO_X)$ heißt
\emph{zusammenhängend}, wenn $(A,\OO_A)$ zusammenhängend ist.\fishhere
\end{defn}

\begin{cor}[Korollar aus \ref{prop:2.2.6}]
\label{prop:2.2.8}
Seien $A,B\le X$ und $X$ topologischer Raum mit $A\subseteq B$, dann ist $A$
als Unterraum von $B$ genau dann zusammenhängend, wenn $A$ als Unterraum von
$X$ zusammenhängend ist.\fishhere
\end{cor}

\begin{prop}
\label{prop:2.2.9}
Sei $(X,\OO_X)$ topologischer Raum und sei $\setdef{A_i}{i\in\II}$ mit $\II$
Indexmenge eine Familie zusammenhängender Teilräume von $X$.

$A_i\le X$ mit Spurtopologie $O_i$ für $i\in\II$.

Sei $\bigcap_{i\in\II} A_i \neq\varnothing$, dann ist $\bigcup_{i\in\II} A_i =
A$ ebenfalls zusammenhängend.\fishhere
\end{prop}
\begin{proof}
Ohne Einschränkung können wir $A$ gleich ganz $X$ setzen. Sind $U,V\in\OO_X$
mit $U\cup V = X$ und $U\cap V =\varnothing$ und $i\in\II$, so ist $A_i\cap U,
A_i\cap V\in\OO_i$ und
\begin{align*}
A_i = A_i\cap X = A_i \cap (U\cup V) = (A_i\cap U)\cup(A_i\cap V),\\
(A_i\cap U)\cap(A_i\cap V) = A_i\cap (U\cap V) = A_i \cap \varnothing =
\varnothing.
\end{align*}
Nach Voraussetzung ist $A_i$ zusammenhängend, es gilt also $A_i\cap
U=\varnothing$ oder $A_i\cap V = \varnothing$ und $A_i\cap V = A_i$ oder
$A_i\cap U = A_i$.

Ohne Einschränkung sei $(A_i\cap U)=A_i,\;(A_i\cap V)=\varnothing$. Dann ist
\begin{align*}
\varnothing\neq \bigcap_{j\in\II} A_j \subseteq A_i \subseteq U,
\end{align*}
dann folgt,
\begin{align*}
\forall i\in\II : A_i\cap U \supseteq \bigcap_{j\in\II} A_j\neq \varnothing,
\end{align*}
wie oben gilt daher $\forall i\in\II : A_i\subseteq U$.

$\Rightarrow \bigcup_{i\in\II} A_i\subseteq U \Rightarrow U=X\text{ und
}V=\varnothing\Rightarrow A=X\text{ ist zusammenhängend}.$\qedhere
\end{proof}

\begin{lem}[Definition/Lemma]
\label{prop:2.2.10}
Sei $(X,\OO_X)$ ein topologischer Raum. Definiere $\sim$ auf $X$ durch $x\sim y
\Leftrightarrow \exists A\le X : A \text{ ist zusammenhängend und } x,y\in A$.
Dann ist $\sim$ eine Äquivalenzrelation.\fishhere
\end{lem}
\begin{proof}
\begin{enumerate}
  \item Sei $x\in X, A = \{x\}$, dann ist $\OO_A = \{\varnothing, \{x\}\}$ 
  zusammenhängend, also ist $x\sim x$. (Reflexivität)
  \item Die Symmetrie ist offensichtlich.
  \item Sei $x\sim y$ und $y\sim z$ $(x,y,z\in X)$. Dann gibt es
  zusammenhängende Unterräume $A,B\subseteq X$ mit $x,y\in A$ und $y,z\in B$.
  
  Da $A\cap B\ni y\Rightarrow A\cap B\neq \varnothing$. Also ist nach
  \ref{prop:2.2.9} $A\cup B$ zusammenhängend aber $x,z\in A\cup B$ also gilt
  $x\sim z$. (Transitivität)\qedhere
\end{enumerate}
\end{proof}

\begin{defn}
\label{defn:2.2.11}
Die Äquivalenzklassen der Relation $\sim$ von \ref{prop:2.2.10} heißen
\emph{Zusammenhangskomponenten}. Ist $x\in X$, so ist die
Zusammenhangskomponente von $X$, die $x$ enthält, gerade die Vereinigung alle
zusammenhängender Teilmengen von $X$, die $x$ enthalten. Diese ist nach
\ref{prop:2.2.9} zusammenhängend und daher die eindeutig bestimmte maximale
zusammenhängende Teilmenge von $X$, die $x$ als Element enthält.\fishhere
\end{defn}

\begin{prop}
\label{prop:2.2.12}
Sei $A$ zusammenhängende Teilmenge des topologischen Raumes $X$. Dann ist
$\overline{A}$ zusammenhängend. Da Zusammenhangskomponenten maximale
zusammenhängende Teilmengen von $X$ sind, sind sie insbesondere abgeschlossen
in $X$.
\end{prop}
\begin{proof}
Ohne Einschränkung können wir $\overline{A}=X$ annehmen. Seien $U,V\in\OO_X$
mit $U\cup V= \overline{A} = X$ und $U\cap V = \varnothing$.

Seien $U' = U\cap A$, $V' = V\cap A$. Dann sind $U', V'\in\OO_A$.
\begin{align*}
A = X\cap A = (U\cup V)\cap A = (U\cap A)\cup(V\cap A) = U'\cup V'.
\end{align*}
Klar: $U'\cap V' = \varnothing$.
$A$ zusammenhängend $\Rightarrow$ $U' = \varnothing$ oder $V'=\varnothing$.

Ohne Einschränkung können wir $U' = \varnothing$ annehmen. Sei $z\in U$. Dann
ist $U\in\UU_z$, weil $U$ offen in $X$ ist. Wegen $z\in\overline{A}$ ist $z$ BP
von $A$ und wir haben andererseits $U' = U\cap A=\varnothing$.\dipper wegen
\ref{defn:1.1.12}. Also ist $A$ zusammenhängend.\qedhere
%TODO: Prüfen, ob die Definition jetzt gefunden wird.
\end{proof}
\begin{cor}
\label{prop:2.2.13}
Sei $(X,\OO_X)$ topologischer Raum und $Z\subseteq X$ Zusammenhangskomponente,
dann ist $Z\in A_X$.\fishhere
\end{cor}

Leider sind Zusammenhangskomponenten im Allgemeinen nicht offen, unter
Zusatzvoraussetzungen erhalten wir jedoch ein positives Ergebnis.

\begin{cor}
\label{prop:2.2.14}
Sei $(X,\OO_X)$ topologischer Raum mit nur endlich vielen
Zusammenhangskomponenten. Dann ist $X$ die topologische Summe seiner
Zusammenhangskomponenten.\fishhere
\end{cor}
\begin{proof}
Seien $Z_1,\ldots,Z_k$ Zusammenhangskomponenten, dann ist $Z_2\cup\ldots\cup
Z_k = X\setminus Z_1$ und daraus folgt, dass $Z_2\cup\ldots\cup
Z_k$ abgeschlossen ist, also ist $Z_1\in\OO_X\cap\AA_X$ analog folgt dies für
die $Z_i$.\qedhere
\end{proof}

\begin{defn}
\label{defn:2.2.15}
Ein topologischer Raum, dessen Zusammenhangskomponenten nur aus je einem Punkt
bestehen heißt \emph{total unzusammenhängend}.\fishhere
\end{defn}
\begin{bspn}
$\R$ ist zusammenhängend in jeder Topologie.\bsphere
\end{bspn}
%TODO: Peadische Zahlen \ldots

\begin{bem}
\label{bem:2.2.16}
\begin{enumerate}
  \item Diskrete Räume sind total unzusammenhängend.
  \item $\Q\le\R$ mit Spurtopologie ist total unzusammenhängend.\maphere
\end{enumerate}
\end{bem}

\begin{prop}[Übung]
\label{prop:2.2.17}
Ist $A\le\R$ zusammenhängend, $x,y\in A$, dann ist $[x,y]\subseteq A$.\fishhere 
\end{prop}

\begin{defn}
\label{defn:2.2.18}
Sei $(X,\OO_X)$ topologischer Raum. Ein \emph{Weg} $\ph$ in $X$ ist eine stetige
Abbildung $\ph: [0,1]\to X$. Dabei heißet $\ph(0)$ \emph{Anfangs-} und $\ph(1)$
\emph{Endpunkt} des Weges $\ph$. Man sagt $\ph$ ist Weg von $\ph(0)$ nach
$\ph(1)$.\fishhere
\end{defn}

Es ist anschaulich klar was gemeint ist. Die uns bekannte Vorstellung eines
``Weges'' ist jedoch mit Vorsicht zu genießen, da sich diese oft auf den $\R^2$
bezieht und dieser nicht gerade der allgemeinste topologische Raum ist.

\begin{defn}
\label{defn:2.2.19}
Ein topologischer Raum $(X,\OO_X)$ heißt \emph{wegzusammenhängend}, wenn zu je
zwei Punkten $x,y\in X$ einen Weg von $x$ nach $y$ existiert, d.h. es gibt eine
stetige Abbildung $\ph: [0,1]\to X$ mit $\ph(0) = x,\;\ph(1)=y$.\fishhere
\end{defn}

Sei nun die Situation folgende: $(X,\OO_X)$ und $(Y,\OO_Y)$ seien topologische
Räume und $(A,\OO_A),(B,\OO_B)\le (X,\OO_X)$, sowie $f: A\to Y,\; g: B\to Y$
stetige Abbildungen mit $f(\alpha) = g(\alpha), \forall \alpha\in A\cap B$.

Sei nun
\begin{align*}
F: A\cup B \to Y,\; \alpha\mapsto\begin{cases}
                                 f(\alpha),& \alpha\in A\\
                                 g(\alpha),& \alpha\in B,
                                 \end{cases}
\end{align*}
$C=A\cup B$ mit Spurtopologie $\OO_C$ in $X$.

Offensichtlich ist $F$ wohldefiniert, die Frage nach der Stetigkeit erfordert
jedoch eine nähere Betrachtung. Für $V\in\OO_Y$ gilt
\begin{align*}
F^{-1}(V) &= \setdef{\alpha\in C}{F(\alpha)\in V}
\\ &= \setdef{\alpha\in A}{f(\alpha)\in V}\cup
\setdef{\alpha\in A}{g(\alpha)\in V}
\\ &= f^{-1}(V)\cup g^{-1}(V).
\end{align*}
Klar ist, dass $f^{-1}(V)\in\OO_A$ und $g^{-1}(V)\in\OO_B$, da $f$ und $g$
stetig sind, damit ist $F^{-1}(V)$ offen in $C$.

Im Allgemeinen müssen $f^{-1}(V)\in\OO_A$ und $g^{-1}(V)\in\OO_B$ aber nicht
notwendigerweise offen in $C$ sein, denn
\begin{align*}
f^{-1}(V) = O_1\cap A,\\
g^{-1}(V) = O_2\cap B.
\end{align*}

Sind aber $A,B\in\OO_X$, dann folgt $A\cup B\in\OO_X$ und mit \ref{prop:2.2.1}
folgt, dass folgt $f^{-1}(V),g^{-1}(V)\in\OO_C$.

Analog dazu argumentiert man für $A,B\in\AA_X$.

Damit haben wir folgenden Satz gezeigt
\begin{prop}
\label{prop:2.2.20}
Seien $X,Y$ topologische Räume, $A,B\le X$ beide offen bzw.
abgeschlossen in $X$ und $f: A\to Y,\;g: B\to Y$ stetige Abbildungen mit
$f(\alpha)=g(\alpha), \forall\alpha\in A\cap B$. Sei
\begin{align*}
F: A\cup B \to Y,\; \alpha\mapsto\begin{cases}
                                 f(\alpha),& \alpha\in A\\
                                 g(\alpha),& \alpha\in B,
                                 \end{cases}
\end{align*}
dann ist $F$ stetig auf $A\cup B$.\fishhere
\end{prop}

\begin{cor}
\label{prop:2.2.21}
Sei $X$ topologischer Raum und $\alpha: [0,1]\to X,\;\beta: [0,1]\to X$ seien
Wege mit $\alpha(1) = \beta(0)$. Dann wird durch folgende Regel,
\begin{align*}
\gamma: [0,1]\to X,\; t\mapsto \begin{cases}
                               \alpha(2t),& 0\le t\le \frac{1}{2},\\
                               \beta(2t-1),& \frac{1}{2}\le t\le 1,
                               \end{cases}
\end{align*}
ein Weg von $\alpha(0)$ nach $\beta(1)$ definiert.\fishhere
\end{cor}
\begin{proof}
Die Abbildungen
\begin{align*}
[0,\frac{1}{2}]\to[0,1],\; t\mapsto 2t \text{ und }
[\frac{1}{2},1]\to[0,1],\; t\mapsto 2t-1,
\end{align*}
sind stetig. Also sind auch
\begin{align*}
\tilde{\alpha}:[0,\frac{1}{2}]\to X,\;t\mapsto \alpha(2t) \text{ und }
\tilde{\beta}: [\frac{1}{2},1]\to[0,1],\;t\mapsto \beta(2t-1),
\end{align*}
stetig. Außerdem ist
$\tilde{\beta}(\frac{1}{2}) = \beta(-1+2\frac{1}{2})=\beta(0) =
\tilde{\alpha}(\frac{1}{2})$. Also stimmen $\tilde{\alpha}$ und $\tilde{\beta}$
auf $\left\{\frac{1}{2}\right\} = [0,\frac{1}{2}]\cap [\frac{1}{2},1]$ überein.

Somit ist nach \ref{prop:2.2.20} die Abbildung
\begin{align*}
\gamma: [0,\frac{1}{2}]\cup[\frac{1}{2},1] \to X,\; t\mapsto
\begin{cases}
\tilde{\alpha}(t)=\alpha(2t),&\text{für } 0\le t\le \frac{1}{2},\\
\tilde{\beta}(t) = \beta(2t-1),&\text{für }\frac{1}{2}\le t\le 1, 
\end{cases}
\end{align*}
stetig und daher Weg von $\alpha(0)$ nach $\beta(1)$.
\end{proof}

\begin{defn}[Definition/Lemma]
\label{defn:2.2.22}
Sei $X$ topologischer Raum. Definiere $x\sim y$ für $x,y\in X$, falls ein Weg
von $x$ nach $y$ existiert. Dann ist $\sim$ eine Äquivalenzrelation. Die
Äquivalenzklassen heißen \emph{\wzhk}, $X$ ist die
disjunkte Vereinigung seiner \wzhk. Die Wegzusammenhangskomponente, die $x\in X$
enthält besteht aus allen Punkten $z\in X$, die mit $x$ durch einen Weg in $X$
verbindbar sind. Sie ist der größte wegzusammenhängende Teilraum von $X$, der
$x\in X$ als Punkt enthält.\fishhere
\end{defn}
\begin{proof}
$\sim$ ist eine Äquivalenzrelation, denn
\begin{enumerate}
  \item Sei $x\in X$. Dann ist die konstante Funktion $\alpha: [0,1]\to X,\;
  t\mapsto x, \forall t\in[0,1]$ ein Weg von $x$ nach $x$ Also ist $x\sim x$.
  (Reflexivität)
  \item Sei $x\sim y$ mit $x,y\in X$ und $\alpha$ Weg von $x$ nach $y$. Dann
  wird durch $\beta: [0,1]\to X,\;t\mapsto \alpha(1-t)$ ein Weg von $y$ nach
  $x$ definiert, also ist $y\sim x$. (Symmetrie)
  \item Dies ist gerade Korollar \ref{prop:2.2.21}.\qedhere
\end{enumerate}
\end{proof}

\begin{bemn}[Warnung:]
Im Gegensatz zu Zusammenhangskomponenten sind Wegzusammenhangskomponenten nicht
zwingend abgeschlossen in $X$.\maphere
\end{bemn}

\begin{lem}
\label{prop:2.2.23}
Jedes Intervall $I\subseteq\R$ mit der natürlichen Topologie ist zusammenhängend
bezüglich der Spurtopologie auf $I$.\fishhere
\end{lem}
\begin{proof}
Angenommen $I=A\cup B$ mit $A\cap B = \varnothing$ und $A,B\in\OO_I$. Sei $a\in
A$ und $b\in B$ und ohne Einschränkung sei $a<b$. Sei $s=\inf \setdef{x\in
B}{a<x}$. Dann gibt es nach Definition des Infimums in jeder Umgebung von $s$
Punkte von $B$. Wegen $I = A\cup B$ ist $s\in A$ oder $s\in B$. Ist $s\in A$,
dann folgt $\exists V\in\UU_s : I\supseteq V$ und $V\subseteq A$. Aber dann ist
$V\cap B = \varnothing$.\dipper\\
Also ist $s\in B$ aber $(a,s)\subseteq A$, sonst wäre $s$ nicht Infimum. Da
$B\in\OO_I$ enthält $B$ eine Umgebung $V$ von $s$, d.h. eine offene Umgebung
der Form $(s-\ep,s+\ep)$ von $s\in \R\cap I$. Daher gibt es ein $x\in I: a<x<s$
mit $x\in B$ im Widerspruch dazu, dass $s$ Infimum ist. Daraus folgt die
Behauptung.\qedhere
\end{proof}

\begin{prop}
\label{prop:2.2.24}
Sei $X$ topologischer Raum, ist $X$ wegzusammenhängend, so ist $X$
zusammenhängend.\fishhere
\end{prop}
\begin{proof}
Seien $\varnothing\neq A,B\in\OO_X$ gegeben mit $A\cup B = X$ und $A\cap B =
\varnothing$, $x\in A,\;y\in B$ und $\alpha:[0,1]\to X$ Weg von $x$ nach
$y$. Dann ist 
\begin{align*}
\varnothing = \alpha^{-1}(A\cap B) = \alpha^{-1}(A)\cap
\alpha^{-1}(B),
\end{align*}
sowie
\begin{align*}
\alpha^{-1}(A)\cup\alpha^{-1}(A\cup B) = \alpha^{-1}(X) =
[0,1].
\end{align*}
Weil $\alpha$ stetig ist, sind $\alpha^{-1}(A)$ und $\alpha^{-1}(B) \in
\OO_I$. Wegen $x\in A,\;y\in B$ sind $\alpha^{-1}(A)$ und $\alpha^{-1}(B)\neq
\varnothing$. Also lässt sich $I$ in eine disjunkte Vereinigung zweier
nichtleerer offener Teilmengen zerlegen.\dipper\qedhere 
\end{proof}

\begin{bem}[Bemerkungen.]
\label{bem:2.2.25}
\begin{enumerate}[label=\arabic{*}.)]
  \item Da Homöomorphismen offene (abgeschlossene) Mengen bijektiv übertragen,
  ist klar, dass Zusammenhang eine topologische Invariante ist. Ebenso ist
  Wegzusammenhang eine topologische Invariante, denn sei $\alpha: [0,1]\to X$
  ein Weg und $f: X\to Y$ ein Homöomorphismus, so ist $f\circ\alpha: [0,1]\to
  Y$ ein Weg in $Y$ von $f\circ\alpha(0)$ nach $f\circ\alpha(1)$.
  \item Es gilt sogar: Ist $f: X\to Y$ stetige Abbildung und ist $X$
  zusammenhängend, so ist $\im f = f(X)$ zusammenhängend. Unter stetigen
  Abbildungen sind Bilder von zusammenhängenden Räumen zusammenhängend.
  \item Damit folgt mit \ref{prop:2.2.17} sofort der
  \begin{propn}[Zwischenwertsatz]
  Sei $(X,\OO_X)$ zusammenhängender topologischer Raum, $f:X\to\R$ stetig und
  $a,b\in\im f$ mit $a<b$, dann ist $[a,b]\subseteq \im f$, d.h. $f$ nimmt
  jeden Wert zwischen $a$ und $b$ an.\fishhere
  \end{propn}
  \item Die Umkehrung von \ref{prop:2.2.24} ist im Allgemeinen falsch. Hier ist
  ein Beispiel für einen zusammenhängenden Raum, der nicht wegzusammenhängend
  ist:
  
  Sei $X = f((0,\infty))$ mit $f:(0,\infty)\to\R^2,\;x\mapsto (x,\sin 1/x)$. $X$
  ist als Bild der stetigen Abbildung $f$ definiert und auf dem Intervall
  $(0,\infty)\subseteq \R$ zusammenhängend, also auch
  $\overline{X} = X\cup\setdef{(0,y)\in\R^2}{-1\le y\le 1}$.
  
  $\overline{X}$ ist nicht wegzusammenhängend, denn sei $\alpha: [0,1]\to
  \overline{X}$ ein Weg von $(0,0)\in\R^2$ nach $(\frac{1}{\pi},0) = \alpha(1)$.
  
  Die Projektionen $p_1: \R^2\to\R,\;(a,b)\mapsto a$ und
  $p_2:\R^2\to\R,\;(a,b)\mapsto b$ sind stetig, also auch die Kompositionen
  $p_1\circ\alpha$ und $p_2\circ\alpha$ von $[0,1]\to\R$.
  
  Nach dem Zwischenwertsatz nimmt daher $p_1\circ\alpha$ alle Werte zwischen
  $0$ und $1/\pi$ an, also insbesondere die Werte $\frac{2}{(2n+1)\pi}$ für
  $n=2,3,\ldots$
  
  Daher nimmt dort $p_2\circ\alpha$ die Werte
  $\sin\left(\frac{(2n+1)\pi}{2}\right)$ und somit die Werte $\pm 1$ in jeder
  Umgebung von $(0,0)\in\R^2$ an.
  
  Es gibt also kein $\delta > 0$ so, dass $[0,\delta)$ durch $p_2\circ\alpha$
  ganz in $(-\frac{1}{2},\frac{1}{2})$ abgebildet wird. Alle offenen Umgebungen
  von $(0,0)$ in $\overline{X}$ sind aber von der Form $[0,\delta)$ mit $\delta
  > 0$ und somit ist $p_2\circ\alpha$ nicht stetig.\dipper
\end{enumerate}
\end{bem}

\begin{prop}
\label{prop:2.2.26}
Seien $(X,\OO_X)$ und $(Y,\OO_Y)$ topologische Räume, $f: X\to Y$ stetig und
ist $X$ wegzusammenhängend, so auch $\im f = f(X)$.\fishhere
\end{prop}
\begin{proof}
\begin{enumerate}[label=\arabic{*}.)]
  \item Sei $X$ wegzusammenhängend und $a,b\in Y$. Ohne Einschränkung ist $\im
  f = Y$. Dann folgt $\exists u,v\in X$ mit $f(u) = a$ und $f(v) = b$ und
  $\alpha: [0,1]\to X$ stetig mit $\alpha(0) = u$ und $\alpha(1) = v$. Dann ist
  $f\circ\alpha:[0,1]\to Y$ stetig und daher Weg von $f\circ\alpha(0) = f(0) =
  a$ nach $f\circ\alpha(1) = f(v) = b$. Also ist $Y$ wegzusammenhängend.
  \item Sei $Y$ nicht zusammenhängend, $Y=A\dot{\cup} B$ mit $A,B\in\OO_Y$,
  dann ist
  \begin{align*}
    X = f^{-1}(A)\dot{\cup}f^{-1}(B) = f^{-1}(A\cup B) = f^{-1}(Y),
  \end{align*}
  disjunkte Vereinigung der offenen, nichtleeren Mengen $f^{-1}(A)$ und
  $f^{-1}(B)$. Also ist auch $X$ nicht zusammenhängend.
  %TODO: Warum zeigen wir das überhaupt?
  \dipper\qedhere
\end{enumerate}
\end{proof}

\begin{bsp}
\label{bsp:2.2.27}
Da $\R$ mit der natürlichen Topologie zusammenhängend ist, gilt der
Zwischenwertsatz. Dann ist $f: X\to Y$ stetig, $X$ zusammenhängend und seien
$x,y\in X$ mit $f(x) = a,\;f(y)=b$ mit $a<b$. Sei $a<c<b$. Angenommen $c\notin
\im f$, dann ist
\begin{align*}
\underbrace{f^{-1}(-\infty,c)}_{\in\OO_X}\cap
\underbrace{f^{-1}(c,\infty)}_{\in\OO_X}=\varnothing \text{ und }
f^{-1}(-\infty,c)\cup f^{-1}(c,\infty) \supseteq f^{-1}(\im f) = X.
\end{align*}
Also haben wir eine disjunkte Zerlegung von $X$ in nichtleere offene
Teilmengen.\dipper

Versieht man dagegen $\R$ mit der von den Intervallen $[a,b)$ erzeugten
Topologie. Dann ist $f:\R\to\R$ mit $f(x)=1$ für $x\ge 0$ und $f(x) = 0$ für
$x<0$ stetig, da $f^{-1}(0) = (-\infty,0)$ und $f^{-1}(1) = [0,\infty)$ beide
offen und abgeschlossen sind.
\begin{align*}
\ [0,\infty)&= \bigcup_{b>0} [0,b) \Rightarrow [0,\infty) \text{ offen},\\
\Rightarrow & (-\infty,0) = \R\setminus[0,\infty) \text{ ist abgeschlossen},\\
(-\infty,0) &= \bigcup_{a<} [a,0) \text{ ist offen}.\bsphere
\end{align*}
\end{bsp}

\begin{defn}
\label{defn:2.2.28}
Sei $(X,\OO_X)$ topologischer Raum, dann heißt $X$ \emph{lokal zusammenhängend
(lokal wegzusammenhängend)}, falls jede offene Umgebung eines Punktes $x\in X$
eine zusammenhängende (wegzusammenhängende) Umgebung von $x$ enthält.\fishhere
\end{defn}

\subsection{Kompaktheit}

\begin{defn}
\label{defn:2.3.1}
Sei $(X,\OO_X)$ ein topologischer Raum. Eine \emph{offene Überdeckung}
einer Teilmenge $A\subseteq X$ ist ein System $\left(U_i\right)_{i\in\II}\subseteq \OO_X$ von offenen Teilmengen $U_i$ von
$X$, so dass $A\subseteq \bigcup_{i\in\II} U_i$.\fishhere
\end{defn}

\begin{bemn}
Die obige Definition ist äquivalent zu $A=\bigcup_{i\in\II} (U_i\cap A)$, da
$U_i\cap A\in\OO_A$ und $(A,\OO_A)\le (X,\OO_X)$.\maphere
\end{bemn}

\begin{defn}
\label{defn:2.3.2}
Eine Teilmenge $A\subseteq X$ heißt \emph{quasikompakt}, wenn es zu jeder
offenen Überdeckung $\left(U_i\right)_{i\in\II}$ von $A$ eine endliche Auswahl
$V_1,\ldots,V_k$ mit $V_j\in\left(U_i\right)_{i\in\II}$ gibt, so dass
$A\subseteq\bigcup_{j=1}^k V_j$.

In diesem Fall sagen wir auch, dass $A$ die \emph{Heine-Borellsche
Überdeckungseigenschaft} besitzt.

Ein quasikompakter $T_2$-Raum heißt \emph{kompakt}.\fishhere
\end{defn}

\begin{bemn}
Ob $A$ als Teilmenge von $X$ oder als topologischer Raum mit Spurtopologie
$\OO_A = \setdef{U\cap A}{U\in\OO_X}$ quasikompakt ist, ist ein und
dasselbe.\maphere
\end{bemn}

\begin{prop}
\label{prop:2.3.3}
(Quasi-)Kompaktheit ist eine topologische Invariante.\fishhere
\end{prop}
\begin{proof}
Trivial.\qedhere
\end{proof}

Damit kommen wir unserem Ziel, der Unterscheidung zweier topologischer Räume
bis auf Homöomorphie, ein ganzes Stück näher, denn es stellt sich äußerst
schwierig heraus, direkt zu zeigen, dass es keinen Homöomorphismus zwischen
zwei Räumen gibt, unterscheiden sich die Räume jedoch bezüglich
(Quasi-) Kompaktheit, dann sind sie nicht homöomorph.

\begin{prop}
\label{prop:2.3.4}
\begin{enumerate}
  \item\label{prop:2.3.4:1} Sei $X$ kompakt und $X\supseteq A\in\AA_X$, dann ist
  $A$ kompakt.
  \item\label{prop:2.3.4:2} Sei $X$ ein topologischer $T_2$-Raum und $A$
  kompakte Teilmenge von $X$, dann ist $A\in\AA_X$.
\end{enumerate}

Eine Teilmenge eines kompakten Raumes ist also genau dann kompakt, wenn sie
abgeschlossen ist.\fishhere
\end{prop}

\begin{proof}
\begin{enumerate}
  \item Sei $A\subseteq X$ abgeschlossen und $X$ kompakt, dann folgt
  $X\setminus A \in\OO_X$. Sei nun $\left(U_i\right)_{i\in\II}\subseteq \OO_X$
  eine offene Überdeckung von $A$, dann ist
\begin{align*}
X=A\cup (X\setminus A)\subseteq   \left(U_i\right)_{i\in\II} \cup (X\setminus A) 
\end{align*}  
eine offene Überdeckung von $X$, also gibt es $V_1,\ldots,V_k\in
\left(U_i\right)_{i\in\II}$, so dass $X=\left(\bigcup_{j=1}^k V_j \right)\cup
X\setminus A$. Wegen $A\subseteq X$ und $A\cap (X\setminus A) = \varnothing$ folgt, dass $A\subseteq
  \bigcup_{j=1}^k V_j$. Also ist $A$ kompakt.
  \item Sei $X$ topologischer $T_2$-Raum und $A\subseteq X$ kompakt. Sei $x\in
  X\setminus A$ und $y\in A$, dann existieren offene Umgebungen $U_y(x)$ und
  $U(y)$ von $x$ bzw. $y$ mit $U_y(x)\cap U(y) = \varnothing$.
  
  Dabei bildet $\bigcup_{y\in A} U(y)$ eine offene Überdeckung von $A$ und da
  $A$ kompakt ist, existieren somit endlich viele $y_1,\ldots,y_k$, sodass
  $A\subseteq \bigcup_{i=1}^k U(y_i)$. Sei $U_{y_j}(x) = V_j\in
  \UU_x\cap\OO_X$, dann ist $V_j$ offene Umgebung von $x$ mit $V_j\cap U(y_j) =
  \varnothing$ und damit ist $U=\bigcap_{j=1}^k V_j$ eine offene Umgebung von
  $x$ und $U\cap U(y_j) = (\bigcap_{j=1}^k V_j)\cap U(y_j) =\varnothing$ für
  $j=1,\ldots,k$. Also gilt
  \begin{align*}
  U\cap A \subseteq U\cap \left(\bigcup_{j=1}^k U(y_j) \right) =
  \bigcup_{j=1}^k \underbrace{\left(U\cap U(y_j)\right)}_{=\varnothing}  =
  \varnothing.
  \end{align*}
$U$ ist also eine Teilmenge von $X\setminus A$, somit enthält $X\setminus A$
mit jedem Punkt $x\in X\setminus A$ auch eine Umgebung von $x$ und ist daher
offen in $X$, also ist $A$ abgeschlossen.\qedhere
\end{enumerate}
\end{proof}

\begin{prop}
\label{prop:2.3.5}
Ein kompakter Raum ist normal.\fishhere
\end{prop}
\begin{proof}
Hausaufgabe (siehe \ref{prop:2.3.4}).\qedhere
\end{proof}

\addtocounter{prop}{1}

\begin{prop}[Satz von Tychonoff]
\label{prop:2.3.7}
Ein topologisches Produkt von beliebig vielen (quasi-) kompakten Räumen ist
(quasi-) kompakt.\fishhere
\end{prop}
\begin{proof}
\begin{enumerate}[label=\arabic{*}.)]
  \item Das topologische Produkt von Hausdorffräumen $X_\alpha (\alpha\in\AA)$
  mit $\AA$ Indexmenge ist ein $T_2$-Raum. Denn seien $x=(x_\alpha)$ und
  $y=(y_\alpha)$ Elemente von $X = \prod\limits_{\alpha\in\AA} X_\alpha$ mit
  $x\neq y$, dann gibt es ein $\beta\in\AA$ mit $x_\beta\neq y_\beta$ und
  $x_\beta,y_\beta \in X_\beta$. Weil $X_\beta$ ein $T_2$-Raum ist, finden wir
  offene disjunkte Umgebungen $U$ von $x_\beta$ und $V$ von $y_\beta$. Dann
  sind $p_\beta^{-1}(U)$ und $p_\beta^{-1}(V)$ offene Umgebungen von $x$ bzw.
  $y$ mit $p_\beta^{-1}(U) \cap p_\beta^{-1}(V) = p_\beta^{-1}(U\cap V) =
  p_\beta^{-1}(\varnothing) = \varnothing$, also ist $X$ ein $T_2$-Raum.
  \item Wir zeigen nun, dass das topologische Produkt zweier (quasi-)kompakter
  Räume (quasi-)kompakt ist. Per Induktion folgt dann der Satz für endliche
  topologische Produkte.
  
  Seien also $X,Y$ topologische Räume und $Z=X\times Y$ ihr topologisches
  Produkt.
  Sei $\left(U_i\right)_{i\in\II}\subseteq \OO_Z$ eine offene Überdeckung von
  $Z$. Da jede offene Menge von $Z$ eine Vereinigung von offenen Mengen einer
  Basis der Topologie $\OO_Z$ sind, können wir annehmen, dass die $U_i$
  Elemente einer Basis von $\OO_Z$ sind, d.h. $U_i = A_i\times B_i$ mit
  $A_i\in\OO_X$ und $B_i\in\OO_Y$ für $i\in\II$.
  
  Sei $x\in X$. Wir betrachten
  $\{x\}\times Y = \setdef{(x,y)\in Z}{y\in y}\le Z$. Eine leichte Übung zeigt,
  dass die Abbildung $p_Y\big|_{\{x\}\times Y}: \{x\}\times Y\to Y$ ein
  Homöomorphismus ist.
  
  Da $Y$ kompakt ist, ist $\{x\}\times Y$ ebenfalls kompakte Teilmenge von $Z$
  und $\left(U_i\right)_{i\in\II} $ überdeckt ganz $X\times Y$, also auch die
  kompakte Teilmenge $\{x\}\times Y$. Wir finden daher $U_1^{x},\ldots,U_k^{x}$
  in dem System der $\setdef{U_i}{i\in\II}$ mit $k=k_x\in\N$ so, dass gilt
  $\{x\}\times Y\subseteq \bigcup_{\nu=1}^k U_\nu^x$.
  
  Ohne Einschränkung ist für $1\le \nu\le k$ der Durchschnitt
  $U_\nu^x\cap\left(\{x\}\times Y\right)\neq\varnothing$, sonst kann $U_\nu^x$
  aus der endlichen Überdeckung entfernt werden.
\begin{align*}
U_\nu^x = A_\nu\times B_\nu,\
A_\nu = A_{j\nu},\ U_\nu^x = U_{j\nu}\text{ mit } A_\nu\in\OO_X \text{ und }B_\nu\in\OO_Y.
\end{align*}
  Dann ist $x\in A_\nu, \forall \nu=1,\ldots,k$, also ist $\bigcap_{\nu=1}^k
  A_\nu$ eine offene Überdeckung von $X$.
  
  Nach Konstruktion der Produkttopologie ist $A^x\times Y = \bigcup_{\nu=1}^k
  U_\nu^x$. Nun ist $x\in A^x\in\OO_X$ und daher ist $X=\bigcup_{x\in X} A^x$
  eine offene Überdeckung von $X$. Es gibt also ein $n\in\N$ und
  $x_1,\ldots,x_n\in X$, sodass $X=\bigcup_{j=1}^n A^{x_j}$ ist.
  
  Wir haben daher $X\times Y = \left(A^{x_1}\times Y\right) \cup \ldots\cup
  \left(A^{x_n}\times Y \right)$ und daraus folgt, dass
  $X\times Y = \bigcup_{j=1}^n\bigcup_{\nu=1}^n A_\nu^{x_j}\times B_\nu^{x_j} =
  \bigcup_{j=1}^n\bigcup_{\nu=1}^k U_\nu^{x_j}$.
  
  Wir haben daher endlich viele $U_i$ mit $i\in\II$ gefunden, die ganz $X\times
  Y$ überdecken, d.h. $X\times Y$ ist quasikomapkt.
  \item Der Fall für topologische Produkte von endlich vielen (quasi-)kompakten
  topologischen Räumen folgt per Induktion wegen $(X\times Y)\times Z = X\times
  (Y\times Z) = X\times Y\times Z$ für topologische Produkte.
  \item Für unendliche topologische Produkte (quasi-)kompakter Räume ist das
  Ergebnis kontraintuitiv und äquivalent zum Zorn'schen Lemma. Wir verschieben
  den Beweis auf das Ende der Vorlesung.\qedhere
\end{enumerate}
\end{proof}

Die Umkehrung von Satz \ref{prop:2.3.7} gilt ebenfalls und folgt leicht aus
folgendem Satz:
\begin{prop}
\label{prop:2.3.8}
Seien $X,Y$ topologische Räume und $f:X\to Y$ stetig. Ist $X$ quasikompakt, so
auch $f(X)=\im f$. Ist $Y$ darüber hinaus $T_2$-Raum, so ist $f(X)$
kompakt.\fishhere
\end{prop}
\begin{proof}
Sei $\left(U_i\right)_{i\in\II}$ eine offene Überdeckung von $f(X)$. Wegen
$f^{-1}(f(X)) = X$ ist dann $X=\bigcup_{i\in\II} f^{-1}(U_i)$ ebenfalls offene
Überdeckung von $X$.

Also gibt es $V_1,\ldots,V_k\in\setdef{U_i}{i\in\II}$ mit $X=\bigcup_{\nu=1}^k
f^{-1}(V_\nu)$ also ist $f(X)\subseteq \bigcup_{\nu=1}^k V_\nu$ und damit ist
$f(X)$ quasikompakt.

Ist $Y$ ein $T_2$-Raum, so auch der Unterraum $f(X)$ und daher ist $f(X)$
kompakt.\qedhere
\end{proof}

\begin{cor}
\label{prop:2.3.9}
\begin{enumerate}
  \item Sei das topologische Produkt $X=\prod\limits_{i\in\II} X_i$ der
  topologischen Räume $X_i$ quasikompakt, dann ist $X_i$ quasikompakt für alle
  $i\in\II$.
  \item Sei das topologische Produkt $X=\prod\limits_{i\in\II} X_i$ der
  topologischen Räume $X_i$ ein $T_2$-Raum, dann ist $X_i$ ein $T_2$-Raum für
  alle $i\in\II$.\fishhere
\end{enumerate}
\end{cor}
\begin{proof}
\begin{enumerate}
  \item Sei $i\in\II$, dann ist $p_i: X\to X_i$ stetig und surjektiv, also ist
  $p_i(X) = X_i$ quasikompakt (nach \ref{prop:2.3.8}).
  \item Hausaufgabe.\qedhere
\end{enumerate}
\end{proof}

\begin{cor}
\label{prop:2.3.10}
Ein topologisches Produkt ist genau dann quasikompakt, wenn es alle Faktoren
sind.\fishhere
\end{cor}

\addtocounter{prop}{1}

\begin{prop}
\label{prop:2.3.12}
Eine Teilmenge $X$ des $\R^n$ ist genau dann kompakt, wenn sie abgeschlossen
und beschränkt ist.\fishhere
\end{prop}
\begin{proof}
``$\Leftarrow$'': Dies ist gerade der Überdeckungssatz von Heine-Borell aus der
Analysis.

``$\Rightarrow$'': Sei $X$ kompakt. Nach \ref{prop:2.3.4} ist $X$ abgeschlossen
in $\R^n$. Betrachte die offenen Kugeln um $0$ mit Radius $k\in\N$
\begin{align*}
B(k) = \setdef{x\in\R^n}{\abs{x}<k},
\end{align*}
dann ist $X\subseteq \R^n=\bigcup_{k\le1} B(k)$ eine offene Überdeckung von
$X$. Da $X$ kompakt ist, genügen endlich viele dieser Kugeln. Da diese
ineinanderliegen, finden wir ein $k\in\N$ so, dass $X\subseteq B(k)$ ist. Dann
ist $X$ beschränkt.\qedhere
\end{proof}

\begin{cor}
\label{prop:2.3.13}
Sei $X$ kompakt und $f:X\to\R$ stetig. Dann ist $f(X)$ beschränkt und nimmt
sein Minimum und Maximum an.\fishhere
\end{cor}
\begin{proof}
$f(X)\subseteq\R$ ist das Bild einer kompakten Menge unter einer stetigen
Abbildung, also kompakt und damit nach
\ref{prop:2.3.12} abgeschlossen und beschränkt. Daher sind $\sup f(X),\inf
f(X)\in f(X)$.\qedhere
\end{proof}

\begin{prop}
\label{prop:2.3.14}
Sei $f:X\to Y$ stetig und bijektiv, $X$ kompakt und $Y$ ein $T_2$-Raum, dann
ist $f$ ein Homöomorphismus.\fishhere
\end{prop}
\begin{proof}
$Y=f(X)$ ist kompakt nach \ref{prop:2.3.8}. Zu zeigen ist, dass $g = f^{-1}:
Y\to X$ stetig ist. Sei also $B\subseteq X$ offen. Zu zeigen ist, dass
$g^{-1}(B)\in\OO_Y$. Nun ist $A=X\setminus B$ abgeschlossen und daher kompakt
nach \ref{prop:2.3.4}. Also ist auch
\begin{align*}
f(A) = g^{-1}(A) = g^{-1}(X\setminus B) =
Y\setminus g^{-1}(B)
\end{align*}
kompakt und daher abgeschlossen in $Y$. Also ist
$g^{-1}(B)$ offen in $Y$ und $g$ ist stetig.\qedhere
\end{proof}

\begin{defn}
\label{defn:2.3.15}
Sei $X$ ein $T_2$-Raum, $Y$ ein topologischer Raum. Für eine kompakte Teilmenge
$K$ von $X$ und $U\in\OO_Y$ sei,
\begin{align*}
M(K,U) = \setdef{f\in C(X,Y)}{f(K)\subseteq U}.
\end{align*}
Die von $\setdef{M(K,U)}{K\subseteq X\text{ kompakt}, U\in\OO_Y}$ erzeugte
Topologie der Menge
\begin{align*}
C(X,Y) = \setdef{f:X\to Y}{f\text{ ist stetig}},
\end{align*}
heißt \emph{kompakt offene Topologie}. $C(X,Y)$ zusammen mit dieser Topologie wird
mit \emph{$C_{C,O}(X,Y)$} bezeichnet.\fishhere
\end{defn}

\begin{prop}[Probleme]
\label{prop:2.3.16}
\begin{enumerate}
  \item Seien $U_1,\ldots,U_n\in\OO_Y$, $K\subseteq Y$ kompakt, dann ist
  \begin{align*}
  M(K,U_1)\cap\ldots\cap M(K,U_k) = M(K,U_1\cap\ldots\cap U_k).
  \end{align*}
  \item Sei $K\subseteq X$ kompakt, $U_i$ mit $i\in\II$ offene Mengen in $Y$,
  dann ist
  \begin{align*}
  \bigcup_{i\in\II} M(K,U_i) \subseteq M(K,\bigcup_{i\in\II} U_i).
  \end{align*}
  \item Sei $K\subseteq L\subseteq X$ kompakt, $U\in\OO_Y$,
  dann ist $M(L,U)\subseteq M(K,U)$.\fishhere
\end{enumerate}
\end{prop}

\begin{prop}[Probleme]
\label{prop:2.3.17}
Sei $X$ kompakt und $Y=\R$.
\begin{enumerate}
  \item Eine Subbasis der kompakt offenen Topologie auf $C(X,\R)$ ist gegeben
  durch
  \begin{align*}
   \setdef{M(K,(a,b))}{K\subseteq X\text{ kompakt}, a,b\in\R}.
  \end{align*}
\item Ist $f:X\to\R$ stetig, dann bilden die Mengen
\begin{align*}
\setdef{f+\ph}{\ph\in M(X,(-\ep,\ep))},
\end{align*}
eine Basis des Umgebungsfilters $\UU_f$.\fishhere
\end{enumerate}
\end{prop}
\begin{proof}
\begin{enumerate}[label=\arabic{*}.)]
  \item\label{proof:2.3.17:1} Es genügt zu zeigen, dass für $K\subseteq X$
  kompakt und $O\in\OO_\R$, $M(K,O)$ Vereinigung von endlichen Schnitten von Mengen aus der Subbasis ist
  oder äquivalent dazu, dass für $f\in M(K,O)$ eine kompakte Menge $K_i\subseteq
  X$ und $a_i,b_i\in\R$ für $i=1,\ldots,k$ existiert, so dass 
  \begin{align*}f\in
  \bigcup_{i=1}^k M(K_i,(a_i,b_i))\subseteq M(K,O).
  \end{align*}
$O$ ist offen in $\R$, also gibt es $a_i,b_i\in\R\cup\{\pm\infty\}$ mit
\begin{align*}
O = \bigcup_{i\in\II} (a_i,b_i)\text{ und }K\subseteq f^{-1}(O) =
f^{-1}(\bigcup_{i\in\II}(a_i,b_i)) = \bigcup_{i\in\II} f^{-1}(a_i,b_i).
\end{align*}
Nun ist $f^{-1}(a_i,b_i)\in\OO_X$, also gibt es eine endliche Anzahl von
Intervallen mit $K\subseteq \bigcup_{i=1}^N f^{-1}(a_i,b_i)$, da $K$ kompakt
ist.

Es folgt, dass $f(K)\subseteq \bigcup_{i=1}^k (a_i,b_i)$ und wir können
annehmen, dass
\begin{itemize}
  \item die Intervalle paarweise disjunkt sind.
  \item $a,b\in\R$ existieren mit $a_i,b_i\in[a,b]$, da $f(K)$ kompakt ist.
\end{itemize}
$f(K)$ ist abgeschlossen in $\R$, da kompakt, also gibt es ein $\ep >0$ mit
\begin{align*}
f(K)\subseteq \bigcup_{i=1}^k [a_i+\ep,b_i-\ep].
\end{align*}
Sei $K_i = f^{-1}([a_i+\ep,b_i-\ep])\cap K$. $f^{-1}([a_i+\ep,b_i-\ep])$ ist
abgeschlossen in $X$, also ist $K_i$ abgeschlossen in $K$ und es gilt
$K=\bigcup_{i=1}^k K_i$.
\begin{align*}
\bigcap_{i=1}^k M(K_i,(a_i,b_i)) \subseteq M(K,O),\\
f(K_i)\subseteq (a_i,b_i), \forall i=1,\ldots,k,\\
\Rightarrow f(\bigcup_{i=1}^k K_i) \subseteq \bigcup_{j=1}^k (a_j,b_i)\subseteq
O.
\end{align*}
Weiter ist $f(K_i)\subseteq [a_i+\ep,b_i-\ep]\subseteq (a_i,b_i)$, d.h. $f\in
M(K_i,(a_i,b_i))$ für alle $i=1,\ldots,k$.
\item Wir wollen nun zeigen, dass die $\ep$-Schläuche um $f$ eine Basis
des Umgebungsfilters um $f$ bilden. Dazu zeigen wir
\begin{enumerate}[label=(\alph{*})]
  \item Ist $U\in U_f$ in $C_{C,O}(X,\R)$, dann gibt es $\ep >0$, sodass der
  $\ep$-Schlauch um $f$ Teilmenge von $U$ ist.
  \item Alle $\ep$-Schläuche um $f$ sind Umgebungen von $f$.
\end{enumerate}

\begin{enumerate}[label=(\alph{*})]
  \item Sei $f\in C(X,\R)$ und $U\in U_f$, dann gibt es kompakte Mengen $K_i$
  und $a_i,b_i\in\R$ mit
  \begin{align*}
  f\in\bigcap_{i=1}^k M(K_i,(a_i,b_i))\subseteq U.
  \end{align*}
Sei also $\ep >0$ wie in \ref{proof:2.3.17:1} gewählt so, dass $f(K_i)\subseteq
[a_i+\ep,b_i-\ep]$, dann ist der $\ep$-Schlauch um $f$ in $\bigcap_{i=1}^k
M(K_i,(a_i,b_i))\subseteq U$ enthalten.
\item Wir suchen $K_i\in X$ kompakt und $O_i\in\OO_\R$ mit
\begin{align*}
f\in\bigcap_{i=1}^k M(K_i,O_i) \subseteq \setdef{f+\ph}{\ph\in M(X,(-\ep,\ep))},
\end{align*}
Sei also $x\in X$. Dann ist $O(x) := f^{-1}(f(x)-\frac{\ep}{3},
f(x)+\frac{\ep}{3})$ offen und es gilt $X=\bigcup_{x\in X} O(x)$. Da $X$
kompakt ist, gibt es $x_1,\ldots,x_n\in X$ mit $X=\bigcup_{i=1}^n O(x_i)$. Sei
$A(x) = f^{-1}([f(x)-\frac{\ep}{3}, f(x)+\frac{\ep}{3}])$, dann ist $A(x)$
abgeschlossen in $X$, also kompakt.

Behauptung: $f\in\bigcup M(A(x_i),(f(x)-\frac{\ep}{3}, f(x)+\frac{\ep}{3}))$
ist eine Teilmenge des $\ep$-Schlauchs, dann sind wir fertig.

\begin{align*}
f(A(x_i))\subseteq [f(x)-\frac{\ep}{3}, f(x)+\frac{\ep}{3}] \subseteq
(f(x)-\frac{\ep}{2}, f(x)+\frac{\ep}{2})
\end{align*}
für alle $i=1,\ldots,n$, also gilt $f\in M(A(x_i), (f(x)-\frac{\ep}{2},
f(x)+\frac{\ep}{2}))$. Sei $g\in \bigcup_{i=1}^N M(A(x_i), (f(x)-\frac{\ep}{2},
f(x)+\frac{\ep}{2}))$ und $x\in X$.

Zu zeigen ist nun, dass $\abs{g(x)-f(x)}<\ep$.

Wegen $X=\bigcup_{i=1}^N A(x_i)$ ist $x\in A(x_i)$ für ein $i$, also
\begin{align*}
f(x)\in [f(x)-\frac{\ep}{3}, f(x)+\frac{\ep}{3}],
\end{align*}
also ist
\begin{align*}
\abs{f(x)-f(x_i)} < \frac{\ep}{3}\text{ und } \abs{g(x)-f(x_i)} < \frac{\ep}{2},
\end{align*}
da $g\in M(A(x_i), (f(x)-\frac{\ep}{2},
f(x)+\frac{\ep}{2}))$, also gilt
\begin{align*}
\abs{f(x)-g(x)} \le \frac{\ep}{3} + \frac{\ep}{2} < \ep.\qedhere
\end{align*}
\end{enumerate}
\end{enumerate}
\end{proof}

\begin{defn}
\label{defn:2.3.18}
Ein $T_2$-Raum heißt \emph{lokal kompakt}, wenn jeder Punkt eine kompakte
Umgebung besitzt.\fishhere
\end{defn}

Es ist offensichtlich, dass lokale Kompaktheit ebenfalls eine topologische
Invariante ist.

\begin{prop}[Probleme]
\label{prop:2.3.19}
\begin{enumerate}
  \item Lokal kompakte Räume sind regulär. (vgl. \ref{prop:2.3.4} und
  \ref{prop:2.1.12})
  \item In einem lokal kompakten Raum enthält jede Umgebung eines Punktes eine
  abgeschlossene und daher kompakte Umgebung.
  \item Eine offene bzw. abgeschlossene Teilmenge eines lokal kompakten Raumes
  ist wieder lokal kompakt.
  \item Der Unterraum $X = \setdef{(x,y)\in\R^2}{x^2+y^2<1} \cup
  \{(1,0)\}\subseteq \R^2$ ist  nicht lokal kompakt bezüglich der Spurtopologie
  als Unterraum des lokal kompakten $\R^2$, denn $(1,0)\in X$ hat keine kompakte
  Umgebung.\fishhere
\end{enumerate}
\end{prop}

\begin{bsp}
\label{bsp:2.3.20}
$\Q\leqslant \R$ ist nicht lokal kompakt.

Sei $n\in\N$, dann sind $\Q\cap(1,\sqrt{2}-\frac{1}{n})$ und
$\Q\cap(\sqrt{2}+\frac{1}{n},2)$ offen in der Spurtopologie von $\Q$. Die
Vereinigung dieser Intervalle überdeckt $\Q\cap[1,2]$ aber jede endliche
Vereinigung spart ein Intervall
$\left(\sqrt{2}-\frac{1}{N},\sqrt{2}+\frac{1}{M}\right)$ für $N,M\in\N$ aus, das
auch rationale Zahlen enthält, also enthält diese Überdeckung keine endliche
Überdeckung von $\Q\cap[1,2]$.

Ähnlich kann man zeigen, dass kein rationales Intervall kompakt in $\Q$ sein
kann und daher keine kompakte Teilmenge von $\Q$ ein rationales Intervall
enthalten kann. Daraus folgt die Behauptung.\bsphere
\end{bsp}

\begin{prop}[Ein-Punkt-Kompaktifizierung]
\label{prop:2.3.21}
Sei $(X,\OO_X)$ ein lokalkompakter aber nicht kompakter Raum. Sei $X_\infty =
X\cup\{\infty\}$, wobei $\infty$ lediglich ein Symbol $\notin X$ bezeichnet.

Definiert man $\UU_\infty = \setdef{X_\infty\setminus K}{K\subseteq X,\;K\text{
kompakt}}$ und $\OO_\infty=\OO_X\cup\UU_\infty$, dann gilt
\begin{enumerate}
  \item 
$\OO_\infty$ definiert eine Topologie auf $X_\infty$ und $X\subseteq X_\infty$
trägt die Spurtopologie $\OO_X$.
\item $X$ liegt dicht in $X_\infty$.
\item $X_\infty$ ist kompakt.\fishhere
\end{enumerate}
\end{prop}
\begin{proof}
\begin{enumerate}
  \item Wegen \ref{prop:2.3.4} \ref{prop:2.3.4:2} gilt für $K$ kompakt 
%TODO:   Prüfen, ob die Referenz jetzt aufgelöst wird. 
\begin{align*}
&K\in\AA_X\Rightarrow U = X\setminus K \in \OO_X,\\
\Rightarrow\;& X_\infty\setminus K = U\cap \{\infty\}.
\end{align*}
$\OO_\infty$ entsteht also, indem man zu den schon in $X$ offenen Mengen noch
Komplemente kompakter Mengen von $X$ unter Hinzufügung des Punktes $\infty$
hinzufügt.
\begin{bemn}[Klar:]
\begin{enumerate}[label=(\alph{*})]
\item Ist $U\in\OO_\infty$, so folgt $U\cap X\in\OO_X$.
\item $\OO_\infty$ ist abgeschlossen gegenüber endlichen Durchschnitten und
beliebigen Vereinigungen.
\end{enumerate}
\end{bemn}

Also ist $\OO_X$ eine Topologie auf $X_\infty$ und $X\subseteq X_\infty$ trägt
die Spurtopologie von $(X_\infty,\OO_\infty)$.
\item Sei $\AA_\infty$ das Mengensystem der abgeschlossenen Teilmengen von
  $X_\infty$, dann gilt
  \begin{align*}
  A\in\AA_\infty &\Rightarrow U=X_\infty \setminus A \in\OO_\infty,\\
  &\Rightarrow U\in\OO_X \lor U\in\UU_\infty.
  \end{align*}
Ist $U\in\OO_X$, dann folgt $A=X_\infty\setminus U = \underbrace{X\setminus
U}_{\in\AA_X} \cup \{\infty\}$

Ist $U\in\OO_\infty$, dann existiert eine kompakte Menge
$K\subseteq X$, sodass $U=(X\setminus K)\cup\{\infty\}$, also ist
$A = X_\infty\setminus U = X_\infty \setminus (X\K \cup \{\infty\}) = K$.

Daher ist
\begin{align*}
\AA_\infty = \setdef{A\cup\{\infty\}}{A\in \AA_X}\cup \setdef{K\subseteq
X}{K\text{ kompakt}}.
\end{align*}
Also besteht $\AA_\infty$ aus allen kompakten Teilmengen von $X$ und aus den
Teilmengen von $X_\infty$, die man erhält, wenn man zu abgeschlossenen
Teilmengen von $X$ den Punkt $\{\infty\}$ hinzufügt.

Also ist insbesondere $X\notin \AA_\infty$ und daher $\overline{X}=X_\infty$,
d.h. $X$ ist dicht in $X_\infty$.

\item
Beachte, dass eine kompakte Teilmenge von $X$ auch als Teilmenge von $X_\infty$
kompakt ist, da $\OO_X$ die Spurtopologie von $X$ in $X_\infty$ ist.

Wegen $\infty\in X_\infty$ gibt es mindestens ein $i_o\in\II$, mit $U_{i_0} =
X_\infty \setminus K$ für eine kompakte Teilmenge $K$ von $X_\infty$. Da $K$
kompakt ist, gibt es auch ein $n\in\N$, so dass $K\subseteq U_{i_1}\cup \ldots
U_{i_n}$.
\begin{align*}
&\Rightarrow X_\infty \setminus K \cup K \subseteq U_{i_1}\cup \ldots
U_{i_n}\\
&\Rightarrow X_\infty \text{ ist kompakt}.\qedhere
\end{align*}
\end{enumerate}
\end{proof}
