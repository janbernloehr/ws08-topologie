\section{Topologische Räume und stetige Abbildungen}

\subsection{Topologische Räume}

\begin{defnn}
Sei $X$ ein topologischer Raum mit Topologie $\TT: X\to\PP\PP(X),\;x\mapsto
U_x$. Sei $A\subseteq X$, dann heißt $A$ \emph{offen in $X$} (bezgl. $\TT$),
falls $A$ Umgebung aller seiner Punkte ist.\fishhere
\end{defnn}

\begin{prop}
\label{prop:1.1.1}
$A$ ist offen in $X$ genau dann, wenn $A = A^\circ$.\fishhere
\end{prop}

\begin{prop}
\label{prop:1.1.2}
Sei $X$ topologischer Raum und sei $\OO_X\subseteq \PP(X)$ das System der
offenen Teilmengen von $X$, dann gilt
\begin{enumerate}[label=TOP \arabic{*}]
  \item $\OO_X$ ist abgeschlossen gegenüber der Vereinigung, d.h. ist $\II$
  Indexmenge, $\OO_i \in \OO_X,\;i\in \II$, so ist $\bigcup_{i\in\II}
  \OO_i \in \OO_X$.
  \item $\OO_X$ ist abgeschlossen gegenüber endlichen Durchschnitten,
  d.h. ist $\OO_i \in \OO_X,\; 1\le i\le k$, so ist $\OO_1 \cap \ldots
  \cap \OO_k \in \OO_X$.\fishhere
\end{enumerate}
\end{prop}

\begin{bemn}
Insbesondere sind $\bigcup_{i\in\varnothing} = \varnothing \in \OO_X$ und
$\bigcap_{i\in\varnothing} \OO_i = X \in \OO_X$.\maphere
\end{bemn}
\begin{prop}
\label{prop:1.1.3}
Sei jetzt umgekehrt $\OO_X\subseteq \PP(X)$ ein System von Teilmengen von $X$
(genannt ``offene Mengen''), das abgeschlossen gegenüber endlichen
Durchschnitten und beliebigen Vereinigungen ist, d.h. TOP 1 und TOP 2 genügt.

Für $x\in X$ und $U\subseteq X$ definieren wir: $U$ ist Umgebung von $X$, falls
es ein $\OO\in\OO_X$ gibt mit $x\in\OO$ und $\OO\subseteq U$.

Sei $U_x = \setdef{U\subseteq X}{U\text{ ist Umgebung von } X}$. Dann definiert
\begin{align*}
\TT: X\to\PP(X),\;x\mapsto U_X,
\end{align*}
eine Topologie $\TT$ auf $X$. Die offenen Mengen bezüglich $\TT$ sind exakt die
Mengen in $\OO_X$.\fishhere
\end{prop}

\begin{defnn}
Sei $X$ Menge, $\OO_X\subseteq \PP(X)$ ein System von Teilmengen von $X$, das
TOP~1 und TOP~2 genügt, dann heißt $\OO_X$ \emph{Topologie auf $X$} und
\emph{$(X,\OO_X)$ topologischer Raum}. Die Elemente von $\OO_X$ heißen
\emph{offene Mengen}.\\
Eine Teilmenge $\AA\subseteq X$, deren mengentheoretisches Komplement
$X\setminus \AA$ offen ist, heißt
\emph{abgeschlossen in $X$}.\fishhere
\end{defnn}

\begin{bemn}
$\varnothing = X\setminus X \Rightarrow \varnothing$ ist abeschlossen.\\
$X = X\setminus \varnothing \Rightarrow X$ ist abeschlossen.\maphere
\end{bemn}

\begin{prop}
\label{prop:1.1.4}
Sei $X$ Menge und $\AA_X\subseteq \PP(X)$ mit
\begin{enumerate}[label=TOP \arabic{*}']
  \item $\AA_X$ ist abgeschlossen gegenüber Durchschnitten,
  \item $\AA_X$ ist abgeschlossen gegenüber endlichen Vereinigungen.
\end{enumerate}
Dann wird durch $\OO_X = \setdef{\OO\subseteq X}{X\setminus \OO\in \AA_X}$ eine
Topologie auf $X$ definiert, deren System aus abgeschlossenen Teilmengen von $X$
exakt $\AA_X$ ist.

TOP 1' und TOP 2' sind in einem topologischen Raum mit offenen Mengen $\OO_X$
und zugehörigen abgeschlossen Mengen $\AA_X$ immer erfüllt.\fishhere
\end{prop}

Die Möglichkeiten Topologien auf Mengen zu definieren sind zahlreich - vielmehr
zahllos. Glücklicherweise sind die drei Definitionen die wir bereits gesehen
haben äquivalent.

\begin{prop}[Topologische Räume]
\label{prop:1.1.5}
Folgende Definitionen erzeugen äquivalente Topologien:
\begin{enumerate}
  \item $\TT: X\to\PP\PP(X), z\mapsto U_z$ \ref{defn:0.3.1},
  \item $\OO_X\subseteq \PP(X)$ mit TOP 1 und TOP 2 \ref{prop:1.1.2},
  \item $\AA_X \subseteq \PP(X)$ mit TOP 1' und TOP 2'
  \ref{prop:1.1.4}.\fishhere
\end{enumerate}
\end{prop}

Im Folgenden werden wir uns topologische Räume $X$ immer zusammen mit $\TT,
\OO_X$ und $\AA_X$ denken.

\begin{defnn}
$(X,\OO_X)$ ist \emph{hausdorffsch}, falls gilt: Sind $x,y\in X$ mit $x\neq y$,
so gibt es $U_1\in U_x$ und $U_2\in U_y$ mit $U_1,U_2$ offen und $U_1\cap U_2 =
\varnothing$.
\end{defnn}

Selbstverständlich sind nicht alle topologischen Räume auch hausdorffsch. Dies
ist eine Eigenschaft, die wir explizit fordern müssen. Die meisten
topologischen Räume, mit denen wir arbeiten werden, sind es aber. Eine Ausnahme
bildet beispielsweise die Zariski Topologie, die in der algebraischen Topologie
eine große Rolle spielt.

\begin{bsp}
\label{prop:1.1.6}
\begin{enumerate}[label=\arabic{*}.]
  \item Sei $X$ Menge, dann wird durch $\OO_X$ = $\{\varnothing, X\}
  \Rightarrow \AA_X = \OO_X$ die \emph{indiskrete} Topologie definiert. Sie ist
  die gröbste Topologie auf $X$.
  \item
  Sei $X$ Menge, dann wird durch $\OO_X = \PP(X) = \AA_X = \setdef{U_x}{x\in X}$
  die \emph{diskrete} Topologie definiert. Sie ist
  die feinste Topologie auf $X$. 
  \item Sei $(X,d)$ metrischer Raum und $x\in X$. Die ``offene Kugel''
  $U_\delta(x)$ mit Radius $\delta> 0$ um $x$ ist $U_\delta(x) = \setdef{z\in
  X}{d(x,z) < \delta}$. Dann sind die $\delta$-Kugeln um Elemente von $X$ die
  offenen Mengen einer Topologie auf $X$. Diese nennt man die \emph{von $d$
  induzierte} Topologie.
\begin{proof}
Beliebige Vereinigungen von $\delta$-Kugeln um Elemente von $X$ sind die
offenen Mengen (TOP 1).

Für TOP 2 genügt es zu zeigen, dass der Durchschnitt zweier offener Mengen
wieder offen ist, denn seien $\bigcup\limits_\alpha
U_{\delta_\alpha}(x_\alpha),\;\bigcup\limits_\beta
U_{\delta_\beta}(x_\beta)$ zwei offene Mengen in $X$, dann gilt
\begin{align*}
\left(\bigcup\limits_\alpha U_{\delta_\alpha}(x_\alpha) \right) \cap \left(\bigcup\limits_\beta
U_{\delta_\beta}(x_\beta) \right)
 = \bigcap\limits_{\alpha,\beta} \left(\underbrace{U_{\delta_\alpha}(x_\alpha)
 \cup U_{\delta_\beta}(x_\beta)}_{\text{Vereinigung offener Kugeln}}
 \right).
\end{align*}
Sei nun $M = U_\delta(x)\cap U_\ep(z)$ mit $\delta, \ep >0,\;x,y\in X$ und
\begin{align*}
\rho_y = \min\{\delta - d(x,y),\ep - d(z,y)\},
\end{align*}
dann ist $U_{\rho_y}(y) \subseteq U_\delta(x)\cap U_\ep(z) = M$ und daher ist
$M = \bigcup\limits_{y\in M} U_{\rho_y}(y)$.\qedhere
\end{proof}
\item
\begin{defnn}
Zwei Metriken $d_1$ und $d_2$ auf $X$ heißen \emph{topologisch äquivalent},
falls sie dieselbe Topologie auf $X$ induzieren.\fishhere
\end{defnn}
Auf dem $\R^n$ sind die Metriken
\begin{enumerate}
  \item $d_p(x,y) = \sqrt[p]{ \sum\limits_{i=1}^n (x_i-y_i)^p}$ für $p\ge 2$
  \item $d_\infty(x,y) = \max\limits_{i=1,\ldots,n} \abs{x_i-y_i}$.
\end{enumerate}
topologisch äquivalent. Ihre Umgebungen sind
\begin{enumerate}
  \item[$d_2$)] $U_\delta(x) := $ offene Kugel um $x$ mit Radius $\delta$,
  \item[$d_\infty$)] $U_\delta(x) :=$ offener Würfel um $x$ mit Kantenlänge
  $2\delta$.
\end{enumerate}
\item
Auf $\R$ besteht $\OO\subseteq\PP(\R)$ aus Intervallen der Form
$(-\infty,a), a\in\R$ zusammen mit $\varnothing,\R$.
\begin{align*}
\bigcap\limits_{i=1}^n (-\infty,a_i) = (-\infty,b), & b = \min\limits_{i} a_i,\\
\bigcup\limits_{i\in \II} (-\infty,a_i) = (-\infty,b), & b = \sup\limits_i a_i.
\end{align*}
% TODO: Ich hab keinen Plan mehr, was Dipper mit folgendem sagen wollte \ldots

\begin{align*}
Y_a &= (-\infty,a]\\
U_i &= \begin{cases}
\R, & \text{ falls } a_i \text{ unbeschränkt},\\
Y_b, & \text{ falls } b = \max_i a_i \text{ existiert},\\
X_b = (-\infty,b), & \text{ sonst, mit } b = \sup a_i.
\end{cases}
\end{align*}
\item
\begin{defnn}
Sei $X$ Menge, dann heißt $A\subseteq X$ \emph{kofinit}, falls
$X\setminus A$ endlich ist.

Sei $\OO$ die leere Menge zusammen mit der Menge der kofiniten Teilmengen auf
$X$. Eine leichte Übung zeigt, dass $\OO$ wirklich eine Topologie auf $X$
definiert. Diese Topologie heißt \emph{kofinite Topologie} auf $X$.\fishhere
\end{defnn}
\begin{bemn}
Ist $X$ endlich, so ist die kofinite Topologie die diskrete.\maphere
\end{bemn}
\item
Sei $(X,\le)$ linear geordnet. Für $a\in X$ seien
\begin{align*}
K_a = \setdef{x\in X}{x< a},\;G_a = \setdef{x\in X}{x > a}.
\end{align*}
$\BB$ bestehe aus endlichen Durchschnitten von
Mengen $K_a$ und $G_b$. $\OO$ bestehe aus beliebigen Vereinigungen von Mengen
aus $\BB$.\\
Dann ist $\OO$ abgeschlossen gegenüber endlichen Durschnitten und beliebigen
Vereinigungen, d.h. $\OO$ ist Topologie auf $X$, genannt
\emph{Ordnungstopologie auf $X$}.
\item 
Sei $y = x^2$ die Normalparabel, die Gleichung lässt sich auch schreiben als
$f(x,y) = y-x^2$. Dann ist die Parabel gerade die Nullstellenmenge von $f$.

Sei $K$ Körper, $X = K^n = \setdef{(\alpha_1, \ldots, \alpha_n)\in K\times
\ldots\times K}{\alpha_i\in K}$ und $f$ aus dem Polynomring $K[x_1, \ldots,
x_n]$ in den Variablen $x_1, \ldots, x_n$.

Dann definiert $f$ eine Abbildung $\tilde{f} : K^n\to K^n$ gegeben durch
\begin{align*}
&f = \sum_{\mui} \lambda_{\mui} x^{\mui},\quad \lambda_i \text{ fast alle 0},\\
&\tilde{f}(\alpha_1, \ldots,\alpha_n) = \sum_{\mui} \lambda_{\mui}
\alpha^{\mui} \in K.
\end{align*}
Solche Abbildungen von $K^n\to K^n$ heißen \emph{polynomial}. Die Menge
\begin{align*}
Y = (K)^{K^n} = \setdef{f:K^n\to K}{f \text{ ist Abbildung}},
\end{align*}
wird zur $K$-Algebra durch
\begin{align*}
&f,g\in Y : (f+g)(x) = f(x)+g(x),\\
&(f\cdot g)(x) = f(x)\cdot g(x),\\
&(\lambda f)(x) = \lambda f(x). 
\end{align*}
Man sieht leicht, dass
\begin{align*}
\sim :\;K[x_1, \ldots, x_n] \to Y,\;f\mapsto
\tilde{f}\in Y
\end{align*}
ein $K$-Algebrahomomorphismus ist mit der Menge der
polynomialen Abbildungen von $K^n\to K$ als Bild.
\begin{propn}[Behauptung]
Ist $K$ unendlich, dann ist $\sim$ injektiv, d.h.
\begin{align*}
\ker \sim = \setdef{f\in
K[x_1, \ldots, x_n]}{f(\alpha_1, \ldots, \alpha_n) = 0, \forall \alpha_i \in
K}.\fishhere
\end{align*}
\end{propn}
\begin{proof}
Der Beweis wird in der Algebra Vorlesung geführt.\qedhere
\end{proof}

Sei nun $K$ unendlich, $p\in K[x_1,\ldots,x_n]$. $(\alpha_1,\ldots,\alpha_n)\in
K^n$ ist Nullstelle von $p$, falls $p(\alpha_1,\ldots,\alpha_n) = 0$. Sei
\begin{align*}
z(p) = \setdef{(\alpha_1,\ldots,\alpha_n)\in
K^n}{p(\alpha_1,\ldots,\alpha_n) = 0},
\end{align*}
dann ist für $p(x,y) = y-x^2\in K[x,y]$ der Graph der Standardparabel gleich
$z(p)$.

Allgemeiner: Sei $Y \subseteq K[x_1,\ldots,x_n]$ eine Familie von Polynomen,
dann ist
\begin{align*}
z(I) = \setdef{(\alpha_1, \ldots, \alpha_n)\in K^n}{f(\alpha_1,\ldots,\alpha_n)
= 0 \forall f\in Y}\subseteq K^n.
\end{align*}
Solche Teilmengen des $K^n$ heißen \emph{algebraisch}.

Umgekehrt: Sei $A\subseteq K^n$ algebraisch, $p\in K[x_1,\ldots,x_n]$, dann ist
\begin{align*}
\NN(A) &= \setdef{p}{p(\alpha_1,\ldots,\alpha_n) = 0,
\forall \alpha_1,\ldots,\alpha_n \in A},\\
\NN(A) &\ideal K[x_1,\ldots,x_n].
\end{align*}
Sei $A\subseteq B$ algebraische Teilmenge von $K^n$, dann gilt $\NN(B) \subseteq
\NN(A)$.

Seien $Y_1\subseteq Y_2$ Teilmenen von $K[x_1,\ldots,x_n]$, dann ist analog
$z(Y_2)\subseteq z(Y_1)$.

Sind $Y_i, i\in\II$ Teilmenen von $K[x_1,\ldots,x_n]$, so ist
\begin{align*}
\bigcap\limits_{i\in\II} z(Y_i) = z\left(\bigcup\limits_{i\in\II} Y_i \right), 
\end{align*}
also sind beliebige Durchschnitte von algebraischen Mengen algebraisch.

Seien $Y_1,\ldots,Y_m\subseteq K[x_1,\ldots,x_n]$ wir definieren
\begin{align*}
\prod\limits_{i=1}^m Y_i = \setdef{p_1\cdot\cdot\cdot p_m}{p_i\in Y_i},
\end{align*}
dann ist das Nullstellengebilde $\bigcup\limits_{i=1}^m z(Y_i) = z(\prod_i
Y_i)$, d.h. endliche Vereinigungen von algebraischen Mengen sind algebraisch.

Also haben wir eine Topologie auf $K^n$. Die algebraischen Mengen mit
$\varnothing$ und $K^n$ sind die abgeschlossenen Mengen dieser Topologie, diese
wird \emph{Zariski Topologie} genannt, nach dem Erfinder.

\begin{bemn}
Die abgeschlossenen Mengen in der Zariski Topologie sind ``dünn'', die offenen
``fett''. Eine offene Menge ist in keiner echten abgeschlossenen Teilmenge von
$K^n$ enthalten.

Zwei offene, nichtleere Teilmengen haben demnach nichtleeren Schnitt.
Insbesondere ist die Zariski Topologie nicht hausdorffsch.\maphere
\end{bemn}

Die algebraische Geometrie studiert algebraische Teilmengen des $K^n$ mit Hilfe
der Zariski Topologie, artet dabei aber schnell in Ringtheorie aus. Das ist
auch der Grund dafür, dass viele Geometer behaupten, die algebraische Topologie
sei keine Geometrie sondern Algebra.\bsphere
\end{enumerate}
\end{bsp}

\begin{defn}
\label{defn:1.1.7}
Sei $X$ Menge und $\OO_X$ Topologie auf $X$.
\begin{enumerate}
  \item $\BB \subseteq \OO_X$ heißt \emph{Basis} von $\OO_x$, falls alle Mengen
  in $\OO_X$ aus Vereinigungen von Mengen aus $\BB$ bestehen, d.h. $\forall
  \OO\in\OO_X \exists \BB_i\in \BB, i\in\II : O = \bigcup_{i\in\II} \BB_i$.
  \begin{bspn}
  $\setdef{U_\delta(x)}{\delta > 0,\;x\in\R^n}$ ist eine Basis der
  natürlichen Topologie des $\R^n$, ebenso die abzählbare Menge
  $\setdef{U_\delta(x)}{0<\delta\in\Q,\ x\in\Q^n\subseteq\R^n}$.\bsphere
  \end{bspn}
  Der Begriff ``Basis'' ist hier nicht so scharf zu sehen, wie in der linearen
  Algebra.
  \item $S\subseteq \OO_X$ heiß \emph{Subbasis} von $\OO_X$, falls alle Mengen
  in $\OO_X$ aus Vereinigungen von endlichen Durschnitten von Mengen in $S$
  bestehen.\fishhere
\end{enumerate}
\end{defn}

\begin{prop}
\label{prop:1.1.8}
Sei $X$ Menge, $\varnothing\neq S\subseteq \PP(X)$. Sei
\begin{enumerate}
  \item $\BB = \setdef{A\subseteq X}{\exists S_1,\ldots,S_n\in S,k\in\N : A =
  \bigcap\limits_{i=1}^n S_i}$.
  \item $\OO_x = \setdef{A\subseteq X}{\exists \BB_i\in\BB, i\in\II : A =
  \bigcup\limits_{i\in\II} \BB_i}$.
\end{enumerate}
Dann ist $\OO_x$ Topologie auf $X$ mit Basis $\BB$ und Subbasis $S$, die von
$S$ erzeugte Topologie auf $X$. Dabei ist $\OO_x$ eindeutig durch $S$
bestimmt.\fishhere
\end{prop}

Insbesondere erzeugt jede nichtleere Teilmenge der $\PP(X)$ als Subbasis eine
Topologie auf $X$.

\begin{bspn}
\begin{enumerate}[label=\arabic{*}.)]
  \item $\OO_X$ ist für einen topologischen Raum Basis und Subbasis.
  \item Offene Kugeln (Würfel) im $\R^n$ bilden eine Basis.
  \item Offene Kugeln im $\R^n$ mit rationalem Radius bilden eine Basis.
  \item Die Menge der offenen Kugeln im $\R^n$ deren Radius und
  Mittelpunktskoordianten rational sind (das sind abzählbar viele) ist eine
  Basis.
  \item $S = \setdef{(-\infty,b)}{b\in\Q}\cup \setdef{(a,\infty)}{a\in\Q}$ ist
  Subbasis von $\R$.\bsphere
\end{enumerate}
\end{bspn}

\begin{defn}
\label{defn:1.1.19}
Seien $\OO_X$ und $\tilde{\OO}_X$ Topologien auf $X$, dann heißt
$\tilde{\OO}_X$ \emph{feiner} als $\OO_X$ und $\OO_X$ \emph{gröber} als
$\tilde{\OO}_X$, falls $\OO_X \subseteq \tilde{\OO}_X$ ist.\fishhere
\end{defn}

\begin{lem}
\label{prop:1.1.10}
Sei $X$ Menge
\begin{enumerate}
  \item Die indiskrete Topologie $\{\varnothing,X\}$ ist die eindeutig
  bestimmte gröbste Topologie auf $X$ und die diskrete Topologie $\PP(X)$ ist
  die eindeutig bestimmte feinste Topologie auf $X$.
  $\{\varnothing,X\}\subseteq\OO_X\subseteq \PP(X)$.
  \item Sei $S\subseteq \PP(X)$ und $\OO_X$ die von $S$ erzeugte Topologie auf
  $X$, dann ist $\OO_X$ die gröbste Topologie, die $S$ als Teilmenge enthält.
  \item Seien $S_1\subseteq S_2\subseteq \PP(X)$ und $\OO_1$ bzw. $\OO_2$ die
  von $S_1$ bzw. $S_2$ erzeugte Topologie auf $X$. Dann ist $\OO_1$ gröber als
  $\OO_2$. Ist $S_2\subseteq \OO_1$, so ist $\OO_1=\OO_2$.
\end{enumerate}
\end{lem}

\begin{defn}
\label{defn:1.1.11}
Sei $X$ topologischer Raum, $A\subseteq X$, dann heißt $x\in X$,
\begin{enumerate}
  \item \emph{Berührpunkt (BP)} von $A$ ($\in\BP{A}$), falls gilt $U\cap A \neq
  \varnothing,\forall U\in\UU_x$.
  
  Klar: $x\in A \Rightarrow x$ BP von $A$.
  \item \emph{Häufungspunkt (HP)} von $A$ ($\in\HP{A}$), falls $(U\cap
  A)\setminus\{x\} \neq \varnothing,\forall U\in\UU_x$.
  
  Klar:
  \begin{enumerate}[label=\arabic{*}.)]
    \item HPs sind BPs ($\HP{A}\subseteq\BP{A}$).
    \item Jeder Punkt von $A$ ist BP von $A$ aber nicht notwendigerweise HP.
    \item Ist $a\notin A$ aber ein BP von $A$, so ist $a$ ein HP von $A$.
  \end{enumerate}
  Also ist $\BP{A} = A\cup\HP{A}$.
  \item \emph{isoliert} in $A$, falls $x\in A$ und $x\notin \HP{A}$.
  \item \emph{innerer Punkt} von $A$, falls $A$
  Umgebung von $x$ ist und damit $A$ eine offene Umgebung von $x$ enthält.
  \item \emph{Randpunkt} von $A$, falls $x\in\BP{A}$und $A^c =
  X\setminus A$.\\ $\{\text{Randpunkte von }A \} = \partial A = \text{Rand von
  }A$.\fishhere
\end{enumerate}
\end{defn}

\begin{defn}
\label{defn:1.1.12}
Sei $A\subseteq X$ und $X$ topologischer Raum
\begin{enumerate}
  \item Die Menge der inneren Punkte von $A$ heißt \emph{innerer Kern} von $A$
  und wird mit $A^\circ$ bezeichnet. (Siehe \ref{defn:0.3.1}).
  \item $A\cup\HP{A} = \BP{A} = \overline{A}$ heißt
  \emph{abgeschlossene Hülle} von $A$.\fishhere
\end{enumerate}
\end{defn}

\begin{lem}
\label{prop:1.1.13}
Sei $X$ topologischer Raum, $A\subseteq X$
\begin{enumerate}
  \item $\overline{A}$ ist der Durchschnitt aller abgeschlossenen Teilmengen von
  $X$, die $A$ enthalten.
\begin{align*}
\overline{A} = \bigcap\limits_{\atop{A\subseteq B\subseteq X,}{B\in\AA_X}} B
\end{align*}
Insbesondere ist $\overline{A}\in\AA_X$. In der Tat ist $\overline{A}$ die
eindeutig bestimmte kleinste abgeschlossene Teilmenge von $X$, die $A$ enthält.
\item $A^\circ$ ist die eindeutig bestimmte größte offene Menge von $X$, die in
$A$ enthalten ist und daher insbesondere offen.
\begin{align*}
A^\circ = \bigcup\limits_{\atop{C\subseteq A,}{C\in\OO_X}} C.
\end{align*}
\item $\overline{A} \setminus A^\circ = \partial A = \overline{A} \cap
\overline{A}^c$.\fishhere
\end{enumerate}
\end{lem}
\begin{proof}
\begin{enumerate}
  \item Sei $D= \bigcap\limits_{\atop{A\subseteq B\subseteq X,}{B\in\AA_X}} B$.
  Wir wollen zeigen, dass $D=\overline{A}$. Dann ist $D\in\AA_X$ und damit die
  kleinste abgeschlossene Teilmenge von $X$, die $A$ enthält. Somit folgt aus
  $\overline{A} = D \Rightarrow $ Rest von 1) 
  
``$\supseteq$'': Sei also $a\in\overline{A}$. Ist $a\in A\Rightarrow a\in D$,
  weil $A\subseteq D$. Sei $a\notin A$, dann ist $a$ ein HP von $A$ (nach
  Definition).\\
  Wäre $a\notin D$, so wäre $a\in X\setminus D = D^c$. Nun ist $D^c\in \OO_X$ als
  Komplement einer abgeschlossenen Menge $D$. Also ist wegen $a\in D^c$, $D^c$
  eine offene Umgebung von $a$ nach \ref{prop:1.1.1} und \ref{prop:1.1.3}. Dann
  ist aber $A\cap D^c\subseteq D\cap D^c = \varnothing$ und $a$ wäre kein HP von
  $A$.\dipper
  
  Also ist $a\in D$.
  
  ``$\subseteq$'': Sei $x\in D$, $x\notin A$, denn sonst ist $x\in\overline{A}$
  so oder so. Sei $x$ kein BP von $A$ also $x\notin \overline{A}$, dann gibt es
  eine Umgebung $U$ von $x$ mit $U\cap A = \varnothing$. Da nach
  \ref{prop:1.1.3} jede Umgebung von $x$ eine offene Umgebung von $x$ enthält,
  ist $U$ ohne Einschränkung offen und $A$ ist daher in der abgeschlossenen
  Menge $U^c = X\setminus U$ enthalten. Also kommt $U^c$ als eine der Mengen $B$
  im Durchschnitt $D=\bigcap_{\atop{A\subseteq B\subseteq X,}{B\in\AA_X}} B$
  vor und ist $D\subseteq U^c$. Wegen $x\in U$ ist $x\notin U^c$ und daher
  $x\notin D$.\dipper
  
  Also $x\in \overline{A}$.
  \item Übung.
  \item Nach Definition ist $\partial A = \{\text{BP von }A\}\cap \{\text{BP
  von }A^c\}$, $A^c = X\setminus A$.
  Sei $x\in \partial A$, dann ist $x\in\overline{A}$. Wäre $x\in A^\circ$, so
  gäbe es eine offene Umgebung $U$ von $x$ mit $U\subseteq A^\circ\subseteq A$.
  Dann ist aber $U\cap A^c=\varnothing$ und $x$ ist kein BP von $A^c$. Also ist
  $x\notin A^\circ$.
  
  Also ist $\partial A \subseteq \overline{A}\setminus A^\circ = \setdef{a\in
  A}{a\notin A^\circ}$. Sei $x\in \overline{A}\setminus A^\circ$. Dann ist
  insbesondere $x\in \overline{A}$ und daher BP von $A$. Wir haben z.Z. dass
  $x\in\overline{A}^c$.
  
  Sei $x\notin \overline{A}^c$. Dann gibt es eine Umgebung $U$ von $x$ mit
  $U\cap A^c =\varnothing$. Dann ist $U\subseteq A$. Also ist $A\in U_x$ und
  daher $x\in A^\circ$.\dipper
  
  Also ist $x\in\overline{A}^c$ und daher ist $x\in \overline{A}\cap
  \overline{A}^c = \partial A$.\qedhere
\end{enumerate}
\end{proof}

\begin{defn}
\label{defn:1.1.14}
Sei $X$ topologischer Raum, $A\subseteq X$
\begin{enumerate}
  \item $A$ heißt \emph{dicht} in $X$, falls $\overline{A} = X$ ist.
  \item $A$ heißt \emph{nirgends dicht} in $X$, falls
  $\left(\overline{A}\right)^\circ = \varnothing$.\fishhere
\end{enumerate}
\end{defn}

\begin{bspn}
\begin{enumerate}[label=(\alph{*})]
\item $\Q$ und $\R\setminus \Q$ sind beide dicht in $\R$, da jedes offene
  Intervall um eine rationale bzw. irrationale Zahl, irrationale bzw. rationale
  Zahlen enthält. So sind die rationalen bzw. irrationalen Zahlen HP von
  $\R\setminus \Q$ bzw. $\Q$ in $\R$.
  
  Dies ist ein Beispiel dafür, dass das Komplement einer dichten Teilmenge
  nicht zwingend nirgends dicht ist.
  \item In der Zariskitopologie auf $K^n$ ($K$ = unendlicher Körper) ist jede
  offene Teilmenge $\neq \varnothing$ dicht in $K^n$ und jede abgeschlossene
  Teilmenge $\neq X$ nirgends dicht in $X$.\bsphere
\end{enumerate}
\end{bspn}

\begin{prop}
Sei $X$ topologischer Raum und $A,B\subseteq X$, dann gilt
\begin{align*}
& (A\cap B)^\circ = A^\circ \cap B^\circ,\\
& \overline{A\cup B} = \overline{A}\cup\overline{B}.\fishhere
\end{align*}
\end{prop}

\subsection{Stetige Abbildungen}
Seien $X,Y$ topologische Räume
\begin{bemn}[Erinnerung an Definition \ref{defn:0.3.2}.]
Eine Abbildung $f: X\to Y$ heißt stetig in $x\in X$, falls gilt $f^{-1}(V)\in
\UU_x$ für alle $V\in \UU_{f(x)}$. $f$ heißt stetig, falls $f$ in allen Punkten
$x\in X$ stetig ist.\maphere
\end{bemn}

\begin{lem}
\label{prop:1.2.1}
Sei $f: X\to Y$ Abbildung, dann sind folgende Aussagen äquivalent
\begin{enumerate}
  \item $f$ ist stetig.
  \item $\forall x\in X, \forall V\in \UU_{f(x)} \exists U\in \UU_x : f(U) 
  \subseteq V$.
  \item Sei $V\in \OO_Y$, dann ist $f^{-1}(V)\in\OO_X$.
  \item Sei $B\in \AA_Y$, dann ist $f^{-1}(B)\in \AA_X$.\fishhere
\end{enumerate}
\end{lem}

\begin{bem}
\label{bem:1.2.2}
Seien $\OO_X,\OO_X'$ zwei Topologien auf $X$, dann ist
\begin{align*}
\id: (X,\OO_X) \to
(X,\OO_X'),\;x\mapsto x,
\end{align*}
genau dann stetig, wenn $\OO_X'\subseteq \OO_X$, d.h.
$\OO_X$ feiner als $\OO_X'$ ist.\maphere
\end{bem}

\begin{lem}
\label{prop:1.2.3}
Die Komposition zweier stetiger Abbildungen ist stetig.\fishhere
\end{lem}
\begin{proof}
Seien $X,Y,Z$ topologische Räume, $f:X\to Y$, $g: Y\to Z$ stetige Abbildungen.
Sei $V\in\OO_Z$, dann ist $g^{-1}(V) \in \OO_Y$ und damit ist
$(f\circ g)^{-1}(V) = f^{-1}(g^{-1}(V)) \in \OO_X$. Also ist $g\circ f$
stetig.\qedhere
\end{proof}

\begin{bemn}
Sei $B\subseteq X$. In \ref{defn:0.3.4} definierten wir die Spurtopologie auf
$B$ (über Umgebungen). Dann ist das System $\OO_B, \AA_B$ in der Spurtopologie
auf $B$ gegeben durch
\begin{align*}
\OO_B &= \setdef{O\cap B}{O\in\OO_X},\\
\AA_B &= \setdef{A\cap B}{A\in\AA_X}.\maphere
\end{align*}
\end{bemn}

\begin{defn}[Bezeichnung]
\label{defn:1.2.4}
Sei $A\subseteq X$. Wir schreiben $A\le X$, wenn wir $A$ mit der Spurtopologe
versehen und sagen $A$ ist \emph{Unterraum} von $X$.\fishhere
\end{defn}

\begin{lem}
\label{prop:1.2.5}
Sei $f: X\to Y$ stetig und $A\subseteq X$. Dann ist die Einschränkung
\begin{align*}
f\big|_A: A\to Y,
\end{align*}
stetig auf $A$.\fishhere
\end{lem}
\begin{proof}
$f\big|_A^{-1}(V) = \setdef{x\in A}{f\big|_A(x) = f(x)\in V} = f^{-1}(V) \cap A
\in \OO_A$.\qedhere
\end{proof}

\begin{bem}
\label{bem:1.2.6}
Die Umkehrabbildung ist Inklusionserhalten und daher gilt,
\begin{align*}
f^{-1}(A\cap B) &=
f^{-1}(A)\cap f^{-1}(B),\\
f^{-1}\left(\bigcup\limits_{i\in \II} A_i \right) &= \bigcup\limits_{i\in \II}
f^{-1}(A_i).
\end{align*}
Daher muss man $f^{-1}(V)\in \OO_X$ lediglich für eine (Sub)-Basis der
Topologie auf $Y$ nachprüfen, um die Stetigkeit von $f$ zu überprüfen.\maphere
\end{bem}

\begin{bemn}
$f: X\to Y$ induziert $f^{-1}: \PP(Y) \to \PP(X),\; A\mapsto f^{-1}(A)$.
Siehe Mengenlehre.\maphere
\end{bemn}

\begin{defn}
\label{defn:1.2.7}
$f: X\to Y$ heißt \emph{offen} bzw. \emph{abgeschlossen}, wenn gilt
\begin{align*}
f(A) \in \OO_Y \text{ bzw. } \in \AA_Y, \text{ für jedes }
A\in\OO_Y\text{ bzw. }\in\AA_Y.\fishhere
\end{align*}
\end{defn}

\begin{defn}
\label{defn:1.2.8}
Eine bijektive Abbildung heißt \emph{Homöomorphismus}, wenn $f: X\to Y$ und
$f^{-1}: Y\to X$ stetig sind oder äquivalent, wenn $f$ stetig und offen bzw.
abgeschlossen ist.

Wir schreiben $X\cong Y$, falls ein Homöomorphismus von $X$ nach $Y$ existiert.
$\cong$ ist eine Äquivalenzrelation auf der Klasse der topologischen
Räume.\fishhere
\end{defn}
\begin{bemn}[Beachte:]
Ist $f: (X,\OO_X)\to (Y,\OO_Y)$ ein Homöomorphismus, so ist eine Bijektion
gegeben durch $f: \OO_X\to \OO_Y,\; U\mapsto f(U)$.\maphere
\end{bemn}

\subsection{Initiale und finale Topologien}

Im Folgenden sei $\II$ stets Indexmenge und $i\in\II$.

\begin{defn}[Definition/Lemma]
\label{defn:1.3.1}
\begin{enumerate}
  \item Sei $X$ Menge, $(Y,\OO_Y)$ topologischer Raum und sei $f: X\to Y$
  Abbildung. Dann ist durch
  \begin{align*}
  \OO_X = \setdef{f^{-1}(U)\subseteq X}{U\in \OO_Y},
  \end{align*}
  eine Topologie auf $X$ definiert. Diese ist die eindeutig bestimmte gröbste
  Topologie auf $X$, so dass $f$ stetig ist und heißt \emph{Initialtopologie
  auf $X$ bezüglich $f$}. Ist $f$ surjektiv, so ist $f$ offen bezüglich der
  Initialtopologie auf $X$.
  \item Sei $(X,\OO_X)$ topologischer Raum, $Y$ Menge und $f: X\to Y$ Abbildung.
  Dann wird durch
\begin{align*}
\OO_Y = \setdef{U\subseteq Y}{f^{-1}(U)\in\OO_X},
\end{align*}
 eine Topologie auf $Y$ definiert. Sie ist die eindeutig bestimmte feinste
 Topologie auf $Y$, so dass $f: (X,\OO_X)\to (Y,\OO_Y)$ stetig ist. Sei heißt die
  \emph{finale Topologie auf $Y$ bezüglich $f$}. Ist $f$ injektiv, so ist $f$
  offen bezüglich der finalen Topologie.\fishhere
\end{enumerate}
\end{defn}

Natürlich kann man diese Definition noch verallgemeinern:

\begin{defn}[Definition/Lemma]
\label{defn:1.3.2}
\begin{enumerate}
  \item Sei $X$ Menge, $(Y_i, \OO_i)$ ein System von topologischen
  Räumen und $f_i: X\to Y_i$ Abbildungen. Sei
  $S_i = \setdef{f_i^{-1}(U)}{U\in\OO_i}$ die Menge der Urbilder offener Mengen
  unter $f_i$ und $S = \bigcup\limits_{i\in\II} S_i$ die Vereinigung aller 
  dieser Mengen, dann bildet $S$ eine Subbasis einer Topologie $\OO_X$. Diese
  ist die gröbste Topologie für die alle $f_i : (X,\OO_X)\to (Y_{i}, \OO_i)$
  stetig sind.
  
  $\OO_X$ heißt \emph{Initialtopologie bezüglich den Abbildungen $f_i$}. Sind
  die $f_i$ alle surjektiv, dann sind sie auch offen.
  \item Sei $Y$ Menge, $(X_i,\OO_i)$ ein System von topologischen
  Räumen und seien $f_i: X_i\to Y$ Abbildungen. Die \emph{finale Topologie
  $\OO_Y$ auf $Y$ bezüglich den Abbildungen $f_i$} ist definiert durch
  $U\subseteq Y$ ist offen genau dann, wenn $f_i^{-1}(U)\in\OO_i$ für alle
  $i\in\II$.\fishhere
\end{enumerate}
\end{defn}

\begin{bsp}
\label{bsp:1.3.3}
Sei $X$ topologischer Raum mit Topologie $\OO_X$.
\begin{enumerate}
  \item Sei $A\subseteq X$, dann ist die Spurtopologie die Initialtopologe
  auf $A$ bezüglich der natürlichen Inklusionsabbildung $\iota: A\to X$, denn
  \begin{align*}
  \OO_A = \setdef{\iota^{-1}(U)}{U\in \OO_X} = \setdef{U\cap A}{U\in\OO_X}.
  \end{align*}
  \item
  \begin{defnn} Sei $\sim$ eine Äquivalenzrelation auf $X$. Für $x\in X$ sei
  $\overline{x} = \setdef{z\in X}{z\sim x}$ die Äquivalenzklasse. Sei
  $\overline{X} = \setdef{\overline{x}}{x\in X} = X/\sim$ und $\pi: X\to 
  \overline{X},\;x\mapsto \overline{x}$ Abbildung von $X$ auf
  $\overline{X}$. Wir versehen $\overline{X}$ mit der finalen Topologie
  bezüglich $\pi$. Diese heißt dann \emph{Quotiententopologie} auf
  $\overline{X}$ und $(\overline{X},\OO_{\overline{X}})$ heißt
  \emph{Quotientenraum} von $X$ bezüglich $\sim$.
\end{defnn}
  \begin{bemn}[Beachte:]
  $\OO_{\overline{X}} = \setdef{\overline{A}\subseteq \overline{X}}{\setdef{a\in
  X}{\overline{a}\in\overline{A}}\in\OO_X} =
  \setdef{\overline{A}\subseteq \overline{X}}{A\in\OO_X}$.
  \end{bemn}
  \item Sei $X=\R^2$ und $x\sim y \Leftrightarrow x-y \in \Z^2\subseteq \R^2$
  % HA: Über dies nachdenken.
  \bsphere
\end{enumerate}
\end{bsp}

\subsection{Topologische Summen und Produkte}
\begin{defn}
\label{defn:1.4.1}
Sei $\setdef{X_i}{i\in\II}$ Familie von topologischen Räumen
$(X_i,\OO_i)$ und $\II$ Indexmenge. Sei $X=\prod\limits_{i\in\II} X_i$ das
karthesische Produkt der Menge $X_i$,
\begin{align*}
\prod\limits_{i\in\II} := \setdef{f:I\to
\bigcup_{i\in\II}X_i}{f(i)\in X_i}\footnote{Das Auswahlaxiom garantiert,
dass diese Menge nicht leer ist.}.
\end{align*}
Die initiale Topologie $\OO_X$ auf $X$ bezüglich der
Projektionen $p_i$,
\begin{align*}
p_i: X\to X_i,\;(x_j)_{j\in \II} \mapsto x_i,\quad i\in\II,
\end{align*}
mit $p_i$ surjektiv, heißt \emph{Produkttopologie}. Eine Basis von
$\OO_X$ besteht aus karthesischen Produkten $\prod\limits_{i\in\II}\BB_i$ mit
$\BB_i\in\OO_i$ und $\BB_i = X_i$ für fast alle $i\in\II$.\fishhere
\end{defn}
\begin{bemn}[Beachte:]
Für $i\in\II$ und $\BB_i\in\OO_i = \OO_{X_i}$ ist,
\begin{align*}
p_i^{-1}(\BB_i) = \prod\limits_{j\in\II} C_j \text{ mit }
C_j =\begin{cases}
     X_j,& \text{für } i\neq j\in \II,\\
     \BB_i, & \text{sonst}.
     \end{cases} 
\end{align*}
\begin{align*}
\prod\limits_{j\in\II} X_j \to X_i \supseteq \BB_i \in\OO_i,\; (x_j)_{j\in\II}
\mapsto x_i,
\end{align*}
\begin{align*}
p_i^{-1}(\BB_i) = \prod\limits_{j\in\II} C_j \supseteq \prod\limits_{j\in \II}
X_j \text{ mit } C_j = \begin{cases}
                       \BB_i,& i= j,\\
                       X_j, & i\neq j.
                       \end{cases}
\end{align*}
\begin{bemn}[Also:]
Die offenen Mengen in der Produkttopologie $\OO_X$ auf
$X=\prod\limits_{i\in\II} X_i$ bestehen aus Vereinigungen von endlichen
Durchschnitten solcher Teilmengen * von $X$.
%TODO: Was ist *?
\begin{align*}
\bigcap\limits_{k=1}^l A_{i_k} = A_{i_1}\cap \ldots\cap A_{i_l} =
\prod\limits_{j\in\II} D_j,
\end{align*}
mit
\begin{align*}
D_j = \begin{cases}
      \BB_{i_\nu}, & \text{ für } j=i_\nu,\\
      X_j,&\text{für } j \neq i_\nu, \nu = 1,\ldots,l.\maphere
      \end{cases}
\end{align*}
\end{bemn}
\end{bemn}
Die Produkttopologie $\OO_X$ auf $X=\prod\limits_{i\in\II} X_i$ besteht aus
Vereinigungen von Mengen der Form $\prod\limits_{i\in\II}\BB_i$ mit
$\BB_i\in\OO_i$ und $\BB_i =X_i$ für fast alle $i\in \II$.

Die Topologie $\OO_X$ auf $X = \prod\limits_{i\in\II} X_i$ mit
$B\in\tilde{\OO}_X$ $\Leftrightarrow$
%TODO: Hä warum hier ein Leftrightarrow?
$B$ ist Vereinigung von Mengen der Form $\prod\limits_{i\in\II} \BB_i$ mit
$\BB_i\in\OO_i\forall i\in \II$ ist echt feiner als die Produkttopologie auf
$X$, falls $\II$ unendlich ist.

\begin{bsp}
\label{bsp:1.4.3}
\begin{enumerate}
  \item $\R^n$ mit der natürlichen Topologe ist das topologische Produkt von $n$
  Faktoren $\R$ mit natürlicher Topologie. (Basis ist $\prod\limits_{i=1}^n O_i$
  mit $O_i$ offenem Intervall in $\R$ $\mathrel{\widehat{=}}$ inneres eines
  $n$-dimensionalen Quaders)
  \item Sei $f:\R\to\R$ Abbildung, schreibe
  $f=(f(\alpha))_{\alpha\in\R}\in\R^\R$, d.h. als Element von
  $\prod\limits_{\alpha\in\R} \R_\alpha$ mit $\R_\alpha = \R, \forall \alpha\in
  \R$.
  \begin{align*}
  \R^\R &= \setdef{f:\R\to\R}{f \text{ ist Abbildung}} \\ &= \text{kartesisches
  Produkt von $\abs{\R}$ vielen Kopien von $\R$}.
  \end{align*}
Versieht man die Faktoren $\R=\R_\alpha (\alpha\in\R)$ mit der natürlichen
Topologie, so ist die Produkttopologe auf $\R^\R$ die der punktweisen
Konvergenz.
\begin{align*}
(f\in\R^\R,\alpha\in\R, \ep > 0 : U_{\ep,\alpha}(f) &=
\setdef{g:\R\to\R}{\abs{f(\alpha)-g(\alpha)}<\ep} \\ &= p_g^{-1}(U_\ep(\alpha)))
\end{align*}
für $p_x:\R^\R \mapsto \R_\alpha,\;f\mapsto f(\alpha)$ die natürliche
Projektion.\bsphere
\end{enumerate}
\end{bsp}

\begin{prop}
\label{prop:1.4.3}
Das topologische Produkt $(X,\OO_X)$ mit $X = \prod\limits_{i\in\II} X_i$,
($X_i$ topologischer Raum mit Topologie $\OO_i, i\in\II$) und $\OO_X$
der Produkttopologie auf $X$ erfüllt zusammen mit den natürlichen Projektionen
$p_j: X= \prod\limits_{\in\in\II} X_i \to X_j$ die folgende universelle
Eigenschaft:

Sei $(Y,\OO_Y)$ topologischer Raum und $f_i: Y\to X_i, i\in\II$ seien
Abbildungen. Dann gibt es genau eine Abbildung $f: Y\to X$ mit $f_i: p_i\circ
f, \forall i\in\II$, d.h. folgendes Diagramm kommutiert:
\begin{center}
\psset{unit=0.7cm}
\begin{pspicture}(-1,0)(6,4.5)
\rput[B](0,3){\Rnode{A}{$Y$}}
\rput[B](5,3){\Rnode{B}{$X$}}
\rput[B](2.5,0.2){\Rnode{C}{$X_i$}}

\ncLine[nodesep=3pt]{->}{A}{B}
\Aput{$\exists ! f$}

\ncLine[nodesep=3pt]{->}{B}{C}
\Aput{$p_i$}

\ncLine[nodesep=3pt]{->}{A}{C}
\Bput{$f_i$}
\end{pspicture}
\end{center}
Weiter ist $f$ stetig genau dann, wenn alle $f_i = p_i\circ f$ stetig
sind.\fishhere
\end{prop}
\begin{proof}
$f: Y\to X, y\mapsto (f_i(y))_{i\in\II}\in X$ (wegen $f_i(y)\in X_i$).

Die Existenz und die Eindeutigkeit von $f$ ist die universelle Eigenschaft des
karthesischen Produkts ist gerade die universelle Eigenschaft der
Initialtopologe mit der wir hier $X$ versehen haben.
(Produkttopologie = Initialtopologe bezüglich der $p_i,i\in\II$)\qedhere
\end{proof}

Finales Gegenstück zum topologischen Produkt ist die topologische Summe.

\begin{defn}
\label{defn:1.4.4}
Seien $(X_i,\OO_i), i\in\II$ topologische Räume. Sei $X$ die disjunkte
Vereinigung $\dot{\bigcup\limits_{i\in\II}} X_i$ (z.B. indem man $X_i$ mit
$X\cup\{i\}$ identifiziert). Offene Mengen in $\tilde{X_i}$ haben die Form
$O\times \{i\}$ mit $O\in\OO_i$.

Für $i\in\II$ sei $j_i: X_i\to X = \dot{\bigcup}_{j\in \II} X_j$ die natürliche
Einbettung.

Die \emph{Summentopologie} auf $X$ ist die finale Topologie auf $X$ bezüglich
der Einbettung $j_i: X_i\to X, (i\in\II)$. $(X,\OO_X)$ heißt topologische Summe der
$X_i (i\in\II)$.

Bezeichnung: $(X,\OO_X) = \sum\limits_{i\in\II} (X_i,\OO_i)$.

Beachte: Für $i\in\II$ und $U\subseteq X$ ist $j_i^{-1}(U) = U\cap X_i$. Daher
ist nach Definition \ref{defn:1.3.2} der finalen Topologie eine Menge
$U\subseteq X$ offen in $(X,\OO_X)$ genau dann, wenn $j_i^{-1}(U) = U\cap
X_i\in\OO_i, \forall i\in\II$.\fishhere
\end{defn}

\begin{prop}
\label{prop:1.4.5}
Sei $(X,\OO_X)$ topologische Summe der topologischen
Räume $(X_i,\OO_i)$. Dann sind die Inklusionen $j_i: X_i\to X$ stetig und offen
und $(X,O)$ hat die folgende universelle Eigenschaft:
\begin{enumerate}
  \item Jede Familie $f_i: X_i\to Y$ ($Y$ Menge) von Abbildungen induziert
  eindeutig eine Abbildung $f: X\to Y$ mit $f_i = f\circ j_i$:
\begin{center}
\psset{unit=0.7cm}
\begin{pspicture}(-1,0)(6,4.5)
\rput[B](0,3){\Rnode{A}{$X$}}
\rput[B](5,3){\Rnode{B}{$Y$}}
\rput[B](2.5,0.2){\Rnode{C}{$X_i$}}

\ncLine[nodesep=3pt]{->}{A}{B}
\Aput{$\exists ! f$}

\ncLine[nodesep=3pt]{->}{C}{B}
\Aput{$f_i$}

\ncLine[nodesep=3pt]{->}{C}{A}
\Bput{$p_i$}
\end{pspicture}
\end{center}
  \item $f$ ist stetig genau dann, wenn $f_i = f\circ j_i$ stetig ist $(f: X\to
  Y, x\mapsto f_i(x) \Leftrightarrow x\in X_i \subseteq X)$. So ist klar warum
  man die Vereinigung braucht, sonst wäre $X$ nicht eindeutig
  platziert.\fishhere
\end{enumerate}
\end{prop}
\begin{proof}
Übung.\qedhere
\end{proof}

Die Summentopologie ist weniger interessant als die Produkttopologie, da sich
für sie kaum allgemeine Sätze beweisen lassen.\footnote{Entweder sind diese
trivial oder falsch.}
\begin{bemn}[Pholosophische Bemerkung.]
In der topologischen Summe stehen die Teile $X_i$ einfach nebeneinander, $X$
ist ``unzusammenhängend'' (später). Erst wenn die Teile $X_i$ geeignet
``zusammengenäht'' werden (\emph{Surgery}), d.h. an geeigneten Stellen
identifiziert und dann zum 2. Mal die finale Topologie auf dem Quotienten
(Quotiententopologie) genommen wird, entstehen interessante topologische
Räume.\maphere
\end{bemn}


\subsection{Exkurs: Analysis ohne Metrik}
Wir wollen kurz darauf eingehen, wie man den Begriff der Konvergenz in der
Analysis topologisch begründen kann. Nicht alle Formen von Konvergenz lassen
sich mithilfe von Metriken ausdrücken. Beispielsweise für die punktweise
Konvergenz einer Funktion in einem Funktionenraum benötigen wir ein
allgemeineres Konzept. Dieses liefert uns die Topologe.

\begin{lem}[Definition/Lemma]
\label{lem:1.5.1}
\begin{enumerate}
  \item Sei $X$ Menge, ein Mengensystem $\varnothing\neq \FF \subseteq \PP(X)$
  heißt \emph{Filter}, falls es folgenden Eigenschaften genügt:
  \begin{enumerate}[label=F\arabic{*})]
    \item $\varnothing\notin \FF$.
    \item $U_1,U_2\in\FF\Rightarrow U_1\cap U_2\in \FF$.\\
     $\FF$ ist abgeschlossen gegenüber endlichen Durchschnitten.
    \item $U\in \FF$ und $U\subseteq V\subseteq X \Rightarrow V\in \FF$.\\
    Jede Obermenge von $U$ ist in $\FF$ enthalten.  
\end{enumerate}
\begin{bspn}
$(X,\OO_X)$ topologischer Raum, $z\in X$, dann ist das System $\UU_z$ ein Filter
auf $X$.\bsphere
\end{bspn}
\item Ein Mengensystem $\varnothing \neq \BB \subseteq \PP(X)$
heißt \emph{Filterbasis}, falls gilt:
\begin{enumerate}[label=FB\arabic{*})]
  \item $\varnothing\notin \BB$,
  \item $B_1,B_2\in\BB \Rightarrow \exists V\in\BB : V\subseteq B_1\cap B_2$.\\
Endliche Durchschnitte von Mengen in $\BB$ enthalten eine Menge aus $\BB$.
\end{enumerate}
\item Ist $B\subseteq\PP(X)$ Filterbasis, so ist
\begin{align*}
\FF = \setdef{V\subseteq X}{\exists B\in \BB : B\subseteq V},
\end{align*}
Filter auf $X$, der \emph{von $\BB$ erzeugte Filter}. ($\FF$ beseteht aus der
Menge der Obermengen von Elementen aus $\BB$).\fishhere
\end{enumerate}
\end{lem}
\begin{proof}
Klar!\qedhere
\end{proof}

\addtocounter{prop}{1}

\begin{defn}
\label{defn:1.5.3}
Ein Filter $\FF$ auf dem topologischen Raum $X$ \emph{konvergiert} gegen den
Punkt $z\in X$, falls $\FF$ feiner ist als der Umgebungsfilter $\UU_z$ von $z$,
d.h. $\UU_z\subseteq \FF$.

Insbesondere konvergiert $\UU_z\to z$.\fishhere
\end{defn}

\begin{bsp}
\label{1.5.4}
\begin{enumerate}
  \item Sei $(X,\OO_X)$ ein topologischer Raum und sei $(a_n)_{n\in\N}$ ein
  Folge in $X$. Für $m\in\N$ sei $M_m = \setdef{a_n}{n\ge m}$ das \emph{$m$-te
  Endstück} der Folge,
  \begin{align*}
  \MM = \setdef{M_m}{m\in\N}\subseteq \PP(X).
  \end{align*}
Dann ist $\varnothing\neq \MM$, $\varnothing\notin\MM$ und $\MM$
ist abgeschlossen über endlichen Durchschnitten, denn
\begin{align*}
(M_m\cap M_n) = M_{\max \{m,n\}}
\end{align*}
und erfüllt daher FB1 und FB2, d.h. $\MM$ erzeugt einen Filter,
\begin{align*}
\FF = \setdef{U\subseteq X}{\exists m\in\N : M_m \subseteq U}.
\end{align*}
Diesen Filter nennt man den \emph{Endstückfilter} der Folge $(a_n)_{n\in\N}$.

\begin{propn}[Problem]
Sei $X\subseteq \R$ mit natürlicher Topologie. Dann konvergiert der
Endstückfilter einer Folge $(a_n)_{n\in\N}$ in $\R$ genau dann gegen $x\in\R$,
wenn $(a_n)_{n\in\N}$ konvergiert mit $\lim\limits_{n\to\infty} a_n =
x$.\fishhere
\end{propn}

\begin{defn}
In $(X,\OO_X)$ \emph{konvergiert eine Folge} $(a_n)_{n\in\N}$ in $X$ gegen ein
$x\in X$ genau dann, wenn der Endstückfilter der Folge $(a_n)_{n\in\N}$ gegen $x$
konvergiert.\fishhere
\end{defn}
\item In der diskreten Topologie besteht der Umgebungsfilter von $x\in X$ aus
allen Umgebungen von $x$, in der indiskreten Topologie nur aus $X$ selbst.

In der indiskreten Topologie konvergiert daher jede Folge gegen jeden Punkt von
$X$. In der diskreten Topologie konvergieren nur Umgebungsfilter.
\item Sei $\varnothing\neq A\subseteq X$, so ist $\{A\}$ eine Filterbasis. Der
davon erzeugte Filter besteht aus allen Obermengen von $A$. Er wird
\emph{Hauptfilter} von $A$ genannt. Im Allgemeinen konvergiert er nicht, (aber
z.B. für $A= \{X\},\;x\in X$.)
\item Es kann durchaus vorkommen, dass ein Filter gegen mehrere Punkte eines
topologischen Raumes konvergiert. (z.B. in der indiskreten Topologie)
\item Sei $X$ topologischer Raum, dann ist $\FF = \setdef{A\subseteq
X}{X\setminus A}$ mit $X\setminus A$ endlich bzw. abzählbar ein
Filter, falls $X$ unendlich bzw. überabzählbar ist. Ob er konvergiert hängt von
der Topologie ab.\bsphere
\end{enumerate}
\end{bsp}

\begin{bemn}[Bemerkung zu $\R^\R$.]
Man kann $\R^\R$ als Menge von Folgen $(x_\alpha)_{\alpha\in\R}$
identifizieren. Mit der Produkttopologie erhalten wir die Umgebungen
\begin{align*}
U_{\ep,x} = \setdef{g:\R\to\R}{\abs{f(x)-g(x)}<\ep},
\end{align*}
für $\ep > 0, x\in\R$. Es gibt keine Metrik auf $\R^\R$, die diese Umgebungen so
erzeugen könnte. Die Umgebungen sind die der punktweisen Konvergenz.\maphere
\end{bemn}